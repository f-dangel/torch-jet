\documentclass[a0paper]{tikzposter}
% \geometry{paperwidth=46.8in,paperheight=33.1in}
\geometry{paperwidth=60.0in,paperheight=36.0in}

\usepackage{amsmath}
\usepackage{siunitx}
\usepackage{booktabs}
\usepackage{makecell} % table cells with multiple lines of text
\usepackage{multirow}
\usepackage{postertheme/vector_institute/vector_institute}

\definecolor{colorbrewer-green}{RGB}{27, 158, 119}
\colorlet{tab-green}{colorbrewer-green}
\definecolor{colorbrewer-orange}{RGB}{217, 95, 2}
\colorlet{tab-orange}{colorbrewer-orange}
\definecolor{colorbrewer-blue}{RGB}{117, 112, 179}
\colorlet{tab-blue}{colorbrewer-blue}

\input{../paper/preamble/goodfellow.tex}
\usetikzlibrary{calc,positioning}

% \useblockstyle{Envelope}
% \usebackgroundstyle{Empty}
\usetitlestyle{Filled}

\title{
  \fontsize{180}{198}\selectfont
  \bf Collapsing Taylor Mode Automatic Differentiation
}
\author{
  \fontsize{60}{66}\selectfont
  Felix Dangel*, Tim Siebert*, Marius Zeinhofer, Andrea Walther
}
\institute{
  \fontsize{50}{55}\selectfont
  Vector Institute (Canada),
  Humboldt University \& Zuse Institute Berlin (Germany),
  ETH Z\"urich (Switzerland)
}

% Make the most-used fonts bigger
\renewcommand{\normalsize}{\fontsize{40}{44}\selectfont}
\renewcommand{\large}{\fontsize{45}{49.5}\selectfont}
\renewcommand{\Large}{\fontsize{50}{55}\selectfont}
\renewcommand{\LARGE}{\fontsize{55}{60.5}\selectfont}


\begin{document}
% ==============================================================================
% HEADER & FOOTER

\backgroundgradient % Adds the background features
\maketitle
\headerlogo % Adds Vector logo to header
% \posterfooter{Poster footer with additional information} % Footer


\hspace{-35cm}
\begin{columns}
  \centering
  \begin{column}{1.85}
    \centering \ribbon{\centering\fontsize{120}{80}\selectfont\textcolor{white}{
        \hfill \includegraphics[width=9.75cm]{logos/qrcode}\\[-4.4ex]
        \bf
        \hspace{-5ex}\underline{TL;DR:}\quad We accelerate Taylor mode for practical differential operators.
        \\[0.75ex]
        \hspace{1.5ex}\underline{\phantom{;\!}How:}\quad By linearity-based graph rewrites that collapse the propagation.\qquad}}
  \end{column}
\end{columns}
\hspace{2cm}

\begin{columns}
  \column{0.598}
  \block{Background: What is Taylor Mode Autodiff?}{

    \vspace{0.5ex}
    \begin{center}
      \textbf{Taylor mode generalizes forward mode to higher-order derivatives.}
    \end{center}

    \vspace{2ex}
    \begin{center}
      \includegraphics[width=0.9\linewidth]{figures/taylor}
    \end{center}

    \vspace{1.75ex}

    Fa\`a di-Bruno's formula describes the propagation of Taylor coefficients (jets). Choosing different Taylor coefficients computes different derivatives.

    \vspace{1.5ex}
    \begin{center}
      \textbf{Only JAX implements Taylor mode. We implement it for PyTorch.}
      \vspace{0.5ex}

      \texttt{pip install jet-for-pytorch}
    \end{center}

    \vspace{1.5ex}

    \begin{center}
      \textbf{For PDE operators, we typically need to run Taylor mode along multiple directions, then sum the resulting derivatives.}
    \end{center}

    \vspace{1ex}

    \hspace*{-0.1cm}
    \begin{tikzpicture}
      \node [inner xsep=20pt, inner ysep=30pt, draw=tab-orange, rounded corners=15pt, line width=10pt] (graph-before) {\includegraphics[height=12.0cm]{figures/sin_2jet_0}};
      \coordinate (separator) at ($(graph-before.north east)!0.7!(graph-before.south east)$);
      \draw[tab-orange, fill=tab-orange, line width=5pt, rounded corners=15pt] (graph-before.south west) rectangle (separator);
      % Re-draw
      \node [inner xsep=20pt, inner ysep=30pt, draw=tab-orange, rounded corners=15pt, line width=10pt] {\includegraphics[height=12.0cm]{figures/sin_2jet_0}};
      \node [anchor=north east, align=center, yshift=6.5cm] at (separator) {\textbf{\textcolor{tab-orange}{Propagate then sum}}};
    \end{tikzpicture}
  }

  \column{0.598}

  \block{}{
    \begin{minipage}[c]{0.5\linewidth}
      \centering
      \LARGE \bfseries
      Interested in similar topics?

      Felix is hiring for 2026!

      % (Concordia University \& Mila, Montreal)
    \end{minipage}
    \begin{minipage}[c]{0.49\linewidth}
      \includegraphics[height=3cm]{logos/logo-concordia.pdf}
      % [trim={left bottom right top},clip]
      \includegraphics[trim={6cm 8cm 6cm 8cm},clip, height=3cm]{logos/logo-mila.png}

    \end{minipage}
  }

  \block{Main Idea: Use Linearity to Collapse the Propagation}{

    \vspace{0.5ex}

    \begin{center}
      \textbf{Linearity is nice because order does not matter (associativity).}
    \end{center}
    \vspace{2.25ex}

    \hspace{1.5ex}
    \begin{tikzpicture}
    \node[anchor=west, draw=VectorBlack, minimum width=0.92\linewidth, rounded corners=15pt, line width=7pt, align=center]{
      \begin{minipage}[t]{0.89\linewidth}
        \centering
      \vspace{0.5ex}
      \textbf{Quiz:}
      Given a matrix $\mW$ and vectors $\{\vx_n\}$, what is better? \\[0.75ex]
      \scalebox{1.25}{%
      $
      \begin{array}{c c c}
        \displaystyle
        \color{tab-orange}
        \sum_n \left( \mW \vx_n \right)
        & \qquad\text{or}\qquad
        & \color{tab-green}
          \displaystyle \mW \Big( \sum_n \vx_n \Big)
        \\
        \color{tab-orange}
        \text{\normalsize \textbf{(act then sum)}}
        &
        &
          \color{tab-green}
          \text{\normalsize \textbf{(sum then act)}}
      \end{array}
      $}
      \end{minipage}
    };
    \end{tikzpicture}


    \vspace{3ex}

    \textbf{Our insight:} Fa\`a di-Bruno's formula exhibits linearity in the sum of highest Taylor components.
    We can choose the better order and propagate the summed Taylor coefficient.
    \textbf{We call this \textcolor{tab-green}{`collapsing' Taylor mode}.}

    \vspace{2.5ex}

    \textbf{Collapsing reduces FLOPs, memory, and can be automated (e.g.\,in \texttt{jit}).}

    \vspace{1ex}
    \hspace*{-0.4cm}
    \begin{tikzpicture}
      \node [inner xsep=20pt, inner ysep=30pt, draw=tab-green, rounded corners=15pt, line width=10pt] (graph-after) {\includegraphics[height=13.0cm]{figures/sin_2jet_3}};
      \coordinate (separator) at ($(graph-after.north east)!0.7!(graph-after.south east)$);
      \draw [tab-green, line width=5pt, rounded corners=15pt, fill=tab-green] (graph-after.south west) rectangle (separator);
      % Re-draw
      \node [inner xsep=20pt, inner ysep=30pt, draw=tab-green, rounded corners=15pt, line width=10pt] {\includegraphics[height=13.0cm]{figures/sin_2jet_3}};
      \node [anchor=north east, align=center, yshift=6.5cm] at (separator) {\textbf{\textcolor{tab-green}{Sum then propagate}}};
    \end{tikzpicture}
  }


  \column{0.598}
  \block{Scope \& Performance}{
    \begin{minipage}[t]{0.4\linewidth}
      \vspace{0.5ex}
      \begin{center}
        \textbf{Collapsed Taylor mode contains and improves existing methods.}
      \end{center}

      \vspace{1.5ex}

      \begin{itemize}
      \item \textcolor{tab-orange}{\textbf{Laplacian}}
        \\
        {\color{tab-green} $\Leftrightarrow$ \textbf{forward Laplacian}}
        \\[0.1ex]

      \item \textcolor{tab-orange}{\textbf{Stochastic Taylor mode}}
        \\
        {\color{tab-green}$\to$ \textbf{collapses samples}}
        \\[0.1ex]

      \item \textcolor{tab-orange}{\textbf{General linear PDE ops}}
        \\
        {\color{tab-green}$\to$ \textbf{collapses interpolation}}
      \end{itemize}
      \vspace{3.4ex}
      \begin{center}
        \textbf{Collapsing improves vanilla Taylor mode and outperforms nesting  first-order autodiff.}
      \end{center}

    \end{minipage}
    \hfill
    \begin{minipage}[t]{0.55\linewidth}
      \centering
      \vspace*{-0.25ex}
      \includegraphics[width=0.92\linewidth]{figures/overview.pdf}
    \end{minipage}
    \begin{center}
    \end{center}

    \vspace{1.9ex}

    \begin{center}
      \input{tables/benchmark}
    \end{center}
  }
\end{columns}

\end{document}
%%% Local Variables:
%%% mode: LaTeX
%%% TeX-master: t
%%% End:

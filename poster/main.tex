\documentclass[a0paper]{tikzposter}
% Force a 16 by 9 ratio, 33.1 is same height as a0
\geometry{paperwidth=58.844in,paperheight=33.1in}

\usepackage{amsmath}
\usepackage{postertheme/vector_institute/vector_institute}

\input{../paper/preamble/goodfellow.tex}

% \useblockstyle{Envelope}
% \usebackgroundstyle{Empty}
\usetitlestyle{Filled}

\title{
  \fontsize{180}{90}\selectfont
  \bf Collapsing Taylor Mode Automatic Differentiation
}
\author{\Huge
  Felix Dangel*, Tim Siebert*, Marius Zeinhofer, Andrea Walther
}
\institute{
  \LARGE
  Vector Institute (Canada), TODO (Germany), ETH Z\"urich)
}


\begin{document}
% ==============================================================================
% HEADER & FOOTER

\backgroundgradient % Adds the background features
\maketitle
\headerlogo % Adds Vector logo to header
\posterfooter{Poster footer with additional information} % Footer

\hspace{-35cm}
\begin{columns}
  \centering
  \begin{column}{1.85}
    \centering \ribbon{\centering\fontsize{120}{80}\selectfont\textcolor{white}{\bf
        \underline{TL;DR:}\quad We accelerate Taylor mode for practically relevant differential operators. \\[0.5ex]
        \hspace{-9.75ex}\underline{\phantom{;\!}How:}\quad By simple graph rewrites that collapse Taylor coefficients.}}
  \end{column}
\end{columns}
\hspace{2cm}

\begin{columns}
  \column{0.598}
  \block{Background: Taylor Mode}{
    \begin{minipage}{0.63\linewidth}
      TODO
    \end{minipage}
    \hfill
    \begin{minipage}{0.33\linewidth}
      \centering
      \includegraphics[width=\linewidth]{example-image-a}
    \end{minipage}

    \begin{center}
      \centering
      Figure of compute graph

      \includegraphics[width=0.5\linewidth]{example-image-a}
    \end{center}

    FYI: So far, only JAX implemented Taylor mode.

    Our paper introduces a PyTorch library for Taylor mode.
    \begin{center}
      \texttt{pip install jet-for-pytorch}
    \end{center}
  }

  \column{0.598}
  \block{Main idea: Use linearity to collapse computations}{
    What is better?
    \begin{align*}
      \mW \left( \sum_n \vx_n \right)
      \intertext{or}
      \sum_n \left( \mW \vx_n \right)
    \end{align*}

    Applies to:
    \begin{itemize}
    \item Laplacian $\to$ forward Laplacian
    \item Randomized Taylor mode (e.g. stochastic Laplacian estimator)
    \item General linear PDE operators (e.g. Bilaplacian)
    \end{itemize}
  }

  \column{0.598}
  \block{Practical Improvements}{

    Some experiments here.

    \vspace{5cm}

    {\Huge \bfseries Want to work on similar topics? Felix is hiring!}


  }
\end{columns}

\end{document}
%%% Local Variables:
%%% mode: LaTeX
%%% TeX-master: t
%%% End:

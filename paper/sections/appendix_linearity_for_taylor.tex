 Here, we present the schematic idea behind propagating $R$ $K$-jets through $\vf = \vg \circ \vh$ with starting jets $\left(J^K\vx\right)_r(t) = \sum_{k=0}^K \frac{t^k}{k!} \vx_{k,r}$.
The following Taylor mode AD scheme results from inserting \cref{eq:sum-k-directional} into the \cref{eq:taylor-mode-composition}:
\begin{align}\label{eq:sum-taylor-mode-naive}
  \begin{split}
      \begin{pmatrix*}
        \vx_0
        \\
       \{\vx_{1,r}\}
        \\
        \{\vx_{2,r}\}
        \\
        \vdots
        \\
        \{\vx_{K, r}\}
      \end{pmatrix*}
      &\overset{\text{(\ref{eq:faa-di-bruno})}}{\to}
        \begin{pmatrix*}[l]
          \vh_0 =  \vh(\vx_0)
          \\
          \{\vh_{1,r}\} =  
          \left\{
          \left< 
          \partial \vh(\vx_0),
          \vx_{1, r} 
          \right>
          \right\}
          \\
          \{\vh_{2, r}\} = 
          \left\{
          \left< 
          \partial^2 \vh(\vx_0),
          \vx_{1, r} \otimes \vx_{1, r} 
          \right>
          +
          \left<
          \partial \vh(\vx_0),
          \vx_{2, r}
          \right>
          \right\}
          \\
          \vdots
          \\
          \{\vh_{K, r}\} =
          \left\{
          \displaystyle \sum_{
          \sigma \in \partitioning(K)
          }
          \nu(\sigma) \left<
          \partial^{|\sigma|} \vh(\vx_0),
          \tensorprod{s \in \sigma} \vx_{s, r}
          \right>
          \right\}
        \end{pmatrix*} 
        \\
        &\overset{\text{(\ref{eq:faa-di-bruno})}}{\to}
        \begin{pmatrix*}[l]
          \vg_0 =  \vg(\vh_0)
          \\
          \{\vg_{1,r}\} = 
          \left\{
          \left< 
          \partial \vg(\vh_0), 
          \vh_{1, r}
          \right>
          \right\}
          \\
          \{\vg_{2,r}\} = 
          \left\{
          \left<
          \partial^2 \vg(\vh_0), 
          \vh_{1, r} \otimes \vh_{1, r}\right>
          + 
          \left< \partial \vg(\vh_0),
          \vh_{2, r}
          \right>
          \right\}
          \\
          \vdots
          \\
          \{\vg_{K,r}\} =
          \left\{
          \displaystyle\sum_{
          \sigma \in \partitioning(K)
          } 
          \nu(\sigma) \left<
          \partial^{|\sigma|} \vg(\vh_0),
          \tensorprod{s \in \sigma} \vh_{s, r}
          \right>
          \right\}
        \end{pmatrix*}
        \\
        &\overset{\text{(\ref{eq:faa-di-bruno})}}{=}
        \begin{pmatrix*}[l]
          \vf_0 =  \vf(\vx_0)
          \\
          \{\vf_{1,r}\} = \left\{
          \left<
          \partial \vf(\vx_0), 
          \vx_{1, r}
          \right>
          \right\}
          \\
          \{\vf_{2,r}\} = \left\{
          \left<
          \partial^2 \vf(\vx_0), 
          \vx_{1,r} \otimes \vx_{1, r}
          \right>
          + 
          \left< \partial \vf(\vx_0),
          \vx_{2, r}
          \right>
          \right\}
          \\
          \vdots
          \\
          \{\vf_{K, r}\} =
          \left\{
          \displaystyle\sum_{
          \sigma \in \partitioning(K)
          }
          \nu(\sigma) \left< 
          \partial^{|\sigma|} \vf(\vx_0),
          \tensorprod{s \in \sigma} \vx_{s, r}
          \right>
          \right\}
        \end{pmatrix*}
        \\
        &\overset{\text{slice}}{\to} \{ \vg_{K,r} \}
        \\
        &\overset{\text{sum}}{\to} \sum_{r=1}^R \vg_{K,r} 
          \overset{\{\vx_{1,r} = \vv_n, \vx_{2,r} = \vzero, \dots, \vx_{K, r} = \vzero \}}{=} 
          \sum_{n=r}^R \left<
          \partial^K \vf(\vx_0), 
          \otimes_{k=1}^K \vv_r
          \right>
    \end{split}
\end{align}
Leveraging linearity in the certain terms (in green) of the highest coefficient instead, leads to
\begin{align}\label{eq:sum-taylor-mode-efficient}
  \begin{split}
      \begin{pmatrix*}
        \vx_0
        \\
       \{\vx_{1,r}\}
        \\
        \{\vx_{2,r}\}
        \\
        \vdots
        \\
        \textcolor{\colorcTM}{\displaystyle\sum_{r = 1}^R\vx_{K, r}}
      \end{pmatrix*}
      &\overset{\text{(\ref{eq:faa-di-bruno})}}{\to}
        \begin{pmatrix*}[l]
          \vh_0 &=  \vh(\vx_0)
          \\
          \{\vh_{1,r}\} &=  
          \left\{
          \left< 
          \partial \vh(\vx_0),
          \vx_{1, r} 
          \right>
          \right\}
          \\
          \{\vh_{2, r}\} &= 
          \left\{
          \left< 
          \partial^2 \vh(\vx_0),
          \vx_{1, r} \otimes \vx_{1, r} 
          \right>
          +
          \left<
          \partial \vh(\vx_0),
          \vx_{2, r}
          \right>
          \right\}
          \\
          \vdots
          \\
          \textcolor{\colorcTM}{ \displaystyle \sum_{r = 1}R \vh_{K, r}} &=
          \displaystyle \sum_{r=1}^R \sum_{
          \sigma \in \partitioning(K) \setminus \{\sigma_t\}
          }
          \nu(\sigma) \left<
          \partial^{|\sigma|} \vh(\vx_0),
          \tensorprod{s \in \sigma} \vx_{s, r}
          \right>
          \\
          &+ 
          \left<
          \partial \vh(\vx_0),
          \textcolor{\colorcTM}{
          \sum_{n=1}^N \vx_{K, n}
          }
          \right>
        \end{pmatrix*} 
        \\
        &\overset{\text{(\ref{eq:faa-di-bruno})}}{\to}
        \begin{pmatrix*}[l]
          \vg_0 &=  \vg(\vh_0)
          \\
          \{\vg_{1,n}\} &= 
          \left\{
          \left< 
          \partial \vg(\vh_0), 
          \vh_{1, n}
          \right>
          \right\}
          \\
          \{\vg_{2,n}\} &= 
          \left\{
          \left<
          \partial^2 \vg(\vh_0), 
          \vh_{1, n} \otimes \vh_{1, n}\right>
          + 
          \left< \partial \vg(\vh_0),
          \vh_{2, n}
          \right>
          \right\}
          \\
          \vdots
          \\
          \textcolor{\colorcTM}{\displaystyle\sum_{n=1}^N\vg_{K,n}} &=
          \displaystyle \sum_{n=1}^N \sum_{
          \sigma \in \partitioning(K) \setminus \{\sigma_t\}
          } 
          \nu(\sigma) \left<
          \partial^{|\sigma|} \vg(\vh_0),
          \tensorprod{s \in \sigma} \vh_{s, n}
          \right>
          \\
          &+
          \left<
          \partial \vg(\vh_0),
          \textcolor{\colorcTM}{\displaystyle \sum_{n=1}^N\vh_{K, n}}
          \right>
        \end{pmatrix*}
        \\
        &\overset{\text{(\ref{eq:faa-di-bruno})}}{=}
        \begin{pmatrix*}[l]
          \vf_0 &=  \vf(\vx_0)
          \\
          \{\vf_{1,n}\} &= \left\{
          \left<
          \partial \vf(\vx_0), 
          \vx_{1, n}
          \right>
          \right\}
          \\
          \{\vf_{2,n}\} &= \left\{
          \left<
          \partial^2 \vf(\vx_0), 
          \vx_{1, n} \otimes \vx_{1, n}
          \right>
          + 
          \left< \partial \vf(\vx_0),
          \vx_{2, n}
          \right>
          \right\}
          \\
          \vdots
          \\
          \textcolor{\colorcTM}{\displaystyle \sum_{n=1}^N \vf_{K, n}} &=
          \displaystyle \sum_{n=1}^N\sum_{
          \sigma \in \partitioning(K) \setminus \{ \sigma_t \}
          }
          \nu(\sigma) \left< 
          \partial^{|\sigma|} \vf(\vx_0),
          \tensorprod{s \in \sigma} \vx_{s, n}
          \right> 
          \\
          &+
          \left<
          \partial \vf(\vx_0),
          \textcolor{\colorcTM}{\displaystyle \sum_{n=1}^N\vx_{K, n}}
          \right>
        \end{pmatrix*}
        \\
        &\overset{\text{slice}}{\to} \textcolor{\colorcTM}{\sum_{n=1}^N \vg_{K,n}}
          \overset{\{\vx_{1,n} = \vv_n, \vx_{2,n} = \vzero, \dots, \vx_{K, n} = \vzero \}}{=} 
          \sum_{n=1}^N \left<
          \partial^K \vf(\vx_0), 
          \otimes_{k=1}^K \vv_n
          \right>
    \end{split}
\end{align}

\subsection{Examples -- Laplace Operator}

\paragraph{Naive Laplacian}
The propagation scheme below is obtained, by applying \cref{eq:sum-taylor-mode-naive} to \cref{eq:laplacian}.
\begin{align}\label{eq:laplacian-naive}
  \begin{split}
    \begin{pmatrix*}
      \vx_0
      \\
      \{\vx_{1,d} \}
      \\
      \{\vx_{2,d} \}
    \end{pmatrix*}
    &\overset{\text{(\ref{eq:taylor-mode-scalar})}}{\to}
      \begin{pmatrix*}[l]
        \vh_0 =  \vh(\vx_0)
        \\
        \{\vh_{1,d}\} = \{
        \left<
        \partial \vh(\vx_0),
        \vx_{1,d}
        \right>
        \}
        \\
        \{\vh_{2,d}\} = 
        \{
        \left<
        \partial^2 \vh(\vx_0),
        \vx_{1,d} \otimes \vx_{1,d} 
        \right>
        + 
        \left<
        \partial \vh(\vx_0),
        \vx_{2,d}
        \right>
        \}
      \end{pmatrix*}
    \\
    &\overset{\text{(\ref{eq:taylor-mode-scalar})}}{\to}
      \begin{pmatrix*}[l]
        g_0 =  g(\vh_0)
        \\
        \{g_{1,d}\} = \{
        \left<\partial g(\vh_0), 
        \vh_{1,d}
        \right>
        \}
        \\
        \{g_{2,d}\} = \{
        \left< 
        \partial^2 g(\vh_0),
        \vh_{1,d} \otimes \vh_{1,d} 
        \right>
        + 
        \left<
        \partial g(\vh_0), 
        \vh_{2,d}
        \right>
        \}
      \end{pmatrix*}
      \\
      &\overset{\text{(\ref{eq:taylor-mode-scalar})}}{=}
      \begin{pmatrix*}[l]
        f_0 =  \vf(\vx_0)
        \\
        \{f_{1,d} \} = \{
        \left<
        \partial f(\vx_0),
        \vx_{1,d}
        \right>
        \}
        \\
        \{ f_{2,d} \} = \{
        \left<
        \partial^2 f(\vx_0),
        \vx_{1,d} \otimes \vx_{1,d}
        \right>
        + 
        \left<
        \partial f(\vx_0), 
        \vx_{2,d}
        \right>
        \}
      \end{pmatrix*}
    \\
    &\overset{\text{slice}}{\to} \{ g_{2,d} \}
    \\
    &\overset{\text{sum}}{\to} \sum_{d=1}^D \{ g_{2,d} \}
      \overset{\{\vx_{1,d} = \ve_d, \vx_{2,d} = \vzero\}}{=} \Delta f(\vx_0)
  \end{split}
\end{align}


\paragraph{Efficient Laplacian}
Instead of naively applying \cref{eq:sum-taylor-mode-naive}, \cref{eq:sum-taylor-mode-efficient} is used to obtain a efficient Laplacian propagation approach:
\begin{align}\label{eq:laplacian-efficient}
  \begin{split}
    \begin{pmatrix*}
      \vx_0
      \\
      \{\vx_{1,d} \}
      \\
      \textcolor{\colorcTM}{\displaystyle\sum_{d=1}^D \vx_{2,d}}
    \end{pmatrix*}
    &\overset{\text{(\ref{eq:taylor-mode-scalar})}}{\to}
      \begin{pmatrix*}[l]
        \vh_0 =  \vh(\vx_0)
        \\
        \{\vh_{1,d}\} = \{
        \left<
        \partial \vh(\vx_0),
        \vx_{1,d}
        \right>
        \}
        \\
        \textcolor{\colorcTM}{\displaystyle\sum_{d=1}^D \vh_{2,d}} = \displaystyle\sum_{d=1}^D 
        \left< \partial^2 \vh(\vx_0), 
        \vx_{1,d} \otimes \vx_{1,d} 
        \right> 
        + 
        \left<
        \partial \vh(\vx_0),
        \textcolor{\colorcTM}{\displaystyle\sum_{d=1}^D\vx_{2,d}}
        \right>
      \end{pmatrix*}
    \\
    &\overset{\text{(\ref{eq:taylor-mode-scalar})}}{\to}
      \begin{pmatrix*}[l]
        \vg_0 =  \vg(\vh_0)
        \\
        \{\vg_{1,d}\} = \{
        \left<
        \partial \vg(\vh_0), 
        \vh_{1,d}
        \right>
        \}
        \\
        \textcolor{\colorcTM}{\displaystyle\sum_{d=1}^D\vg_{2,d}} 
        = 
        \displaystyle\sum_{d=1}^D
        \left< 
        \partial^2 \vg(\vh_0),
        \vh_{1,d} \otimes \vh_{1,d}
        \right>
        + 
        \left< 
        \partial \vg(\vh_0), 
        \textcolor{\colorcTM}{\displaystyle\sum_{d=1}^D\vh_{2,d}}
        \right>
      \end{pmatrix*} 
      \\
      &\overset{\text{(\ref{eq:taylor-mode-scalar})}}{=}
      \begin{pmatrix*}[l]
        \vf_0 =  \vf(\vx_0)
        \\
        \{\vf_{1,d} \} = \{ 
        \left<
        \partial \vf(\vx_0),
        \vx_{1,d}
        \right>
        \}
        \\
        \textcolor{\colorcTM}{\displaystyle\sum_{d=1}^D \vf_{2,d}}
        =
        \displaystyle\sum_{d=1}^D 
        \left<
        \partial^2 \vf(\vx_0),
        \vx_{1,d} \otimes \vx_{1,d}
        \right>
        +
        \left<
        \partial \vf(\vx_0),
        \textcolor{\colorcTM}{\displaystyle\sum_{i=1}^D\vx_{2,d}}
        \right>
      \end{pmatrix*}
    \\
    &\overset{\text{slice}}{\to} \textcolor{\colorcTM}{\sum_{d=1}^D \{ \vg_{2,d} \}}
      \overset{\{(\vx_{1,d} = \ve_d, \vx_{2,d} = \vzero)\}}{=}
      \Delta f(\vx_0)
  \end{split}
\end{align}
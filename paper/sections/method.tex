In this section, we exploit the linearity in the $K$th coefficients when propagating a family of $K$-jets through a computational graph. Such families of $K$-jets were already considered for the Forward-Laplace \cite{li2023forward}, Randomized Taylor-Mode ~\cite{shi2024stochastic, hu2023hutchinson} and the generalized trace operator\todo{T: Cite the papers that also considered this, and maybe an application?}. Afterwards, we show that differential operators with a suitable structure can be computed via the propagation of a appropriately chosen family of jets using an algorithm of Griewank et al. \cite{griewank_evaluating_1999}. The result will also include the approach of \cite{shi2024stochastic} for computing arbitrary differential operators. However, our approach only needs jets of the highest derivative degree occurring in the differential operator. In addition, it has to be emphasized that our proposed method can be completely automated, meaning that it can be implemented as part of a JIT-toolchain. \todo{T: BTW the automation is currently just a claim of me. I did not fully think through the idea of automating the TTC formula. }


\subsection{Exploiting Linearity in Taylor-mode AD}
To derive the proposed method, the considerations start with a sum over $K$th-order directional derivatives of a function $\vf$ like the Laplacian. In our notation, this sum can be represented as 
\begin{align}\label{eq:sum-k-directional}
  \sum_{n=1}^N\left< 
  \partial^{K} f(\vx_0),
  \otimes_{k=1}^K \vv_n
  \right>.
\end{align}
For the Laplacian, one has  $K = 2$, $N$ equals the dimension of $\vx$, i.e., $N= D$, and the vectors are chosen as the unit vectors $\vv_n = \ve_n$.
Instead of nesting derivative calculations to compute the full $K$-th order derivative tensor, $K$-jets can be utilized to efficiently calculate those derivatives. To this end, one
could propagate a family of $K$-jets as illustrated above for a simple example through the whole computational using the Taylor-mode of AD. %, see Equation~\cref{eq:sum-taylor-mode-naive}. 
\todo{AW: Do we need Equation~\cref{eq:sum-taylor-mode-naive}? This is just one more step of propagation through the computation graph and the sum over the entries?
T: I think it is helpful to exactly know what and how we want to compute something, but I'm open to discuss it} 
Note, that it is crucial to select appropriate input coefficients $\vx_i$. For example, one could initialize the $n$-th $K$-jet with $\vx_{0, n} = \vv_n$ and $\vx_{k, n} = \vzero\in\R^D$ if the differential operator is given by the sum of the pure $K$-th derivatives with respect to the components of $\vx$. In \cref{eq:sum-taylor-mode-naive} we selected the $n$-th $K$-Jet with $\vx_{0, n} = \vv_n$ and $\vx_{k, n} = \vzero$.

Without further optimizations, the Taylor-mode of AD( \cref{eq:sum-taylor-mode-naive}) propagates $1 + KN$ through every node of the computational graph. Subsequently, one would 
have to sum the corresponding highest derivatives. \todo{AW: I think we have to be careful to state the complexity precisely. Therefore, I would prefer the formulation as stated here, i.e., X vectors aof length Y, or we state the number of coefficients propagated. WIther way is fine for me, we should just clearly state is.
T: From my perspective, it is already really clear. We propagate 1 + NK coefficients through every node. The size of the coefficients itself depends on the actual node. From my point of view it would not help to introduce a dimension for the nodes, it might even be to complicated then. But of course it only my point of view! Open to discuss.}

The approach we propose here is based on the observation that there is a special element in the set of integer partitions $\partitioning(K)$, namely the trivial partition $\sigma_{t} = \{K\}$ corresponding, e.g., to the term
\begin{equation}
    \nu(\sigma_{t}) \left<
    \partial \vg(\vh_0),
    \vh_{K, n}
    \right>, 
\end{equation}
of the Faà di Bruno formula. Using the definitions of \cite{hardy2006combinatorics}, one obtains $\nu(\sigma_{t}) = 1$. \todo{AW: I see the $t$ for trivial, however we also have $t$ as argument where the jet is evaluated, I would prefer a different notation here.
T: What do you think about $\sigma_{tr}$? Since we use this sigma just here, its fine for me to even call it $\sigma_{trivial}$
}
Therefore, one has
\begin{equation}\label{eq:faa-di-bruno-expanded}
     \vf_{K, n} 
     = 
     \vg_{K,n} 
     = 
     \displaystyle\sum_{
      \sigma \in \partitioning(K) \setminus \{\sigma_t\}
     } 
     \nu(\sigma) \left<
     \partial^{|\sigma|} \vg(\vh_0),
     \tensorprod{s \in \sigma} \vh_{K, n}
     \right>
     +
     \left<
     \partial \vg(\vh_0),
     \vh_{K, n}
     \right>.
\end{equation}
The last inner-product term on the right-hand side is linear in $\vh_{K,n}$, marking the key property for our method. Hence, to compute the sum of $\{\vg_{K, n}\}_n$ and thus \cref{eq:sum-k-directional}, instead of passing the highest coefficients, like $\{\vh_{K,n}\}_n$, separately along the nodes of the computational graph, it is possible to directly propagate their sum like $\sum_n \vh_{K,n}$, due to the linearity of $\left<\partial \vh, \bullet \right>$ and $\left< \partial \vg, \bullet \right>$. Combining this insight with the Taylor-mode of AD  yields our proposed collapsed Taylor-mode as detailed in the appendix, see \cref{eq:sum-taylor-mode-efficient}  where the changes are highlighted in \textcolor{maincolor}{color}. This scheme propagates only $1 + (K - 1)N + 1$ vectors at every node in the computational graph.
To emphasize that the saving of $N-1$ coefficients at every node in the propagation scheme is indeed significant for common applications, important operators are discussed below. 

\subsection{Examples}
A highlight of our general approach \cref{eq:sum-k-directional} is that it naturally includes common important differential operators or their approximation.

\paragraph{Laplace Operator} The Laplace operator plays a central role in many physical and engineering applications, including electrostatics, fluid dynamics, and heat conduction. In the context of variational Monte Carlo simulations in quantum mechanics \cite{foulkes2001quantum, pfau2020ab}, the Laplacian of the wavefunction must be evaluated repeatedly as part of the wavefunction optimization process. For a function $f: \sR^D \to \sR$ the Laplacian is given as
\begin{align}\label{eq:laplacian}
  \Delta f(\vx_0)
  :=
  \sum_{d=1}^D \frac{\partial^2 f(\vx_0)}{\partial x_d^2}
  =
  \sum_{d=1}^D \left< 
  \partial^2 f(\vx_0),
  \ve_d \otimes \ve_d
  \right>
\end{align}
Assume that $f$ is again decomposed into $g \circ \vh$, where $\vh$ is vector valued, as is the case if $\vh$ is a hidden layer of a neural network. Selecting $K = 2$, $N=D$ and $\vv_n = \ve_n$ lifts \cref{eq:laplacian} into the setting of \cref{eq:sum-k-directional}.
Naively applying the scheme \cref{eq:sum-taylor-mode-efficient} would lead to $1 + 2D$ (see \cref{eq:laplacian-naive}). If we instead apply our proposed efficient scheme \cref{eq:sum-taylor-mode-efficient}, only $1 + D + 1$ vectors are propagated. The scheme is presented in \cref{eq:laplacian-efficient} (changes highlighted in \textcolor{maincolor}{color})

Note that this scheme is aligned with the findings of \cite{li2023forward}, but here it is embedded into a general framework.



\paragraph{Weighted sums of second-order derivatives.} %\todo{We can make the connection \cite{hu2023hutchinson} equation (5) clear and discuss the case of $\mC$ being a function of the input} 

A natural generalization of the Laplace operator involves forming weighted sums of second derivatives. Given a positive semi-definite matrix $\boldsymbol{C}\in\mathbb R^{D\times D}$,  we consider the operator
\begin{align}\label{eq:weighted-laplacian}
  \sum_{i,j} \boldsymbol{C}_{i,j} \frac{\partial^2 \vf(\vx_0)}{\partial x_i \partial x_j}.
\end{align}
This expression represents, for example, the diffusion term in a Kolmogorov-type partial differential equation \cite{hu2023hutchinson}. It is often equivalently written as $\Tr ( \msigma \msigma^\top \partial^2 \vf(\vx_0))$, where $\msigma$ may be a function $\msigma:\mathbb R^{D}\to \mathbb R^{D\times D}$. Since $\boldsymbol{C}$ is positive semi-definite it can be expressed as $\mC = \sum_{n=1}^{\rank(\mC)} \vc_n \vc_n^\top$ for suitable $\vc_n \in \sR^D$. This matches the setting of \cref{eq:sum-k-directional} with $K = 2, N = \rank(\mC)$ and $\vv_n = \vc_n$, and can be computed using \cref{eq:sum-taylor-mode-efficient}. This requires the propagation of $1 + \rank(\mC) + 1$ vectors instead of $1 + 2\rank(\mC)$.
An important special case of \cref{eq:weighted-laplacian} is the Hutchinson Trace Estimator for the Laplacian \cite{hu2023hutchinson}, which is also computable like the Forward-Laplacian scheme \cite{li2023forward}, leveraging the collapsed Taylor-mode. 

%Another important differential operator is represented by a symmetric positive semi-definite (PSD) matrix $\boldsymbol{C} \in \left(\mathbb{R}^D\right)^{\otimes 2}$:
%\begin{align}\label{eq:weighted-laplacian}
%  \sum_{i,j} \boldsymbol{C}_{i,j} \frac{\partial^2 \vf(\vx_0)}{\partial x_i \partial x_j}.
%\end{align}
%Those operators occur, for example, as 
%\begin{equation*}
%    \Tr (\msigma \msigma^\top \partial^2 \vf(\vx_0)
%\end{equation*}
%in Fokker-Planck and Hamiltonian-Jacobi Bellmann equations. \todo{need citations} Since $\boldsymbol{C}$ is PSD it can be expressed as $\mC = \sum_{n=1}^{\rank(\mC)} \vc_n \vc_n^\top$ for suitable $\vc_n \in \sR^D$. Therefore, it fits the setting of \cref{eq:sum-k-directional} with $K = 2, N = \rank(\mC)$ and $\vv_n = \vc_n$, and can be compute using \cref{eq:sum-taylor-mode-efficient}. This requires the propagation of $1 + \rank(\mC) + 1$ vectors instead of $1 + 2\rank(\mC)$.


\paragraph{Traces of higher-order derivative tensors}
\todo{cite papers that consider this. M: I am not so sure where this opeartor appears naturally} The higher-order Laplacian, e.g., considered as theoretical example in \cite{shi2024stochastic}, or as application in ...  is also naturally expressible in our framework:
\begin{align}\label{eq:trace-differential-operator}
    \Tr( \partial^K f(\vx) )
    \coloneqq
    \sum_{d=1}^D
    \frac{\partial^K f(\vx_0)}{\partial \evx_d^K}
    =
    \sum_{d=1}^D
    \left< 
    \partial^K f(\vx_0),
    \otimes_{i=1}^K \ve_d
    \right>,
\end{align}
with $N = D$ and $\vv_n = \ve_n$. t

\subsection{Operators with mixed-partial Derivative}
Despite being already general enough to unify various approaches for the computation of important differential operators, up to now our framework cannot handle operators that include arbitrary mixed-partial derivatives.
To tackle this problem, we extend our framework to transform a differential operator containing arbitrary mixed-partial derivatives into a combination of Jets using the result of Griewank et al. \cite{griewank_evaluating_1999}.

\todo{T: Working on it.}
To extend our method, we start with a generalized version of \cref{eq:sum-k-directional}:
\begin{align}\label{eq:sums-k-directional}
  \sum_{n_1=1}^{N_1} \dots \sum_{n_K=1}^{N_K}\left< 
  \partial^{K} f(\vx_0),
  \otimes_{k=1}^K \vv_{n_k}
  \right>
  =
  \sum_{k = 1}^K \left. \frac{d^K}{dz^K}
  f(\vx_0 + \mV_k \vz) \right|_{\vz = \vzero}.
\end{align}
The equation includes operators like the Biharmonic Operator (see \cref{sec:example2}). In general, the independent summation over the direction vectors $\vv_{n_k}$ is assumed. For distinct $\vv_{n_k}$, the summands are not directly representable by $K$-Jets and thus, not computable via collapsed Taylor-mode. 
The technique of Griewank et al. to reconstruct a mixed-partial of degree $K$ from a family of $K$-Jets has been summarized in the formula

$\vv_k$ coincide such that we have only $P \leq K$ different direction vectors

$\vi = (p_1, \dots, p_P)$ numbers of how often every direction occur.

\begin{equation}
\label{eq:ttc_general}
    \left< 
    \partial^{K} f(\vx_0),
    \otimes_{k=1}^K \vv_{k}
    \right>    
    = 
    \sum_{\underset{\vl \in \mathbb{N}^P}{|\vl| = |\vi|}}
    \gamma_{\vi \vl}
    \frac{1}{\vi!}
    \left<
    \partial^{K}\vf,
    \otimes_{k=1}^K
    \sum_{p=1}^P\vv_p l_p
    \right>.
\end{equation}


\cref{sec:appendix_ttc} provides supplementary details to the formula. From \cref{eq:ttc_general} we conclude
\begin{align} \label{eq:ttc_general_operator}
  \sum_{n_1=1}^{N_1} \dots \sum_{n_K=1}^{N_K}\left< 
  \partial^{K} f(\vx_0),
  \otimes_{k=1}^K \vv_{n_k}
  \right>
  = 
  
  \sum_{n=1}^N \sum_{\underset{\vl \in \mathbb{N}^p}{|\vl| = |\vk_n|}}
    \gamma_{\vk_n \vl}
    \frac{1}{\vk_n!}
    \left<
    \partial^{|\vk_n|}\vf,
    \otimes_{l=1}^{|\vk_n|}
    \boldsymbol{S}\vl
    \right>.
\end{align}
Therefore, a general differential operator with arbitrary mixed-partial derivatives $\mA\vf(\vx_0)$ is expressible in our setting \cref{eq:sum-k-directional}. Thus, for every $\mA$ we can apply our efficient propagation scheme and have to propagate the different $|\vk_n|$-Jets with directions $\mS \vl$ \todo{maybe be a bit more precise here} instead of depending on expensive nested derivative calls. 

As the inventors of the formula already recognized, the coefficients $\gamma_{\vk\vl}$ only depend on the problem structure and are independent of the function $\vf$. As we will show, this allows to exploit symmetry that is often inherent to differential operators.
\todo{are there symmetries directly exploitable?}

\subsection{Examples} \label{sec:example2}

\paragraph{Biharmonic Operator}
The Biharmonic Operator is a fourth-order differential operator with mixed-partial derivatives that is interesting for some practitioners. \todo{cite}
It is defined as
\begin{align}
\label{eq:biharm}
    \Delta^2 \vf(\vx)
    \coloneqq
    \sum_{i=1}^D \sum_{j=1}^D
    \frac{\partial^4 \vf(\vx)}{\partial \evx_i^2 \partial \evx_j^2}
    =
    \sum_{i=1}^D \sum_{j=1}^D
    \left<
    \partial^4 \vf(\vx),
    \ve_i \otimes \ve_i \otimes \ve_j \otimes \ve_j
    \right>.
\end{align}

The classical way to compute such derivatives is to nest all four derivatives, which grows exponentially in run-time and memory with the derivative order. A comparison with this approach is given in \cref{sec:experiments}.

Another way is naturally given by considering the Biharmonic Operator as the second-order Laplacian. Then, \cref{eq:laplacian-naive} could be applied twice. Instead of propagating fourth-order derivative tensors, holding $\left( \begin{matrix} D + 3 \\ 4 \end{matrix} \right) \in O(D^4)$ distinct elements, the twice-applied Laplacian would have to push $(2 + D) + D * (2 + D) + D * (2 + D) = 2D^2 + 5D + 2$ vectors.
Using \cref{eq:laplacian-efficient} we reduce this further to $(2 + D) + D * (2 + D) + 1 * (2 + D) = D^2 + 4D + 5$ vectors. 

A different approach was proposed in \cite{shi2024stochastic}, which relies on the hand-selection of $6$-Jets to extract the searched derivatives. See the appendix for more details: \cref{sec:biharm_felix_approach}. 

\todo{compare with \cite{hu2023hutchinson}}

Our approach first applies \cref{eq:ttc_general} to the Biharmonic Operator \cref{eq:biharm}. To this end, parameters $\vk, p$, auxiliary variables $\vz$, and matrices $\boldsymbol{S}_{ij}$ have to be selected for all occurring mixed-partial derivatives. The parameters are grouped into the case where all four derivatives directions coincide, i.e, $i = j$ and the case where $i \neq j$. The selected parameters are depicted in \cref{tab:params_ttc_biharm}. 

\begin{table}[]
    \centering
    \begin{tabular}{|c|c|c|c|c|}
\toprule
             & $p$ & $\vk$    & $\vz$        & $\boldsymbol{S}_{ij}$  \\
\midrule
     $i = j$ & $1$ & $(4)$    & $(z_1)$      & $(\ve_i)$                \\
\midrule
  $i \neq j$ & $2$ & $(2, 2)$ & $(z_1, z_2)$ & $\left( \ve_i \; \ve_j \right)$ \\
\bottomrule
    \end{tabular}
    \caption{Parameters to apply \cref{eq:ttc_general} to the Biharmonic Operator \cref{eq:biharm}.}
    \label{tab:params_ttc_biharm}
\end{table}


Having defined the parameters, \cref{eq:ttc_general} is utilized to map the Biharmonic Operator \cref{eq:biharm} of a function $\vf$ to a collection of fourth-order jets: \todo{check coefficients} 
\begin{align}
    \Delta^2 \vf(\vx)
    &=
    \sum_{i=1}^D
    \gamma_{(4)(4)}
    \frac{1}{4!}
    \left<
    \partial^{4}\vf(\vx_0),
    \otimes_{i=1}^4 4 \ve_i
    \right>
    \\
    &+
    \sum_{i=1}^D \sum_{\underset{j \neq i}{j=1}}^D
    \sum_{\underset{\vl \in \mathbb{N}^2}{|\vl| = 4}}
    \gamma_{(2, 2) \vl}
    \frac{1}{(2, 2)!}
    \left<
    \partial^{4} \vf(\vx_0),
    \otimes_{i=1}^4
    \boldsymbol{S}_{ij} \vl
    \right>.
\end{align}
As mentioned before, the coefficients $\gamma_{\vk \vl}$ capture the symmetric structure of the differential operator. A close look into the definition \cref{eq:ttc_coeff} shows the equality of $\vl = (4,0)$ with $\vl=(0, 4)$ and $\vl = (3, 1)$ with $\vl = (3, 1)$. Exploiting those symmetries the formula boils down to \todo{think we need 1/4!}
\begin{align}
   \Delta^2 \vf(\vx)
   &=
    \left(
   \gamma_{(4)(4)} + 2 (D - 1) \gamma_{(2,2)(4, 0)}
   \right)
   \sum_{i=1}^D
    \left< \partial^4 \vf(\vx_0),
    \otimes_{i=1}^4
    4\ve_i
    \right>
    \\
    &+
    2 \gamma_{(2, 2)(3, 1)} 
    \sum_{i=1}^D 
    \sum_{\underset{j \neq i}{j=1}}^D
    \left< \partial^4 \vf(\vx_0),
    \otimes_{i=1}^4 3 \ve_i+ \ve_j
    \right>
    \\
    &+ 
    \gamma_{(2, 2)(2, 2)}
    \sum_{i=1}^D 
    \sum_{\underset{j \neq i}{j=1}}^D
    \left< \partial^4 \vf(\vx_0),
    \otimes_{i=1}^4
    2 \ve_i + 2 \ve_j
    \right>.
\end{align}
\Cref{tab:ttc_biharm_coeffs} lists the required coefficients.

\begin{table}[]
    \centering
    \begin{tabular}{|c|c|c|c|}
        \toprule
        $\gamma_{(4)(4)}$ & $\gamma_{(2, 2) (4, 0)}$ & $\gamma_{(2, 2) (3, 1)}$ & $\gamma_{(2, 2) (2, 2)}$ \\
        \midrule
        $\frac{3}{32}$ & $\frac{13}{192}$ &  -$\frac{1}{3}$ & $\frac{5}{8}$  \\
        \bottomrule
    \end{tabular}
    \caption{Coefficients to compute the Biharmonic Operator by \Crefrange{eq:ttc_biharm_full1}{eq:ttc_biharm_full3}.}
    \label{tab:ttc_biharm_coeffs}
\end{table}

The last term can be further simplified by leveraging the symmetry of the direction $2 \ve_i + 2 \ve_j$. This simplification leads to the following efficient propagation scheme for \cref{eq:biharm}:
\begin{align}
   \Delta^2 \vf(\vx)
   &= \label{eq:ttc_biharm_full1}
    \left(
   \gamma_{(4)(4)} + 2 (D - 1) \gamma_{(2,2)(4, 0)}
   \right)
   \sum_{i=1}^D
    \left<\partial^4 \vf(\vx_0),
    \otimes_{i=1}^4 4\ve_i
    \right>
    \\
    &+ \label{eq:ttc_biharm_full2}
    2 \gamma_{(2, 2)(3, 1)} 
    \sum_{i=1}^D 
    \sum_{\underset{j \neq i}{j=1}}^D
    \left< \partial^4 \vf(\vx_0), 
    \otimes_{i=1}^4 3 \ve_i + \ve_j
    \right>
    \\
    &+ \label{eq:ttc_biharm_full3}
    2 \gamma_{(2, 2)(2, 2)}
    \sum_{i=1}^{D-1} 
    \sum_{j=i+1}^D
    \left< \partial^4 \vf(\vx_0),
    \otimes_{i=1}^4 2 \ve_i + 2 \ve_j
    \right>.
\end{align} 

The \Crefrange{eq:ttc_biharm_full1}{eq:ttc_biharm_full3} are computed through $D + D(D-1) + \frac{D(D-1)}{2}$ fourth-order jets. These fourth-order jets exhibit the structure required to apply our proposed graph simplifications, which makes the computations even more efficient. \todo{align with higher order laplacian} Thus, the summation can be pulled inside the highest coefficients in the propagation scheme, like for the Laplacian. From this we conclude that our propagation scheme for the Biharmonic Operator has the effort of propagating $1 + 3D + 1 + 1 + 3 \cdot D(D-1) + 1 + 1 + 3 \frac{D(D-1)}{2} + 1 = 9\frac{D^2}{2} - 3\frac{D}{2} + 6$ vectors. It should be noted, that all jets are independent and can be computed in parallel, which provides further benefits for the run-time complexity.

\todo{in comparison to laplace over laplace, we are worse in theory?}




\paragraph{[Not done] Stochastic Taylor Derivative Estimator}
\todo{consider this for mixed-partials, like estimator of biharmonic, maybe mention this also in weighted sum of derivatives}
\cite{hu2023hutchinson} also considered biharmonic!
The approach of \citep{shi2024stochastic} lacks the possibility to handle differential operators with arbitrary mixed-partials. With our framework however, it is easy to transform a differential operator with mixed-partials into a collection of Jets. Therefore, the randomization approach of their work is automatically applicable in our framework. 

The sampled directions to approximate a differential operator 
This lowers the computational cost when $D$ is prohibitively large for exact evaluation of differential operators like the Laplacian.
Take the weighted Laplacian from \Cref{eq:weighted-laplacian} and assume we have access to random vectors $\rvv$ with and $\E[\vv \vv^{\top}] = \mC$.
Then we can draw $S \ll D$ random vectors $\vv_1, \vv_2, \dots, \vv_S \overset{\text{i.i.d.}}{\sim} \rvv$ and compute
\begin{align}
  \begin{split}
    \E[\partial^2f[\rvv, \rvv]]
    &=
      \E \left[
      \rvv^{\top} \frac{\partial^2 f}{\partial \vx \partial \vx} \rvv
      \right]
      =
      % \E \left[
      %   \sum_{i,j} \ervv_i \frac{\partial^2 f}{\partial x_i \partial x_j} \ervv_j
      % \right]
      % =
      \sum_{i,j} \frac{\partial^2 f}{\partial x_i \partial x_j}
      \E \left[
      \ervv_i  \ervv_j
      \right]
      =
      \sum_{i,j} \frac{\partial^2 f}{\partial x_i \partial x_j} C_{i,j}
    \\
    &\approx
      \frac{1}{S}
      \sum_{s=1}^S
      \partial^2 f[\vv_s, \vv_s]\,.
  \end{split}
\end{align}
Again, we see that the second-order coefficients can be collapsed when setting $\vx_{1,d} = \vv_d, \vx_{2,d} = \vzero$ in \cref{eq:laplacian-naive,eq:laplacian-efficient}.

Here, we make the connection of collapsed Taylor mode to the forward Laplacian \cite{li2023forward} and the randomized estimation of the Laplacian via Hutchinson's trace estimator \cite{hutchinson1989stochastic} from \cite{shi2024stochastic} explicit.
For simplicity, we consider a vector-to-scalar function $f: \sR^{D} \to \sR, \vx \mapsto f(\vx)$ (the general vector-to-vector case is straightforward but requires more notation) whose Laplacian is
\begin{align*}
  \Delta f(\vx) =
  \Tr( \nabla^2 f(\vx) )
  =
  \sum_{d=1}^{D} [\nabla^2 f(\vx)]_{d,d}
  =
  \sum_{d=1}^{D} \ve_d^{\top} \nabla^2f(\vx) \ve_d\,,
\end{align*}
with $\nabla^2 f(\vx) \in \sR^{D \times D}$ the Hessian of $f$ evaluated at $\vx$.

\paragraph{Connection to randomized Laplacian via Hutchinson's trace estimator.}
Because the Laplacian can be expressed as trace of the Hessian, we can use Hutchinson's trace estimator \cite{hutchinson1989stochastic} to estimate it via Hessian-vector products with random vectors.
Specifically, for any matrix $\mA \in \sR^{D \times D}$ and a distribution $p(\rvv)$ over a vector $\rvv$ with unit covariance ($\E[\rvv \rvv^{\top}] = \mI_D$) we can use the cyclic property of the trace and linearity of the expectation to write
\begin{align*}
  \Tr(\mA)
  =
  \Tr(\mA \mI_D)
  =
  \Tr( \mA \E[\rvv \rvv^{\top}])
  =
  \E[\Tr( \mA \rvv \rvv^{\top})]
  =
  \E[\Tr( \rvv^{\top} \mA \rvv)]\,.
\end{align*}
Then, we can compute an unbiased estimate of the right hand side by drawing $S$ vectors $\vv_1, \vv_2, \dots, \vv_S \sim p(\rvv)$ and evaluating the Monte-Carlo estimator
\begin{align*}
  \Tr(\mA)
  \approx
  \frac{1}{S} \sum_{s=1}^{S} \vv_s^{\top} \mA \vv_s\,.
\end{align*}
Applied to the Hessian, we can estimate the Laplacian of $f$ as
\begin{align*}
  \Delta f(\vx)
  \approx&
           \frac{1}{S} \sum_{s=1}^{S} \vv_s^{\top} \nabla^2 f(\vx) \vv_s\,.
           \shortintertext{Using our tensor notation, we can rewrite this into a sum of terms involving $\left\langle \partial^2f(\vx), \vv_s^{\otimes 2} \right\rangle$, which can be computed with \textcolor{tab-orange}{\textbf{vanilla Taylor mode}} using $S$ 2-jets:}
           =&
              \frac{1}{S} \sum_{s=1}^{S} \sum_{i,j=1}^D [\partial^2f(\vx)]_{i,j} [\vv_s]_i [\vv_s]_j
              =
              \frac{1}{S} \sum_{s=1}^{S} \left\langle \partial^2f(\vx), \vv_s^{\otimes 2} \right\rangle\,.
              \shortintertext{Instead of propagating then summing the 2-jets, we can also sum the vectors and then propagate the sum, which is our proposed \textcolor{tab-green}{\textbf{collapsed Taylor mode}}:}
              =& \frac{1}{S} \left\langle \partial^2f(\vx), \sum_{s=1}^{S} \vv_s^{\otimes 2} \right\rangle\,.
\end{align*}

\paragraph{Connection to the forward Laplacian.} TODO

%%% Local Variables:
%%% mode: LaTeX
%%% TeX-master: "../main"
%%% End:

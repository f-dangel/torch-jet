\documentclass{article}

% use numbers for citations to save space
\PassOptionsToPackage{numbers, compress}{natbib}

% either empty (for submission), 'preprint', or 'final'
\def\status{}
\usepackage[\status]{neurips_2024}

\input{preamble/custom_early.tex}
\input{preamble/neurips_2024.tex}
% follow DL notation from the Goodfellow book
\input{preamble/goodfellow.tex}
% ===================================================================
% MATH
% ===================================================================
\usepackage{nicefrac} % fractions that fit into inline text
\usepackage{dsfont} % for \mathds command
\usepackage[%
exponent-product=\ensuremath{\cdot},%
group-minimum-digits={3}%
]{siunitx} % \num command for pretty-formatting large numbers
\newcommand{\mathemph}[1]{{\color{maincolor} #1}}

% ===================================================================
% REFERENCES
% ===================================================================
\usepackage{cleveref} % automatically adds type of reference, MUST BE LOADED AFTER AMSMATH
\crefname{section}{\S\!\!}{\S\!\!} % use paragraph symbol for Section
\crefname{appendix}{\S\!\!}{\S\!\!} % use paragraph symbol for Appendix

% ===================================================================
% TODOS, COMMENTS & WRITING
% ===================================================================
\usepackage{todonotes} % for TODOs, loads xcolor with []
\usepackage{comment} % for comment environment
\usepackage{xspace}
\newcommand*{\ie}{i.e.\@\xspace}
\newcommand*{\iid}{i.i.d.\@\xspace}
\newcommand*{\wrt}{w.r.t.\@\xspace}
\newcommand*{\eg}{e.g.\@\xspace}
\newcommand*{\Ie}{I.e.\@\xspace}
\newcommand*{\Eg}{E.g.\@\xspace}

% ===================================================================
% FIGURES & COLORS
% ===================================================================
\usepackage{wrapfig} % side-wrap text next to a figure
\usepackage{subcaption} % \subfigure environment
\usepackage{tabularx} % tables with automatic line break
\captionsetup[subfigure]{% subfigure captions are left-aligned
  justification=justified,%
  singlelinecheck=false,%
}%
\usepackage{tikz} % for drawings in LaTeX
\usetikzlibrary{
  arrows.meta, % for prettier arrows
  matrix, % for matrix of nodes
}

% VECTOR INSTITUTE PRIMARY COLORS
\definecolor{VectorBlack}{RGB}{34, 34, 34}
\definecolor{VectorGray}{RGB}{239, 238, 237}
% VECTOR INSTITUTE SECONDARY COLORS
\definecolor{VectorBlue}{RGB}{59, 69, 227}
\definecolor{VectorPink}{RGB}{253, 8, 238}
\definecolor{VectorOrange}{RGB}{250, 173, 26}
\definecolor{VectorTeal}{RGB}{82, 199, 222}

% PAPER COLOR THEME
\colorlet{maincolor}{VectorBlue}
\colorlet{secondcolor}{VectorPink}
\colorlet{thirdcolor}{VectorOrange}
\colorlet{fourthcolor}{VectorTeal}
\colorlet{fifthcolor}{VectorGray}

% MATPLOTLIB COLORS
\definecolor{tab:orange}{rgb}{1.0, 0.498, 0.055}
\definecolor{tab:blue}{rgb}{0.121, 0.466, 0.705}
\definecolor{tab:green}{rgb}{0.173, 0.627, 0.173}

% ===================================================================
% LINKS & REFERENCES
% ===================================================================
\hypersetup{%
  colorlinks,
  citecolor = maincolor,%
  linkcolor = maincolor,%
  urlcolor = secondcolor,%
}%

% ===================================================================
% SPECIAL SYMBOLS
% ===================================================================
\usepackage{pifont} % for check and cross marks
% commands from https://tex.stackexchange.com/a/42620
\newcommand{\cmark}{\ding{51}}
\newcommand{\xmark}{\ding{55}}

% ===================================================================
% ALGORITHMS
% ===================================================================
\usepackage{algorithm}
\usepackage{algpseudocode}

% ===================================================================
% TABLES
% ===================================================================
\usepackage{multirow}
\usepackage{array} % vertically centered table cells
\usepackage{makecell} % table cells with multiple lines of text

%%% Local Variables:
%%% mode: latex
%%% TeX-master: "../main"
%%% End:

\newcommand{\papertitle}{%
  Accelerating Differential Operators Through Linearity
}
\title{\papertitle}

% The \author macro works with any number of authors. There are two commands
% used to separate the names and addresses of multiple authors: \And and \AND.
%
% Using \And between authors leaves it to LaTeX to determine where to break the
% lines. Using \AND forces a line break at that point. So, if LaTeX puts 3 of 4
% authors names on the first line, and the last on the second line, try using
% \AND instead of \And before the third author name.

\author{%
  Felix Dangel\thanks{Equal contribution}\\
  Vector Institute \\
  Toronto \\ Canada \\
  \texttt{fdangel@vectorinstitute.ai} \\
  \And
  Marius Zeinhofer\\
  Seminar for Applied Mathematics, ETH Z\"urich, \\
  \texttt{marius.zeinhofer@uniklinik-freiburg.de}
}
%%% Local Variables:
%%% mode: latex
%%% TeX-master: "../main"
%%% End:


\begin{document}

\maketitle

\begin{abstract}
  We explore automating the acceleration of differential operators through compute graph simplifications based on the concept of linearity.
  These occur in common differential operators like the Laplacian, that computes then sums diagonal elements of the Hessian using Taylor mode automatic differentiation (\texttt{jet}s).
  Instead, we show that the Taylor coefficients can first be summed, then propagated, which reduces computational cost.
  Due to the simplicity of this simplification (propagating a sum up a computation graph), we argue it could (or should) be performed by the just-in-time (\texttt{jit}) compiler in machine learning frameworks.
  Our preliminary experiments achieve promising, fully automated, speed-ups, which we believe can easily be integrated into automatic differentiation libraries.
\end{abstract}

\section{Introduction}\label{sec:introduction}
\todo{Biharmonic Operator instead of Bi-laplace!}\todo{Make clear that we consider PDE operators that consist of sums}Using neural networks to learn functions constrained by physical laws is a popular trend in scientific machine learning \cite{carleo2017solving, pfau2020ab, hermann2020deep, karniadakis2021physics, raissi2019physics, hu2023hutchinson, sun2020global}.
Typically, the physics is encoded through partial differential equations (PDEs) that the neural net must satisfy.
The associated loss functions require evaluating differential operators \wrt the network's input, rather than weights.
Evaluating differential operators remains a computational challenge, especially if they contain high-order derivatives.

\paragraph{Computing PDE operators.} Two important fields that require the evaluation of PDE operators are variational Monte-Carlo (VMC) simulations and physics-informed neural networks (PINNs).
VMC employs neural networks as an ansatz for the Schr\"odinger equation \cite{carleo2017solving, pfau2020ab, hermann2020deep} and demands computing the net's Laplacian, i.e., the Hessian trace usually in a non-differentiable fashion.
PINNs represent PDE solutions as a neural network and train the network via minimizing the residuals of the governing equations \cite{raissi2019physics, karniadakis2021physics}. For instance, Kolmogorov-type equations—including the Fokker-Planck and Black-Scholes equation—requires the differentiable evaluation of weighted second-order derivatives on high-dimensional spatial domains \cite{hu2023hutchinson, sun2024dynamical}. For PINN applications in elasticity, the Biharmonic operator \cite{vahab_physics-informed_2022} requires the computation of fourth-order derivatives, making it a commonly considered a benchmarking problem for computing high-order derivatives in PINNs \cite{hu2023hutchinson, vikas_biharm, shi2024stochastic}.

\paragraph{Is backpropagation all we need?}
% Alternative: The gap between theory and practice
Although nesting first-order automatic differentiation (AD) to compute high-order derivatives scales poorly, this approach is common practice.
\todo{Felix, M: still needs citation.}
A promising alternative to nested backpropagation, especially for higher-order derivatives is \emph{Taylor-mode AD}~\cite[or simply \emph{Taylor-mode},][\S13]{griewank2008evaluating}, introduced to the machine learning community in \citeyear{bettencourt2019taylor} and the JAX ecosystem in \cite{bradbury2018jax}.
However, we observe empirically that vanilla Taylor-mode is often not enough to improve upon nested backpropagation: Evaluating the Laplacian of a 5-layer $\tanh$ activated MLP using JAX' \emph{Taylor mode is 50\% slower} than nested backpropagation via computing, then tracing, the Hessian via Hessian-vector products \cite{pearlmutter1994fast,dagreou2024how}. This calls into question the relevance of Taylor mode for computing common PDE operators.

\paragraph{Improved Taylor mode schemes}
However, recent works have successfully demonstrated the potential of modified forward propagation schemes.
For the (weighted) Laplacian, \citet{li2023forward, li2024dof} developed a special forward propagation framework called the \emph{forward Laplacian}.
Relating to our example above, JAX' forward Laplacian \cite{gao2023folx} is roughly twice as fast as nested backpropagation at reduced memory costs.
While the forward Laplacian does not rely on Taylor mode, recent work pointed out a connection \cite{dangel2024kroneckerfactored}; however, it remains unclear if efficient forward schemes can be derived for other differential operators. Another line of work concerns the stochastic approximation of differential operators in high-dimensions \citet{shi2024stochastic}, relying on Taylor mode with suitably sampled random directions. In this work we identify a mechanism to rewrite the computational graph of standard Taylor mode, applicable to general PDE operators and stochastic Taylor mode: Precisely, our contributions are:
\todo{Felix: Make it more explicit that we evaluate Taylor mode in multiple directions and then sum the results. The sum can be pulled inside.}

\begin{figure*}[!t]
  \centering
  \begin{minipage}[b]{0.42\linewidth}
    \centering
    \input{figures/vanilla_taylor_not_enough.tex}

    \caption{\textbf{$\blacktriangle$ Vanilla Taylor mode is not enough to beat nested 1\textsuperscript{st}-order AD.}
      Illustrated for computing the Laplacian of a $\mathrm{tanh}$-activated $50 \!\to\! 768 \!\to\! 768 \!\to\! 512 \!\to\! 512 \!\to\! 1$
      MLP with JAX (+ \texttt{jit}) on GPU (details in \Cref{sec:jax-benchmark}).
      We show how to automatically obtain the specialized forward Laplacian through simple graph transformations of vanilla Taylor mode.
    }\label{fig:vanilla-taylor-not-enough}

    \vspace{0.25ex}
    \caption{\textbf{$\blacktriangleright$ Our collapsed Taylor mode directly propagates the sum of highest degree coefficients.}
      Visualized for propagating four $K$-jets through a $\sR^5 \!\to\! \sR^3 \!\to\! \sR$ function ($K=2$ yields the forward Laplacian).
      \Cref{sec:background} introduces the notation.}\label{fig:visual-abstract}
  \end{minipage}
  \hfill
  \begin{minipage}[b]{0.57\linewidth}
    \centering
    \newcommand{\drawgridrectangle}[4]{%
  \begin{tikzpicture}[scale=#4]
    \pgfmathsetmacro{\ymax}{#1}
    \pgfmathsetmacro{\xmax}{#2}

    % Fill the rectangle
    \fill[#3] (0,0) rectangle (\xmax,\ymax);

    % Draw the border
    \draw[white, line width=#4*3pt] (0,0) rectangle (\xmax,\ymax);


    % Draw vertical grid lines
    \pgfmathsetmacro{\xsteps}{#2}
    \foreach \x in {1,...,\xsteps} {
      \draw[white, line width=#4*3pt] (\x,0) -- (\x,\ymax);
    }

    % Draw horizontal grid lines
    \pgfmathsetmacro{\ysteps}{#1}
    \foreach \y in {1,...,\ysteps} {
      \draw[white, line width=#4*3pt] (0,\y) -- (\xmax,\y);
    }
  \end{tikzpicture}%
}

\newsavebox{\taylorStandard}
\savebox{\taylorStandard}{
  \begin{tikzpicture}
    \matrix [%
    matrix of nodes,%
    ampersand replacement=\&,% to use inside a savebox
    nodes={anchor=center, align=center},%
    column sep=4ex,%
    row sep=1ex,%
    ] (taylor)
    {
      \drawgridrectangle{1}{3}{blue!30}{0.33} \& \drawgridrectangle{1}{2}{blue!30}{0.33} \& \drawgridrectangle{1}{1}{blue!30}{0.33}
      \\[-1.5ex]
      $\vx_0$ \& $\vh_0$ \& $\vg_0$
      \\
      \drawgridrectangle{3}{3}{green!30}{0.33} \& \drawgridrectangle{3}{2}{green!30}{0.33} \& \drawgridrectangle{3}{3}{green!30}{0.33}
      \\[-1.5ex]
      $\{\vx_{1,d}\}$ \& $\{\vh_{1,d}\}$ \& $\{\vg_{1,d}\}$
      \\
      \drawgridrectangle{3}{3}{red!30}{0.33} \& \drawgridrectangle{3}{2}{red!30}{0.33} \& \drawgridrectangle{3}{1}{red!30}{0.33} \& \drawgridrectangle{1}{1}{red!60}{0.33}
      \\[-1.5ex]
      $\{\vx_{2,d}\}$ \& $\{\vh_{2,d}\}$ \& $\{\vg_{2,d}\}$ \& $\sum_d \vg_{2,d}$
      \\
    };

    % draw dependencies
    \pgfmathsetmacro{\K}{3}
    \pgfmathsetmacro{\L}{2}

    \foreach \l in {1,...,\L}{
      \pgfmathsetmacro{\lother}{int(\l+1)}
      \foreach \k in {1,...,\K} {
        \pgfmathsetmacro{\row}{int(2*\k-1)}
        \foreach \kother in {\k,...,\K} {
          \pgfmathsetmacro{\rowother}{int(2*\kother-1)}
          \draw[-Stealth, line width=1pt, white!50!black] (taylor-\row-\l.east) -- (taylor-\rowother-\lother.west);
        }
      }
    }
    \pgfmathsetmacro{\Lstart}{int(\L + 1)}
    \pgfmathsetmacro{\Lend}{int(\L + 2)}
    \pgfmathsetmacro{\rowfinal}{int(2*\K - 1)}
    \draw[-Stealth, line width=1pt, white!50!black] (taylor-\rowfinal-\Lstart.east) -- (taylor-\rowfinal-\Lend.west);
  \end{tikzpicture}
}

\newsavebox{\taylorCollapsed}
\savebox{\taylorCollapsed}{
  \begin{tikzpicture}
    \matrix [%
    matrix of nodes,%
    ampersand replacement=\&,% to use inside a savebox
    nodes={anchor=center, align=center},%
    column sep=4ex,%
    row sep=1ex,%
    ] (taylor)
    {
      \drawgridrectangle{1}{3}{blue!30}{0.33} \& \drawgridrectangle{1}{2}{blue!30}{0.33} \& \drawgridrectangle{1}{1}{blue!30}{0.33}
      \\[-1.5ex]
      $\vx_0$ \& $\vh_0$ \& $\vg_0$
      \\
      \drawgridrectangle{3}{3}{green!30}{0.33} \& \drawgridrectangle{3}{2}{green!30}{0.33} \& \drawgridrectangle{3}{3}{green!30}{0.33}
      \\[-1.5ex]
      $\{\vx_{1,d}\}$ \& $\{\vh_{1,d}\}$ \& $\{\vg_{1,d}\}$
      \\[2ex]
      \drawgridrectangle{1}{3}{red!60}{0.33} \& \drawgridrectangle{1}{2}{red!60}{0.33} \& \drawgridrectangle{1}{1}{red!60}{0.33}
      \\[-1.5ex]
      $\sum_d \vx_{2,d}$ \& $\sum_d \vh_{2,d}$ \& $\sum_d \vg_{2,d}$
      \\[1.1ex]
      \& \&
      \\
    };

    % draw dependencies
    \pgfmathsetmacro{\K}{3}
    \pgfmathsetmacro{\L}{2}

    \foreach \l in {1,...,\L}{
      \pgfmathsetmacro{\lother}{int(\l+1)}
      \foreach \k in {1,...,\K} {
        \pgfmathsetmacro{\row}{int(2*\k-1)}
        \foreach \kother in {\k,...,\K} {
          \pgfmathsetmacro{\rowother}{int(2*\kother-1)}
          \draw[-Stealth, line width=1pt, white!50!black] (taylor-\row-\l.east) -- (taylor-\rowother-\lother.west);
        }
      }
    }
  \end{tikzpicture}
}

\begin{figure*}[!t]
  \centering
  \resizebox{\linewidth}{!}{%
    \begin{tikzpicture}
      \node (standard) [fill=black!5!white, draw=black, rounded corners]{\usebox{\taylorStandard}};
      \node [anchor=north east, align=center, inner sep=10pt] at (standard.north east) {\textbf{Standard} \\ \textbf{Taylor mode}};
      \node (collapsed) [fill=black!5!white, draw=black, rounded corners, anchor=north west, xshift=5pt] at (standard.north east) {\usebox{\taylorCollapsed}};
      \node [anchor=south, align=center, inner sep=3pt] at (collapsed.south) {\textbf{Collapsed Taylor mode (ours)}};
    \end{tikzpicture}
  }
  \caption{\textbf{Visual comparison of standard Taylor mode and our proposed collapsed Taylor mode.}}\label{fig:comparison-standard-vs-collapsed}
\end{figure*}

\begin{figure*}
  \centering
  \newsavebox{\taylorStandardNew}
  \savebox{\taylorStandardNew}{
    \begin{tikzpicture}
      \matrix [%
      matrix of nodes,%
      ampersand replacement=\&,% to use inside a savebox
      nodes={anchor=center, align=center},%
      column sep=5ex,%
      row sep=1ex,%
      ] (taylor)
      {
        \drawgridrectangle{1}{5}{gray!25!white}{0.33}
        \& \drawgridrectangle{1}{3}{gray!25!white}{0.33}
        \& \drawgridrectangle{1}{1}{gray!25!white}{0.33}
        \\[-1.5ex]
        $\vx_0$ \& $\vh_0$ \& $\vg_0$
        \\
        \drawgridrectangle{4}{5}{gray!50!white}{0.33}
        \& \drawgridrectangle{4}{3}{gray!50!white}{0.33}
        \& \drawgridrectangle{4}{1}{gray!50!white}{0.33}
        \\[-1.5ex]
        $\{\vx_{1,d}\}$ \& $\{\vh_{1,d}\}$ \& $\{\vg_{1,d}\}$
        \\[0.5ex]
        \drawgridrectangle{1}{5}{white}{0.33}
        \& \drawgridrectangle{1}{3}{white}{0.33}
        \& \drawgridrectangle{1}{1}{white}{0.33}
        \\[0.5ex]
        \\
        \drawgridrectangle{4}{5}{gray}{0.33}
        \& \drawgridrectangle{4}{3}{gray}{0.33}
        \& \drawgridrectangle{4}{1}{gray}{0.33}
        \\[-1.5ex]
        $\{\vx_{K-1,d}\}$ \& $\{\vh_{K-1,d}\}$ \& $\{\vg_{K-1,d}\}$
        \\[0.5ex]
        \drawgridrectangle{4}{5}{red!50}{0.33}
        \&
        \drawgridrectangle{4}{3}{red!50}{0.33}
        \&
        \drawgridrectangle{4}{1}{red!50}{0.33}
        \\[-1.5ex]
        $\{\vx_{K,d}\}$ \& $\{\vh_{K,d}\}$ \& $\{\vg_{K,d}\}$
        \\[-1.5ex]
        \textcolor{purple!50!red}{$\sum_d \vx_{K,d}$}
        \& \textcolor{purple!50!red}{$\sum_d \vh_{K,d}$}
        \& \textcolor{purple!50!red}{$\sum_d \vg_{K,d}$}
        \\
      };

      \node[xshift=-1pt, yshift=-14pt] at (taylor-9-1) {\drawgridrectangle{1}{5}{purple!50!red}{0.33}};
      \node[xshift=-1pt, yshift=-14pt] at (taylor-9-2) {\drawgridrectangle{1}{3}{purple!50!red}{0.33}};
      \node[xshift=-1pt, yshift=-14pt] at (taylor-9-3) {\drawgridrectangle{1}{1}{purple!50!red}{0.33}};

      \node at (taylor-5-1) {\vdots};
      \node at (taylor-5-2) {\vdots};
      \node at (taylor-5-3) {\vdots};

      % draw dependencies
      \pgfmathsetmacro{\K}{5}
      \pgfmathsetmacro{\L}{2}

      \foreach \l in {1,...,\L}{
        \pgfmathsetmacro{\lother}{int(\l+1)}
        \foreach \k in {1,...,\K} {
          \pgfmathsetmacro{\row}{int(2*\k-1)}
          \foreach \kother in {\k,...,\K} {
            \pgfmathsetmacro{\rowother}{int(2*\kother-1)}
            \draw[-Stealth, line width=1pt, gray] (taylor-\row-\l.east) -- (taylor-\rowother-\lother.west);
          }
        }
      }

      \coordinate (arrowStart) at ($(taylor-1-1.north)+(0,3.5ex)$);
      \coordinate (arrowEnd) at ($(taylor-1-3.north east)+(0,3.5ex)$);
      \draw[-Stealth, line width=2pt, black] (arrowStart) to node [midway, fill=white, align=center] {\textbf{Taylor forward} \\ \textbf{propagation}} (arrowEnd);

      \node [left=1.5ex of taylor-1-1] (zero) {0};
      \node [left=1.5ex of taylor-3-1] {1};
      \node [left=1.5ex of taylor-5-1] {$\vdots$};
      \node [left=1.5ex of taylor-7-1] {$K-1$};
      \node [left=1.5ex of taylor-9-1] {$K$};

      \node [align=center] (coefficientLabel) at ($(zero)+(0, 5.5ex)$) {\textbf{Derivative}\\\textbf{degree}};

      \draw[rounded corners] (taylor-9-1.north west) rectangle (taylor.south east);
    \end{tikzpicture}
  }

  \begin{tikzpicture}
    \node {\usebox{\taylorStandardNew}};
  \end{tikzpicture}
\end{figure*}

%%% Local Variables:
%%% mode: LaTeX
%%% TeX-master: "../main"
%%% End:

  \end{minipage}
\end{figure*}

\begin{enumerate}[leftmargin=0.5cm]
\item \textbf{We propose optimizing standard Taylor mode by collapsing the highest Taylor coefficients} by directly propagating their sum, rather than propagating then summing (\cref{fig:visual-abstract}). This recovers the forward Laplacian and is applicable to randomized Taylor mode. Furthermore, relying on a result of Griewank et al.\, \cite{griewank_evaluating_1999}, we show how to transform general PDE operators into a form amenable to our setting.

%and, relying on [...] can be generalized to 

  %Collapsing standard Taylor mode for the Laplacian yields the forward Laplacian \cite{li2023forward}, but we show that this optimization is applicable to many other differential operators, and stochastic Taylor mode \cite{shi2024stochastic}.
  \todo{Felix: Phrase this so that it becomes clear that this is a theoretical contribution.}

\item \textbf{We show how to collapse standard Taylor mode by simple graph rewrites based on linearity.}
  This leads to a clean separation of concepts:
  Users can build their computational graph using standard Taylor mode, then use graph rewrites to collapse it.
  Due to the simple nature of our proposed rewrites, this feature could easily be absorbed into the just-in-time (JIT) compilation of ML frameworks, without introducing a new interface.

\item \textbf{We empirically demonstrate the performance improvements of collapsed Taylor mode.} We introduce a Taylor mode library for PyTorch \cite{paszke2019pytorch} that realizes the graph simplifications with \texttt{torch.fx} \cite{reed2022torch}. Furthemore, we show that collapsing Taylor mode yields the theoretically expected improvements in run time and memory consumption. Due to the simple nature of our proposed rewrites, this feature could easily be integrated into the just-in-time (JIT) compilation of ML frameworks, without introducing a new interface.

\end{enumerate}

Our work takes an important step towards the broader adoption of Taylor mode as viable alternative to backpropagation for computing PDE operators, while being as easy to use. 

%%% Local Variables:
%%% mode: LaTeX
%%% TeX-master: "../main"
%%% End:


\section{Background}\label{sec:background}
Taylor-mode AD (or, simply, Taylor mode) computes higher-order derivatives---as needed, \eg, for PDE operators---through propagation of Taylor coefficients according to the chain rule.

\begin{figure}[!t]
  \centering
  \begin{minipage}[t]{0.7\linewidth}
    \centering
   \begin{tikzpicture}
    \tikzset{box/.style={rectangle, rounded corners, draw=black, inner sep=3pt}}
    \node[align=center, box] (topleft) {Extent input to \\ smooth path
      $\vx(t)$};

    \node[align=center, right=2.5cm of topleft, box] (topright) {Path in output \\ space $f(\vx(t))$};
    \draw [-Latex] (topleft.east) to node [midway, above] {$f$} (topright.west);

    \node[align=center, below=1cm of topright, box, draw=tab-orange] (bottomright) {%Taylor polynomial of degree $K$
      % \\
      $K$-jet \; $\sum_{k=0}^K \frac{t^k}{k!} \vf_k$
      % \\
      % {\color{maincolor}$(\vf_0, \dots, \vf_K)$}
    };
    \draw [-Latex] (topright.south) to node [midway, right] {$J^K$}
    (bottomright.north);

    \node[align=center, below=1cm of topleft, box, draw=tab-orange] (bottomleft) {{%Taylor polynomial of degree $K$
        % \\
        $K$-jet \; $\sum_{k=0}^K \frac{t^k}{k!} \vx_k$
        % \\
        % {\color{maincolor} $(\vx_0, \dots, \vx_K)$}
      }};
    \draw [-Latex] (topleft.south) to node [midway, left] {$J^K$} (bottomleft.north);

    \draw [-Latex, \colorTM, align=center] (bottomleft.east) to node [midway, above, tab-orange] {\color{\colorTM}Taylor-mode\\ AD} (bottomright.west);
  \end{tikzpicture}
  \end{minipage}
  \hfill
  \begin{minipage}[b]{0.29\linewidth}
    \caption{The Taylor-mode AD: From the $K$-jet of the input to the $K$-jet of the output.}
  \label{fig:utp}
  \end{minipage}
\end{figure}

%%% Local Variables:
%%% mode: LaTeX
%%% TeX-master: "../main"
%%% End:


\paragraph{Scalar case.}
To illustrate Taylor-mode, consider the scalar function $f: \sR \to \sR$ and extend the input variable $x$ to a path $x(t)$ whose form is a univariate Taylor polynomial of degree $K$, $\smash{x(t) = \sum_{k=0}^K \frac{t^k}{k!} x_k}$ with $x_k$ the $k$-th Taylor coefficient.
If $f$ is smooth enough, we can evaluate Taylor coefficients of the transformed path $\smash{f(x(t)) = \sum_{k=0}^K \frac{t^k}{k!} f_k}$ with $\smash{f_k \coloneqq \frac{\mathrm{d}^k}{\mathrm{d}t^k} f(x(t)) |_{t=0}}$.
The chain rule provides the coefficients' propagation rules.
\Eg, for degree $K=3$ we get
\begin{align}
  \label{eq:taylor-mode-scalar}
  \begin{matrix*}[l]
    f_0 = f(x_0)\,,
    \\[0.75ex]
    f_1 = \partial f(x_0) x_1\,,
  \end{matrix*}
  \qquad
  \begin{matrix*}[l]
    f_2 = \partial^2 f(x_0)x_1^2 + \partial f(x_0) x_2\,,
    \\[0.75ex]
    f_3
    = \partial^3 f(x_0)x_1^3 + 3 \partial^2 f(x_0) x_1 x_2 + \partial f(x_0) x_3\,.
  \end{matrix*}
\end{align}
\citet{faa1857note} provided the general formula for $f_k$, and \citet{fraenkel1978formulae} extended it to the multivariate case \cite[see also][]{arbogast1800calcul,hardy2006combinatorics}.
It serves as foundation for the Taylor mode to compute higher-order derivatives, e.g.,  \citep[\S13]{griewank2008evaluating}:
setting $x_1 = 1, x_2 = x_3 = 0$ yields $f_1 = \partial f(x_0), f_2 = \partial^2 f(x_0), f_3 = \partial^3 f(x_0)$.
We call the univariate Taylor polynomial of a function $x(t)$ of degree $K$, represented by the coefficients $(x_0, \dots, x_K)$, the \emph{$K$-jet of $x$}, following the terminology of JAX's Taylor mode.

\paragraph{Notation for multivariate case.}
We consider the general case of computing higher-order derivatives, \eg PDE operators, of a vector-to-vector function $\vf: \sR^D \to \sR^C$.
This requires additional notation to generalize \cref{eq:taylor-mode-scalar}.
Given $K$ vectors $\vv_1, \dots, \vv_K \in \sR^D$, we write their tensor product as
\begin{equation*}
  \otimes_{k=1}^K \vv_k = \vv_1 \otimes \ldots \otimes \vv_K
  \in ( \sR^D )^{\otimes K}
  \quad
  \text{with entries}
  \quad
  \left[\otimes_{k=1}^K \vv_k\right]_{d_1, \dots, d_K}
  = [\vv_1]_{d_1} \cdots [\vv_K]_{d_K}
\end{equation*}
for $d_1, \dots, d_K \in \{1, \dots, D\}$, and compactly write $\vv^{\otimes K} = \otimes_{k=1}^K \vv$.
We define the inner product of two tensors $\smash{\tA, \tB \in (\sR^{D})^{\otimes K}}$ as the Euclidean inner product of their flattened versions
\begin{align}\label{eq:derivative-tensor-scalar-product}
  \textstyle % Comment out this line if we have enough space
  \left\langle
  \tA, \tB
  \right\rangle
  \coloneqq
  \sum_{d_1}
  \sum_{d_2}
  \dots
  \sum_{d_K}
  \tA_{d_1, d_2, \dots, d_K}
  \tB_{d_1, d_2, \dots, d_K} \in \sR\,.
\end{align}
We allow broadcasting in \cref{eq:derivative-tensor-scalar-product}: if one tensor has more dimensions but matching trailing dimensions, we take the inner product for each component of the leading dimensions.
This allows to express contractions with derivative tensors of vector-valued functions, \eg,
contracting the $k$-th derivative tensor $\partial^k \vf(\vx_0) \in \sR^C \otimes (\sR^D)^k$, such that $\langle \tA, \partial^k \vf(\vx_0) \rangle \in \sR^C$. 

\paragraph{Multivariate case \& composition.}
Evaluating the $K$-jet of $\vf$ at an argument $\vx_0 \in \sR^D$ starts with the extension of $\vx_0$ to a smooth path $\vx: \sR \to \sR^D$ with $\vx(0) = \vx_0$.
The $K$-jet of $\vf$ is defined as \todo{Felix@Tim: Wouldn't it be better to define the $K$-jet as mapping from $\times_{k=0}^K \sR^D \to \times_{k=0}^K \sR^C$ ?
  This would be more aligned with the propagation in \cref{eq:taylor-mode-composition}
  T: We could, this is just notation, I thought it would be better because this is the usual motivation of Taylor-mode.
  AW: I think to be precise it should be a mapping from $\times_{k=0}^K \sR^D$ to the polynomials of degree $\le K$ in the sense of Fig. 3 ;-), but that might be too complicated. For me the current statement is fine, since $(J^K \vf)$ indeed maps from $\R \to \R^C$ as stated
  T: The mapping is actually precise in the sense that you could interpret everything as the vectors of the coefficients and use this as the jet definition instead of the sum. Then, we would have indeed the mapping between $\times_{k=0}^K \sR^D$ and $\times_{k=0}^K \sR^C$. Ive seen this in other ML papers about taylor mode. But as I said, for me it doesnt matter :D}
\begin{align*}
  \textstyle % Comment out this line if we have enough space
  J^K \vf : \sR \to \sR^C\,,
  \quad (J^K \vf)(t) := \sum_{k=0}^K \frac{t^k}{k!} \vf_k
  \quad \text{with} \quad
  \vf_k := \left. \frac{\mathrm{d}^k}{\mathrm{d}t^k} \vf(\vx(t)) \right|_{t=0}
\end{align*}
and requires the $K$-jet of $\vx$, $(J^K \vx)(t) := \sum_{k=0}^K \frac{t^k}{k!} \vx_k$ (illustrated in \cref{fig:utp}). \todo{mention the jet form (x1, ... xk)}

As is common for AD, the computation of the $\vf_k$ known as Taylor mode relies on composing $\vf$ of elemental functions with known derivatives, and the chain rule.
In the simplest case, let $\vf$ be given by $\vf = \vg \circ \vh: \sR^D \to \sR^I \to \sR^C$ for two elemental functions $\vg$ and $\vh$. Given the input $K$-jet for $\vx$, the coefficients $\vh_k = \left. \frac{\mathrm{d}^k}{\mathrm{d}t^k}\vh(\vx(t)) \right|_{t=0}$ are computed using the generalized Faà di Bruno formula \eqref{eq:faa-di-bruno}:
\begin{align}
  \label{eq:faa-di-bruno}
  % \textstyle % Comment out this line if we have enough space
  \vh_k
  =
  \sum_{\sigma \in \partitioning(k)}
  \nu(\sigma)
  \left<
  \partial^{|\sigma|} \vh,
  \tensorprod{s \in \sigma} \vx_s
  \right>
  \quad
  \text{with}
  \quad
  \nu(\sigma)
  =
  \frac{k!}{
    \left(
      \prod_{s \in \sigma
      }
      n_s!
    \right)
    \left(
      \prod_{s \in \sigma}
      s!
    \right)
  }\,
\end{align}
Here, $\partitioning(k)$ is the integer partitioning of $k$ (a set of sets), $\nu$ is a multiplicity function, and $n_s$ counts occurrences of $s$ in an integer partitioning $\sigma$ (\eg $n_1(\{1,1,3\}) = 2$ and $n_3 = 1$).
The $\vh_k$'s are propagated through $\vg$ resulting in the $K$-jet for $\vf$.
In summary, this yields the propagation scheme (where $\vx_k \in \sR^D$, $\vh_k \in \sR^I$, $\vf_k \in \sR^C$)
\begin{align}\label{eq:taylor-mode-composition}
  \begin{split}
    &\begin{pmatrix*}
      \vx_0
      \\
      \vx_1
      \\
      \vx_2
      \\
      \vdots
      \\
      \vx_K
    \end{pmatrix*}
      \!\overset{\text{(\ref{eq:faa-di-bruno})}}{\to}\!
      \begin{pmatrix*}[l]
        \vh_0 \!=\!  \vh(\vx_0)
        \\
        \vh_1 \!=\!  \left<
        \partial \vh(\vx_0),
        \vx_1
        \right>
        \\
        \vh_2 \!=\! \left<
        \partial^2 \vh(\vx_0),
        \vx_1 \otimes \vx_1
        \right>
        \!+\!
        \left <
        \partial \vh(\vx_0),
        \vx_2
        \right>
        \\
        \vdots
        \\
        \vh_K \!=\!
        \displaystyle \sum_{
        \mathclap{
        \sigma \in \partitioning(K)
        }
        }
        \nu(\sigma) \left<
        \partial^{|\sigma|} \vh(\vx_0),
        \tensorprod{s \in \sigma} \vx_s
        \right>\!\!\!
      \end{pmatrix*}
    \\
    &\!\!\overset{\text{(\ref{eq:faa-di-bruno})}}{\to}\!
      \left(\!\!\!
      \begin{array}{l}
        \vg_0 \!=\!  \vg(\vh_0)
        \\
        \vg_1 \!=\! \left<
        \partial \vg(\vh_0),
        \vh_1
        \right>
        \\
        \vg_2 \!=\! \left<
        \partial^2 \vg(\vh_0),
        \vh_1 \!\otimes\! \vh_1\right>
        \!+\!
        \left< \partial \vg(\vh_0),
        \vh_2
        \right>
        \\
        \vdots
        \\
        \vg_K \!=\!
        \displaystyle\sum_{
        \mathclap{
        \sigma \in \partitioning(K)
        }
        }
        \nu(\sigma) \left<
        \partial^{|\sigma|} \vg(\vh_0),
        \tensorprod{s \in \sigma} \vh_s
        \right>
      \end{array}
      \!\!\!\!
      \right)
      \!\!=\!\!
      \left(\!\!\!
      \begin{array}{l}
        \vf_0 \!=\!  \vf(\vx_0)
        \\
        \vf_1 \!=\! \left<
        \partial \vf(\vx_0),
        \vx_1
        \right>
        \\
        \vf_2 \!=\! \left<
        \partial^2 \vf(\vx_0),
        \vx_1 \!\otimes\! \vx_1
        \right>
        \!+\!
        \left< \partial \vf(\vx_0),
        \vx_2
        \right>
        \\
        \vdots
        \\
        \vf_K \!=\!
        \displaystyle\sum_{
        \mathclap{
        \sigma \in \partitioning(K)
        }
        }
        \nu(\sigma) \left<
        \partial^{|\sigma|} \vf(\vx_0),
        \tensorprod{s \in \sigma} \vx_s
        \right>
      \end{array}
      \!\!\!\!
      \right)
  \end{split}
\end{align}
which describes the forward propagation of a single $K$-jet.
\citet[][\S13]{griewank2008evaluating} presents explicit formulas of common elemental functions that have quadratic complexity in $K$.
However, computing PDE operators requires propagating \emph{multiple} $K$-jets in parallel accumulating their results.
We propose to pull this accumulation inside Taylor mode's propagation scheme, thereby collapsing it.

%%% Local Variables:
%%% mode: LaTeX
%%% TeX-master: "../main"
%%% End:


% \section{Kronecker-Factored Approximate Curvature for
% PINNs}\label{sec:kfac_pinns}
% \input{sections/contribution.tex}

\section{Experiments}\label{sec:experiments}
\paragraph{Design decisions and limitations.} Although JAX already offers an experimental Taylor mode implementation, we re-implemented Taylor mode in PyTorch, taking heavy inspiration from the JAX implementation for the interface.
We decided so because PyTorch's \texttt{fx} library provides a user-friendly official interface to capture and transform compute graphs of functions to apply our proposed collapsing.
While we believe that such graph transformations should in principle be feasible in JAX as well (and could \eg be integrated into its \texttt{jit} compiler), it seems that this can currently only be achieved with (potentially fragile) internal APIs and requires a deep understanding of its internal tracing mechanisms.
Main advantage seems to be that we can clearly disentangle building the graph from simplifying it using linearity, whereas in JAX we could write a custom interpreter that would have to re-implement the logic of summing the $K$-th component for abitrary $K$.

\paragraph{Usage.} Our implementation takes a PyTorch function (\eg a neural net) and first captures its compute graph using \texttt{torch.fx}'s symbolic tracing mechanism, then replaces each operation with its Taylor arithmetic.
This yields the compute graph of the function's $K$-jet.
Users can use this vanilla Taylor mode to define their differential operator's computation. 
The collapsing is done by again tracing said computation with \texttt{torch.fx} and rewriting the resulting graph, propagating the summation of highest coefficient up.
The resulting graph performs the same computation, but uses our collapsed Taylor mode.
See \Cref{fig:interface-overview} for a visual walk-through of the procedure.

TODO Limitations.

\paragraph{Experimental procedure.} We compare our collapsed Taylor mode with vanilla Taylor mode and nested first-order AD on the previously discussed differential operators in PyTorch on an NVIDIA RTX 6000 with .




\begin{figure*}[!t]
  \centering
  % From https://tex.stackexchange.com/a/7318
  \newcolumntype{C}{ >{\centering\arraybackslash} m{0.12\textwidth} }
  \newcolumntype{D}{ >{\centering\arraybackslash} m{0.27\textwidth} }
  \begin{tabular}{CDDD}
    & \textbf{Laplacian $(D=50)$}
    & \textbf{Weighted Laplacian $(D=50)$}
    & \textbf{Bi-harmonic $(D=5)$}
    \\
    \textbf{Exact}
    & \includegraphics{../jet/exp/exp01_benchmark_laplacian/figures/architecture_tanh_mlp_768_768_512_512_1_device_cuda_dim_50_name_laplacian_vary_batch_size.pdf}
    &  \includegraphics{../jet/exp/exp01_benchmark_laplacian/figures/architecture_tanh_mlp_768_768_512_512_1_device_cuda_dim_50_name_weighted_laplacian_vary_batch_size.pdf}
    & \includegraphics{../jet/exp/exp01_benchmark_laplacian/figures/architecture_tanh_mlp_768_768_512_512_1_device_cuda_dim_5_name_bilaplacian_vary_batch_size.pdf}
    \\
    \textbf{Stochastic}
    & \includegraphics{../jet/exp/exp01_benchmark_laplacian/figures/architecture_tanh_mlp_768_768_512_512_1_batch_size_2048_device_cuda_dim_50_distribution_normal_name_laplacian_vary_num_samples.pdf}
    & \includegraphics{../jet/exp/exp01_benchmark_laplacian/figures/architecture_tanh_mlp_768_768_512_512_1_batch_size_2048_device_cuda_dim_50_distribution_normal_name_weighted_laplacian_vary_num_samples.pdf}
    & \includegraphics{../jet/exp/exp01_benchmark_laplacian/figures/architecture_tanh_mlp_768_768_512_512_1_batch_size_256_device_cuda_dim_5_distribution_normal_name_bilaplacian_vary_num_samples.pdf}
  \end{tabular}

  \newsavebox{\benchmarkLegend}
  \savebox{\benchmarkLegend}{
    \begin{tikzpicture}[font=\small]
      \matrix [%
      matrix of nodes,%
      ampersand replacement=\&,% to use inside a savebox
      nodes={anchor=west, align=left, inner sep=1pt},%
      column sep=1ex,%
      row sep=0ex,%
      ] (legend)
      {
        \& \draw[tab-blue] plot[mark=*] coordinates {(0,0)};
        \& Nested first-order AD\phantom{y}\quad
        \& \draw[tab-orange] plot[mark=triangle*, rotate=270] coordinates {(0,0)};
        \& Standard Taylor mode\quad
        \& \draw[tab-green] plot[mark=triangle*, rotate=90] coordinates {(0,0)};
        \& Collapsed Taylor mode (ours)
        \\
        \& \node[anchor=center]{\tikz\draw[thick] (0, 0) to ++(2.5ex, 0);};
        \& Differentiable
        \& \node[anchor=center, opacity=0.5]{\tikz\draw[thick, dashed] (0, 0) to ++(2.5ex, 0);};
        \& Non-differentiable
        \& \node[anchor=center, opacity=0.0]{\tikz\draw[thick] (0, 0) to ++(2.5ex, 0);};
        \\
      };

      \draw[gray, rounded corners] (current bounding box.north west) rectangle (current bounding box.south east);
    \end{tikzpicture}
  }
  \begin{tikzpicture}
    \node {\usebox{\benchmarkLegend}};
  \end{tikzpicture}

  \caption{\textbf{\textcolor{tab-green}{Collapsed Taylor mode} is faster than \textcolor{tab-orange}{standard Taylor mode} and \textcolor{tab-blue}{nested first-order automatic differentiaion} while using less or the same memory, \textcolor{black!50!white}{opaque} memory consumptions are for non-differentiable computations.}
    Results are on GPU and we use a $D \to 768 \to 768 \to 512 \to 512 \to 1$ MLP with tanh activations with $D=50$ for the Laplacians and $D=5$ for the Bi-harmonic operator.
    The exact computation varies the batch size, and the approximate computation fixes $N=2048$ for the Laplacians, and $N=256$ for the Bi-Laplacian, and varies the number of Monte-Carlo samples such that $S < D$ for the Laplacians, and $2 + 3S < 6D^2 - 3D + 6$ for the Bi-Laplacian (we can compute exactly otherwise).
    For each approach, we fit a line to the data and report the slope in \Cref{tab:benchmark}  to quantify the relative speedup and memory reduction.
  }
  \label{fig:benchmark}
\end{figure*}

\begin{table}[!t]
  \centering
  \caption{\textbf{Benchmark from \Cref{fig:benchmark} in numbers.}
    We fit linear functions and report their slopes, \ie how much run time and memory increase when incrementing the batch size or number of Monte-Carlo samples.
    Our collapsed Taylor mode is up to two times faster than nested first-order autodiff, while using 80\% of memory in the differentiable, and 70\% in the non-differentiable, setting.
    All numbers are shown with two significant digits and bold values are best according to parenthesized values.}
  \label{tab:benchmark}
  \vspace{1.5ex}
  % paths where the performances are stored
  \def\datapathLaplacianExact{../jet/exp/exp01_benchmark_laplacian/performance/architecture_tanh_mlp_768_768_512_512_1_device_cuda_dim_50_name_laplacian_vary_batch_size}
  \def\datapathLaplacianStochastic{../jet/exp/exp01_benchmark_laplacian/performance/architecture_tanh_mlp_768_768_512_512_1_batch_size_2048_device_cuda_dim_50_distribution_normal_name_laplacian_vary_num_samples}
  \def\datapathWeightedLaplacianExact{../jet/exp/exp01_benchmark_laplacian/performance/architecture_tanh_mlp_768_768_512_512_1_device_cuda_dim_50_name_weighted_laplacian_vary_batch_size}
  \def\datapathWeightedLaplacianStochastic{../jet/exp/exp01_benchmark_laplacian/performance/architecture_tanh_mlp_768_768_512_512_1_batch_size_2048_device_cuda_dim_50_distribution_normal_name_weighted_laplacian_vary_num_samples}
  \def\datapathBilaplacianExact{../jet/exp/exp01_benchmark_laplacian/performance/architecture_tanh_mlp_768_768_512_512_1_device_cuda_dim_5_name_bilaplacian_vary_batch_size}
  \def\datapathBilaplacianStochastic{../jet/exp/exp01_benchmark_laplacian/performance/architecture_tanh_mlp_768_768_512_512_1_batch_size_256_device_cuda_dim_5_distribution_normal_name_bilaplacian_vary_num_samples}
  \resizebox{\linewidth}{!}{%
    % configuration options for the \num command
    \sisetup{%
      % scientific-notation=true,%
      round-mode=figures,%
      round-precision=2,%
      detect-weight, % for bolding to work
      tight-spacing=true, % less space around \cdot
    }
    \begin{tabular}{ccc|cccc}
      \toprule
      \textbf{Mode}
      & \makecell{\textbf{Per-datum or } \\ \textbf{-sample cost}}
      & \textbf{Implementation}
      & \textbf{Laplacian}
      & \makecell{\textbf{Weighted} \\ \textbf{Laplacian}}
      & \textbf{Bi-harmonic}
      \\
      \midrule
      \multirow{9}{*}{\textbf{Exact}}
      & \multirow{3}{*}{Time [ms]}
      & \textcolor{tab-blue}{Nested first-order}
      & \input{\datapathLaplacianExact/hessian_trace_best.txt}
      & \input{\datapathWeightedLaplacianExact/hessian_trace_best.txt}
      & \input{\datapathBilaplacianExact/hessian_trace_best.txt}
      \\
      &
      & \textcolor{tab-orange}{Standard Taylor}
      & \input{\datapathLaplacianExact/jet_naive_best.txt}
      & \input{\datapathWeightedLaplacianExact/jet_naive_best.txt}
      & \input{\datapathBilaplacianExact/jet_naive_best.txt}
      \\
      &
      & \textcolor{tab-green}{Collapsed (ours)}
      & \textbf{\input{\datapathLaplacianExact/jet_simplified_best.txt}}
      & \textbf{\input{\datapathWeightedLaplacianExact/jet_simplified_best.txt}}
      & \textbf{\input{\datapathBilaplacianExact/jet_simplified_best.txt}}
      \\ \cmidrule{2-6}
      & \multirow{3}{*}{\makecell{Mem.\,[MiB] \\ (differentiable)}}
      & \textcolor{tab-blue}{Nested first-order}
      & \input{\datapathLaplacianExact/hessian_trace_peakmem.txt}
      & \input{\datapathWeightedLaplacianExact/hessian_trace_peakmem.txt}
      & \textbf{\input{\datapathBilaplacianExact/hessian_trace_peakmem.txt}}
      \\
      &
      & \textcolor{tab-orange}{Standard Taylor}
      & \input{\datapathLaplacianExact/jet_naive_peakmem.txt}
      & \input{\datapathWeightedLaplacianExact/jet_naive_peakmem.txt}
      & \input{\datapathBilaplacianExact/jet_naive_peakmem.txt}
      \\
      &
      & \textcolor{tab-green}{Collapsed (ours)}
      & \textbf{\input{\datapathLaplacianExact/jet_simplified_peakmem.txt}}
      & \textbf{\input{\datapathWeightedLaplacianExact/jet_simplified_peakmem.txt}}
      & \input{\datapathBilaplacianExact/jet_simplified_peakmem.txt}
      \\ \cmidrule{2-6}
      & \multirow{3}{*}{\makecell{Mem.\,[MiB] \\ (non-diff.)}}
      & \textcolor{tab-blue}{Nested first-order}
      & \input{\datapathLaplacianExact/hessian_trace_peakmem_nondifferentiable.txt}
      & \input{\datapathWeightedLaplacianExact/hessian_trace_peakmem_nondifferentiable.txt}
      & \input{\datapathBilaplacianExact/hessian_trace_peakmem_nondifferentiable.txt}
      \\
      &
      & \textcolor{tab-orange}{Standard Taylor}
      & \textbf{\input{\datapathLaplacianExact/jet_naive_peakmem_nondifferentiable.txt}}
      & \textbf{\input{\datapathWeightedLaplacianExact/jet_naive_peakmem_nondifferentiable.txt}}
      & \input{\datapathBilaplacianExact/jet_naive_peakmem_nondifferentiable.txt}
      \\
      &
      & \textcolor{tab-green}{Collapsed (ours)}
      & \textbf{\input{\datapathLaplacianExact/jet_simplified_peakmem_nondifferentiable.txt}}
      & \textbf{\input{\datapathWeightedLaplacianExact/jet_simplified_peakmem_nondifferentiable.txt}}
      & \textbf{\input{\datapathBilaplacianExact/jet_simplified_peakmem_nondifferentiable.txt}}
      \\
      \midrule
      \multirow{9}{*}{\textbf{Stochastic}}
      & \multirow{3}{*}{Time [ms]}
      & \textcolor{tab-blue}{Nested first-order}
      & \input{\datapathLaplacianStochastic/hessian_trace_best.txt}
      & \input{\datapathWeightedLaplacianStochastic/hessian_trace_best.txt}
      & \input{\datapathBilaplacianStochastic/hessian_trace_best.txt}
      \\
      &
      & \textcolor{tab-orange}{Standard Taylor}
      & \input{\datapathLaplacianStochastic/jet_naive_best.txt}
      & \input{\datapathWeightedLaplacianStochastic/jet_naive_best.txt}
      & \input{\datapathBilaplacianStochastic/jet_naive_best.txt}
      \\
      &
      & \textcolor{tab-green}{Collapsed (ours)}
      & \textbf{\input{\datapathLaplacianStochastic/jet_simplified_best.txt}}
      & \textbf{\input{\datapathWeightedLaplacianStochastic/jet_simplified_best.txt}}
      & \textbf{\input{\datapathBilaplacianStochastic/jet_simplified_best.txt}}
      \\ \cmidrule{2-6}
      & \multirow{3}{*}{\makecell{Mem.\,[MiB]\\(differentiable)}}
      & \textcolor{tab-blue}{Nested first-order}
      & \input{\datapathLaplacianStochastic/hessian_trace_peakmem.txt}
      & \input{\datapathWeightedLaplacianStochastic/hessian_trace_peakmem.txt}
      & \input{\datapathBilaplacianStochastic/hessian_trace_peakmem.txt}
      \\
      &
      & \textcolor{tab-orange}{Standard Taylor}
      & \input{\datapathLaplacianStochastic/jet_naive_peakmem.txt}
      & \input{\datapathWeightedLaplacianStochastic/jet_naive_peakmem.txt}
      & \input{\datapathBilaplacianStochastic/jet_naive_peakmem.txt}
      \\
      &
      & \textcolor{tab-green}{Collapsed (ours)}
      & \textbf{\input{\datapathLaplacianStochastic/jet_simplified_peakmem.txt}}
      & \textbf{\input{\datapathWeightedLaplacianStochastic/jet_simplified_peakmem.txt}}
      & \textbf{\input{\datapathBilaplacianStochastic/jet_simplified_peakmem.txt}}
      \\ \cmidrule{2-6}
      & \multirow{3}{*}{\makecell{Mem.\,[MiB]\\(non-diff.)}}
      & \textcolor{tab-blue}{Nested first-order}
      & \input{\datapathLaplacianStochastic/hessian_trace_peakmem_nondifferentiable.txt}
      & \input{\datapathWeightedLaplacianStochastic/hessian_trace_peakmem_nondifferentiable.txt}
      & \input{\datapathBilaplacianStochastic/hessian_trace_peakmem_nondifferentiable.txt}
      \\
      &
      & \textcolor{tab-orange}{Standard Taylor}
      & \textbf{\input{\datapathLaplacianStochastic/jet_naive_peakmem_nondifferentiable.txt}}
      & \textbf{\input{\datapathWeightedLaplacianStochastic/jet_naive_peakmem_nondifferentiable.txt}}
      & \input{\datapathBilaplacianStochastic/jet_naive_peakmem_nondifferentiable.txt}
      \\
      &
      & \textcolor{tab-green}{Collapsed (ours)}
      & \textbf{\input{\datapathLaplacianStochastic/jet_simplified_peakmem_nondifferentiable.txt}}
      & \textbf{\input{\datapathWeightedLaplacianStochastic/jet_simplified_peakmem_nondifferentiable.txt}}
      & \textbf{\input{\datapathBilaplacianStochastic/jet_simplified_peakmem_nondifferentiable.txt}}
      \\
      \bottomrule
    \end{tabular}
  }
\end{table}

%%% Local Variables:
%%% mode: LaTeX
%%% TeX-master: "../main"
%%% End:


% \section{Discussion and Conclusion}\label{sec:conclusion}
% Computing differential operators of high-dimensional functions is a critical component in scientific machine learning, particularly for physics-informed neural networks and variational Monte Carlo. 
While Taylor-mode automatic differentiation promises efficient computation of higher-order derivatives, we found that vanilla implementations often underperform compared to nested backpropagation. 
Our work introduces collapsed Taylor-mode AD, a simple yet effective optimization that propagates the sum of highest-order coefficients directly through the computational graph. 
This approach (1) unifies and generalizes recent advances in forward-mode schemes, showing that the forward Laplacian emerges naturally from collapsing standard Taylor mode, while extending to other differential operators and stochastic variants, 
(2) demonstrates that such optimizations can be achieved through simple graph rewrites based on linearity, making it amenable to integration into existing just-in-time compilers without requiring specialized interfaces, and (3)
provides substantial empirical improvements over both vanilla Taylor mode and nested backpropagation, with up to 2x speedup for Laplacian operators and 9x for randomized Bi-harmonic operators, while often using less memory. 
Our PyTorch implementation and experiments confirm that these theoretical benefits translate into practical performance gains. 
The success of collapsed Taylor mode suggests that forward-mode AD schemes, when properly optimized, can outperform the traditional backpropagation approach for computing PDE operators. 
We believe this work takes an important step toward making Taylor mode a practical alternative in scientific machine learning, while maintaining ease of use through potential compiler integration. 
Future work could focus on integrating these optimizations directly into ML framework compilers, extending support to more primitive operations, and exploring additional graph-based optimizations for automatic differentiation. 
%%% Local Variables:
%%% mode: LaTeX
%%% TeX-master: "../main"
%%% End:

% \begin{ack}
  Funding statements and acknowledgements go here.
\end{ack}
%%% Local Variables:
%%% mode: LaTeX
%%% TeX-master: "../main"
%%% End:


\bibliography{references}
\bibliographystyle{icml2024.bst}

\clearpage
\appendix

\section{Visual Explanation of Graph Simplifications}\label{sec:appendix-simplifications}
\input{sections/appendix_simplifications}

\section{Fa\`a Di Bruno Formula Cheat Sheet}\label{sec:faa-di-bruno-cheatsheet}
\begin{tiny}
  \begin{align*}
    \vx_{0}
    &\to&
          \vh_{0}
          =
          h(\vx_{0})
    &\to&
          \vg_{0}
          =
          g(\vh_{0})
          =
    &\vf_{0}
      =
      f(\vx_{0})
    \\
    \vx_{1}
    &\to&
          \vh_{1}
          =
          \partial h [\vx_{1}]
    &\to&
          \vg_{1}
          =
          \partial g [\vh_{1}]
          =
    &\vf_{1}
      =
      \partial f [\vx_{1}]
    \\
    \vx_{2}
    &\to&
          \vh_{2}
          =
          \begin{matrix}
            \partial^{2} h [\vx_{1}, \vx_{1}]
            \\
            +
            \partial h [\vx_{2}]
          \end{matrix}
    &\to&
          \vg_{2}
          =
          \begin{matrix}
            \partial^{2} g [\vh_{1}, \vh_{1}]
            \\
            +
            \partial g [\vh_{2}]
          \end{matrix}
          =
    &\vf_{2}
      =
      \begin{matrix}
        \partial^{2} f [\vx_{1}, \vx_{1}]
        \\
        +
        \partial f [\vx_{2}]
      \end{matrix}
    \\
    \vx_{3}
    &\to&
          \vh_{3}
          =
          \begin{matrix}
            \partial^{3} h [\vx_{1}, \vx_{1}, \vx_{1}]
            \\
            +
            3 \partial^{2} h [\vx_{1}, \vx_{2}]
            \\
            +
            \partial h [\vx_{3}]
          \end{matrix}
    &\to&
          \vg_{3}
          =
          \begin{matrix}
            \partial^{3} g [\vh_{1}, \vh_{1}, \vh_{1}]
            \\
            +
            3 \partial^{2} g [\vh_{1}, \vh_{2}]
            \\
            +
            \partial g [\vh_{3}]
          \end{matrix}
          =
    &\vf_{3}
      =
      \begin{matrix}
        \partial^{3} f [\vx_{1}, \vx_{1}, \vx_{1}]
        \\
        +
        3 \partial^{2} f [\vx_{1}, \vx_{2}]
        \\
        +
        \partial f [\vx_{3}]
      \end{matrix}
    \\
    \vx_{4}
    &\to&
          \vh_{4}
          =
          \begin{matrix}
            \partial^{4} h [\vx_{1}, \vx_{1}, \vx_{1}, \vx_{1}]
            \\
            +
            6 \partial^{3} h [\vx_{1}, \vx_{1}, \vx_{2}]
            \\
            +
            4 \partial^{2} h [\vx_{1}, \vx_{3}]
            \\
            +
            3 \partial^{2} h [\vx_{2}, \vx_{2}]
            \\
            +
            \partial h [\vx_{4}]
          \end{matrix}
    &\to&
          \vg_{4}
          =
          \begin{matrix}
            \partial^{4} g [\vh_{1}, \vh_{1}, \vh_{1}, \vh_{1}]
            \\
            +
            6 \partial^{3} g [\vh_{1}, \vh_{1}, \vh_{2}]
            \\
            +
            4 \partial^{2} g [\vh_{1}, \vh_{3}]
            \\
            +
            3 \partial^{2} g [\vh_{2}, \vh_{2}]
            \\
            +
            \partial g [\vh_{4}]
          \end{matrix}
          =
    &\vf_{4}
      =
      \begin{matrix}
        \partial^{4} f [\vx_{1}, \vx_{1}, \vx_{1}, \vx_{1}]
        \\
        +
        6 \partial^{3} f [\vx_{1}, \vx_{1}, \vx_{2}]
        \\
        +
        4 \partial^{2} f [\vx_{1}, \vx_{3}]
        \\
        +
        3 \partial^{2} f [\vx_{2}, \vx_{2}]
        \\
        +
        \partial f [\vx_{4}]
      \end{matrix}
    \\
    \vx_{5}
    &\to&
          \vh_{5}
          =
          \begin{matrix}
            \partial^{5} h [\vx_{1}, \vx_{1}, \vx_{1}, \vx_{1}, \vx_{1}]
            \\
            +
            10 \partial^{4} h [\vx_{1}, \vx_{1}, \vx_{1}, \vx_{2}]
            \\
            +
            10 \partial^{3} h [\vx_{1}, \vx_{1}, \vx_{3}]
            \\
            +
            15 \partial^{3} h [\vx_{1}, \vx_{2}, \vx_{2}]
            \\
            +
            5 \partial^{2} h [\vx_{1}, \vx_{4}]
            \\
            +
            10 \partial^{2} h [\vx_{2}, \vx_{3}]
            \\
            +
            \partial h [\vx_{5}]
          \end{matrix}
    &\to&
          \vg_{5}
          =
          \begin{matrix}
            \partial^{5} g [\vh_{1}, \vh_{1}, \vh_{1}, \vh_{1}, \vh_{1}]
            \\
            +
            10 \partial^{4} g [\vh_{1}, \vh_{1}, \vh_{1}, \vh_{2}]
            \\
            +
            10 \partial^{3} g [\vh_{1}, \vh_{1}, \vh_{3}]
            \\
            +
            15 \partial^{3} g [\vh_{1}, \vh_{2}, \vh_{2}]
            \\
            +
            5 \partial^{2} g [\vh_{1}, \vh_{4}]
            \\
            +
            10 \partial^{2} g [\vh_{2}, \vh_{3}]
            \\
            +
            \partial g [\vh_{5}]
          \end{matrix}
          =
    &\vf_{5}
      =
      \begin{matrix}
        \partial^{5} f [\vx_{1}, \vx_{1}, \vx_{1}, \vx_{1}, \vx_{1}]
        \\
        +
        10 \partial^{4} f [\vx_{1}, \vx_{1}, \vx_{1}, \vx_{2}]
        \\
        +
        10 \partial^{3} f [\vx_{1}, \vx_{1}, \vx_{3}]
        \\
        +
        15 \partial^{3} f [\vx_{1}, \vx_{2}, \vx_{2}]
        \\
        +
        5 \partial^{2} f [\vx_{1}, \vx_{4}]
        \\
        +
        10 \partial^{2} f [\vx_{2}, \vx_{3}]
        \\
        +
        \partial f [\vx_{5}]
      \end{matrix}
    \\
    \vx_{6}
    &\to&
          \vh_{6}
          =
          \begin{matrix}
            \partial^{6} h [\vx_{1}, \vx_{1}, \vx_{1}, \vx_{1}, \vx_{1}, \vx_{1}]
            \\
            +
            15 \partial^{5} h [\vx_{1}, \vx_{1}, \vx_{1}, \vx_{1}, \vx_{2}]
            \\
            +
            20 \partial^{4} h [\vx_{1}, \vx_{1}, \vx_{1}, \vx_{3}]
            \\
            +
            45 \partial^{4} h [\vx_{1}, \vx_{1}, \vx_{2}, \vx_{2}]
            \\
            +
            15 \partial^{3} h [\vx_{1}, \vx_{1}, \vx_{4}]
            \\
            +
            60 \partial^{3} h [\vx_{1}, \vx_{2}, \vx_{3}]
            \\
            +
            15 \partial^{3} h [\vx_{2}, \vx_{2}, \vx_{2}]
            \\
            +
            6 \partial^{2} h [\vx_{1}, \vx_{5}]
            \\
            +
            15 \partial^{2} h [\vx_{2}, \vx_{4}]
            \\
            +
            10 \partial^{2} h [\vx_{3}, \vx_{3}]
            \\
            +
            \partial h [\vx_{6}]
          \end{matrix}
    &\to&
          \vg_{6}
          =
          \begin{matrix}
            \partial^{6} g [\vh_{1}, \vh_{1}, \vh_{1}, \vh_{1}, \vh_{1}, \vh_{1}]
            \\
            +
            15 \partial^{5} g [\vh_{1}, \vh_{1}, \vh_{1}, \vh_{1}, \vh_{2}]
            \\
            +
            20 \partial^{4} g [\vh_{1}, \vh_{1}, \vh_{1}, \vh_{3}]
            \\
            +
            45 \partial^{4} g [\vh_{1}, \vh_{1}, \vh_{2}, \vh_{2}]
            \\
            +
            15 \partial^{3} g [\vh_{1}, \vh_{1}, \vh_{4}]
            \\
            +
            60 \partial^{3} g [\vh_{1}, \vh_{2}, \vh_{3}]
            \\
            +
            15 \partial^{3} g [\vh_{2}, \vh_{2}, \vh_{2}]
            \\
            +
            6 \partial^{2} g [\vh_{1}, \vh_{5}]
            \\
            +
            15 \partial^{2} g [\vh_{2}, \vh_{4}]
            \\
            +
            10 \partial^{2} g [\vh_{3}, \vh_{3}]
            \\
            +
            \partial g [\vh_{6}]
          \end{matrix}
          =
    &\vf_{6}
      =
      \begin{matrix}
        \partial^{6} f [\vx_{1}, \vx_{1}, \vx_{1}, \vx_{1}, \vx_{1}, \vx_{1}]
        \\
        +
        15 \partial^{5} f [\vx_{1}, \vx_{1}, \vx_{1}, \vx_{1}, \vx_{2}]
        \\
        +
        20 \partial^{4} f [\vx_{1}, \vx_{1}, \vx_{1}, \vx_{3}]
        \\
        +
        45 \partial^{4} f [\vx_{1}, \vx_{1}, \vx_{2}, \vx_{2}]
        \\
        +
        15 \partial^{3} f [\vx_{1}, \vx_{1}, \vx_{4}]
        \\
        +
        60 \partial^{3} f [\vx_{1}, \vx_{2}, \vx_{3}]
        \\
        +
        15 \partial^{3} f [\vx_{2}, \vx_{2}, \vx_{2}]
        \\
        +
        6 \partial^{2} f [\vx_{1}, \vx_{5}]
        \\
        +
        15 \partial^{2} f [\vx_{2}, \vx_{4}]
        \\
        +
        10 \partial^{2} f [\vx_{3}, \vx_{3}]
        \\
        +
        \partial f [\vx_{6}]
      \end{matrix}
    \\
    \vx_{7}
    &\to&
          \vh_{7}
          =
          \begin{matrix}
            \partial^{7} h [\vx_{1}, \vx_{1}, \vx_{1}, \vx_{1}, \vx_{1}, \vx_{1}, \vx_{1}]
            \\
            +
            21 \partial^{6} h [\vx_{1}, \vx_{1}, \vx_{1}, \vx_{1}, \vx_{1}, \vx_{2}]
            \\
            +
            35 \partial^{5} h [\vx_{1}, \vx_{1}, \vx_{1}, \vx_{1}, \vx_{3}]
            \\
            +
            105 \partial^{5} h [\vx_{1}, \vx_{1}, \vx_{1}, \vx_{2}, \vx_{2}]
            \\
            +
            35 \partial^{4} h [\vx_{1}, \vx_{1}, \vx_{1}, \vx_{4}]
            \\
            +
            210 \partial^{4} h [\vx_{1}, \vx_{1}, \vx_{2}, \vx_{3}]
            \\
            +
            105 \partial^{4} h [\vx_{1}, \vx_{2}, \vx_{2}, \vx_{2}]
            \\
            +
            21 \partial^{3} h [\vx_{1}, \vx_{1}, \vx_{5}]
            \\
            +
            105 \partial^{3} h [\vx_{1}, \vx_{2}, \vx_{4}]
            \\
            +
            70 \partial^{3} h [\vx_{1}, \vx_{3}, \vx_{3}]
            \\
            +
            105 \partial^{3} h [\vx_{2}, \vx_{2}, \vx_{3}]
            \\
            +
            7 \partial^{2} h [\vx_{1}, \vx_{6}]
            \\
            +
            21 \partial^{2} h [\vx_{2}, \vx_{5}]
            \\
            +
            35 \partial^{2} h [\vx_{3}, \vx_{4}]
            \\
            +
            \partial h [\vx_{7}]
          \end{matrix}
    &\to&
          \vg_{7}
          =
          \begin{matrix}
            \partial^{7} g [\vh_{1}, \vh_{1}, \vh_{1}, \vh_{1}, \vh_{1}, \vh_{1}, \vh_{1}]
            \\
            +
            21 \partial^{6} g [\vh_{1}, \vh_{1}, \vh_{1}, \vh_{1}, \vh_{1}, \vh_{2}]
            \\
            +
            35 \partial^{5} g [\vh_{1}, \vh_{1}, \vh_{1}, \vh_{1}, \vh_{3}]
            \\
            +
            105 \partial^{5} g [\vh_{1}, \vh_{1}, \vh_{1}, \vh_{2}, \vh_{2}]
            \\
            +
            35 \partial^{4} g [\vh_{1}, \vh_{1}, \vh_{1}, \vh_{4}]
            \\
            +
            210 \partial^{4} g [\vh_{1}, \vh_{1}, \vh_{2}, \vh_{3}]
            \\
            +
            105 \partial^{4} g [\vh_{1}, \vh_{2}, \vh_{2}, \vh_{2}]
            \\
            +
            21 \partial^{3} g [\vh_{1}, \vh_{1}, \vh_{5}]
            \\
            +
            105 \partial^{3} g [\vh_{1}, \vh_{2}, \vh_{4}]
            \\
            +
            70 \partial^{3} g [\vh_{1}, \vh_{3}, \vh_{3}]
            \\
            +
            105 \partial^{3} g [\vh_{2}, \vh_{2}, \vh_{3}]
            \\
            +
            7 \partial^{2} g [\vh_{1}, \vh_{6}]
            \\
            +
            21 \partial^{2} g [\vh_{2}, \vh_{5}]
            \\
            +
            35 \partial^{2} g [\vh_{3}, \vh_{4}]
            \\
            +
            \partial g [\vh_{7}]
          \end{matrix}
          =
    &\vf_{7}
      =
      \begin{matrix}
        \partial^{7} f [\vx_{1}, \vx_{1}, \vx_{1}, \vx_{1}, \vx_{1}, \vx_{1}, \vx_{1}]
        \\
        +
        21 \partial^{6} f [\vx_{1}, \vx_{1}, \vx_{1}, \vx_{1}, \vx_{1}, \vx_{2}]
        \\
        +
        35 \partial^{5} f [\vx_{1}, \vx_{1}, \vx_{1}, \vx_{1}, \vx_{3}]
        \\
        +
        105 \partial^{5} f [\vx_{1}, \vx_{1}, \vx_{1}, \vx_{2}, \vx_{2}]
        \\
        +
        35 \partial^{4} f [\vx_{1}, \vx_{1}, \vx_{1}, \vx_{4}]
        \\
        +
        210 \partial^{4} f [\vx_{1}, \vx_{1}, \vx_{2}, \vx_{3}]
        \\
        +
        105 \partial^{4} f [\vx_{1}, \vx_{2}, \vx_{2}, \vx_{2}]
        \\
        +
        21 \partial^{3} f [\vx_{1}, \vx_{1}, \vx_{5}]
        \\
        +
        105 \partial^{3} f [\vx_{1}, \vx_{2}, \vx_{4}]
        \\
        +
        70 \partial^{3} f [\vx_{1}, \vx_{3}, \vx_{3}]
        \\
        +
        105 \partial^{3} f [\vx_{2}, \vx_{2}, \vx_{3}]
        \\
        +
        7 \partial^{2} f [\vx_{1}, \vx_{6}]
        \\
        +
        21 \partial^{2} f [\vx_{2}, \vx_{5}]
        \\
        +
        35 \partial^{2} f [\vx_{3}, \vx_{4}]
        \\
        +
        \partial f [\vx_{7}]
      \end{matrix}
    \\
    \vx_{8}
    &\to&
          \vh_{8}
          =
          \begin{matrix}
            \partial^{8} h [\vx_{1}, \vx_{1}, \vx_{1}, \vx_{1}, \vx_{1}, \vx_{1}, \vx_{1}, \vx_{1}]
            \\
            +
            28 \partial^{7} h [\vx_{1}, \vx_{1}, \vx_{1}, \vx_{1}, \vx_{1}, \vx_{1}, \vx_{2}]
            \\
            +
            56 \partial^{6} h [\vx_{1}, \vx_{1}, \vx_{1}, \vx_{1}, \vx_{1}, \vx_{3}]
            \\
            +
            210 \partial^{6} h [\vx_{1}, \vx_{1}, \vx_{1}, \vx_{1}, \vx_{2}, \vx_{2}]
            \\
            +
            70 \partial^{5} h [\vx_{1}, \vx_{1}, \vx_{1}, \vx_{1}, \vx_{4}]
            \\
            +
            560 \partial^{5} h [\vx_{1}, \vx_{1}, \vx_{1}, \vx_{2}, \vx_{3}]
            \\
            +
            420 \partial^{5} h [\vx_{1}, \vx_{1}, \vx_{2}, \vx_{2}, \vx_{2}]
            \\
            +
            56 \partial^{4} h [\vx_{1}, \vx_{1}, \vx_{1}, \vx_{5}]
            \\
            +
            420 \partial^{4} h [\vx_{1}, \vx_{1}, \vx_{2}, \vx_{4}]
            \\
            +
            280 \partial^{4} h [\vx_{1}, \vx_{1}, \vx_{3}, \vx_{3}]
            \\
            +
            840 \partial^{4} h [\vx_{1}, \vx_{2}, \vx_{2}, \vx_{3}]
            \\
            +
            105 \partial^{4} h [\vx_{2}, \vx_{2}, \vx_{2}, \vx_{2}]
            \\
            +
            28 \partial^{3} h [\vx_{1}, \vx_{1}, \vx_{6}]
            \\
            +
            168 \partial^{3} h [\vx_{1}, \vx_{2}, \vx_{5}]
            \\
            +
            280 \partial^{3} h [\vx_{1}, \vx_{3}, \vx_{4}]
            \\
            +
            210 \partial^{3} h [\vx_{2}, \vx_{2}, \vx_{4}]
            \\
            +
            280 \partial^{3} h [\vx_{2}, \vx_{3}, \vx_{3}]
            \\
            +
            8 \partial^{2} h [\vx_{1}, \vx_{7}]
            \\
            +
            28 \partial^{2} h [\vx_{2}, \vx_{6}]
            \\
            +
            56 \partial^{2} h [\vx_{3}, \vx_{5}]
            \\
            +
            35 \partial^{2} h [\vx_{4}, \vx_{4}]
            \\
            +
            \partial h [\vx_{8}]
          \end{matrix}
    &\to&
          \vg_{8}
          =
          \begin{matrix}
            \partial^{8} g [\vh_{1}, \vh_{1}, \vh_{1}, \vh_{1}, \vh_{1}, \vh_{1}, \vh_{1}, \vh_{1}]
            \\
            +
            28 \partial^{7} g [\vh_{1}, \vh_{1}, \vh_{1}, \vh_{1}, \vh_{1}, \vh_{1}, \vh_{2}]
            \\
            +
            56 \partial^{6} g [\vh_{1}, \vh_{1}, \vh_{1}, \vh_{1}, \vh_{1}, \vh_{3}]
            \\
            +
            210 \partial^{6} g [\vh_{1}, \vh_{1}, \vh_{1}, \vh_{1}, \vh_{2}, \vh_{2}]
            \\
            +
            70 \partial^{5} g [\vh_{1}, \vh_{1}, \vh_{1}, \vh_{1}, \vh_{4}]
            \\
            +
            560 \partial^{5} g [\vh_{1}, \vh_{1}, \vh_{1}, \vh_{2}, \vh_{3}]
            \\
            +
            420 \partial^{5} g [\vh_{1}, \vh_{1}, \vh_{2}, \vh_{2}, \vh_{2}]
            \\
            +
            56 \partial^{4} g [\vh_{1}, \vh_{1}, \vh_{1}, \vh_{5}]
            \\
            +
            420 \partial^{4} g [\vh_{1}, \vh_{1}, \vh_{2}, \vh_{4}]
            \\
            +
            280 \partial^{4} g [\vh_{1}, \vh_{1}, \vh_{3}, \vh_{3}]
            \\
            +
            840 \partial^{4} g [\vh_{1}, \vh_{2}, \vh_{2}, \vh_{3}]
            \\
            +
            105 \partial^{4} g [\vh_{2}, \vh_{2}, \vh_{2}, \vh_{2}]
            \\
            +
            28 \partial^{3} g [\vh_{1}, \vh_{1}, \vh_{6}]
            \\
            +
            168 \partial^{3} g [\vh_{1}, \vh_{2}, \vh_{5}]
            \\
            +
            280 \partial^{3} g [\vh_{1}, \vh_{3}, \vh_{4}]
            \\
            +
            210 \partial^{3} g [\vh_{2}, \vh_{2}, \vh_{4}]
            \\
            +
            280 \partial^{3} g [\vh_{2}, \vh_{3}, \vh_{3}]
            \\
            +
            8 \partial^{2} g [\vh_{1}, \vh_{7}]
            \\
            +
            28 \partial^{2} g [\vh_{2}, \vh_{6}]
            \\
            +
            56 \partial^{2} g [\vh_{3}, \vh_{5}]
            \\
            +
            35 \partial^{2} g [\vh_{4}, \vh_{4}]
            \\
            +
            \partial g [\vh_{8}]
          \end{matrix}
          =
    &\vf_{8}
      =
      \begin{matrix}
        \partial^{8} f [\vx_{1}, \vx_{1}, \vx_{1}, \vx_{1}, \vx_{1}, \vx_{1}, \vx_{1}, \vx_{1}]
        \\
        +
        28 \partial^{7} f [\vx_{1}, \vx_{1}, \vx_{1}, \vx_{1}, \vx_{1}, \vx_{1}, \vx_{2}]
        \\
        +
        56 \partial^{6} f [\vx_{1}, \vx_{1}, \vx_{1}, \vx_{1}, \vx_{1}, \vx_{3}]
        \\
        +
        210 \partial^{6} f [\vx_{1}, \vx_{1}, \vx_{1}, \vx_{1}, \vx_{2}, \vx_{2}]
        \\
        +
        70 \partial^{5} f [\vx_{1}, \vx_{1}, \vx_{1}, \vx_{1}, \vx_{4}]
        \\
        +
        560 \partial^{5} f [\vx_{1}, \vx_{1}, \vx_{1}, \vx_{2}, \vx_{3}]
        \\
        +
        420 \partial^{5} f [\vx_{1}, \vx_{1}, \vx_{2}, \vx_{2}, \vx_{2}]
        \\
        +
        56 \partial^{4} f [\vx_{1}, \vx_{1}, \vx_{1}, \vx_{5}]
        \\
        +
        420 \partial^{4} f [\vx_{1}, \vx_{1}, \vx_{2}, \vx_{4}]
        \\
        +
        280 \partial^{4} f [\vx_{1}, \vx_{1}, \vx_{3}, \vx_{3}]
        \\
        +
        840 \partial^{4} f [\vx_{1}, \vx_{2}, \vx_{2}, \vx_{3}]
        \\
        +
        105 \partial^{4} f [\vx_{2}, \vx_{2}, \vx_{2}, \vx_{2}]
        \\
        +
        28 \partial^{3} f [\vx_{1}, \vx_{1}, \vx_{6}]
        \\
        +
        168 \partial^{3} f [\vx_{1}, \vx_{2}, \vx_{5}]
        \\
        +
        280 \partial^{3} f [\vx_{1}, \vx_{3}, \vx_{4}]
        \\
        +
        210 \partial^{3} f [\vx_{2}, \vx_{2}, \vx_{4}]
        \\
        +
        280 \partial^{3} f [\vx_{2}, \vx_{3}, \vx_{3}]
        \\
        +
        8 \partial^{2} f [\vx_{1}, \vx_{7}]
        \\
        +
        28 \partial^{2} f [\vx_{2}, \vx_{6}]
        \\
        +
        56 \partial^{2} f [\vx_{3}, \vx_{5}]
        \\
        +
        35 \partial^{2} f [\vx_{4}, \vx_{4}]
        \\
        +
        \partial f [\vx_{8}]
      \end{matrix}
  \end{align*}
\end{tiny}
%%% Local Variables:
%%% mode: LaTeX
%%% TeX-master: "../main"
%%% End:


\section{Biharmonic Operator --- Multivariate Taylor Polynomial Approach}
\input{sections/appendix_biharmonic}


\section{TTC}
\label{sec:appendix_ttc}
\input{sections/appendix_biharmonic}

% \input{sections/appendix.tex}
% \input{sections/checklist.tex}

\end{document}
%%% Local Variables:
%%% mode: latex
%%% TeX-master: t
%%% End:

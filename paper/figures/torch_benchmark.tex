\begin{figure*}[!t]
  \centering
  \newsavebox{\benchmarkLegend}
  \savebox{\benchmarkLegend}{
    \begin{tikzpicture}[font=\small]
      \matrix [%
      matrix of nodes,%
      ampersand replacement=\&,% to use inside a savebox
      nodes={anchor=west, align=left, inner sep=1pt},%
      column sep=1ex,%
      row sep=0ex,%
      ] (legend)
      {
        \draw[tab-blue] plot[mark=*] coordinates {(0,0)};
        \& Nested 1\textsuperscript{st}-order\phantom{y}\!\!\!\!
        \\
        \draw[tab-orange] plot[mark=triangle*, rotate=270] coordinates {(0,0)};
        \& Standard Taylor
        \\
        \draw[tab-green] plot[mark=triangle*, rotate=90] coordinates {(0,0)};
        \& Collapsed (ours)\phantom{y}\!\!\!\!
        \\[2ex]
        \node[anchor=center]{\tikz\draw[thick] (0, 0) to ++(2.5ex, 0);};
        \& Differentiable\phantom{y}
        \\
        \node[anchor=center, opacity=0.5]{\tikz\draw[thick, dashed] (0, 0) to ++(2.5ex, 0);};
        \& Non-diff.\phantom{y}
        \\
      };

      \draw[gray, rounded corners] (current bounding box.north west) rectangle (current bounding box.south east);
    \end{tikzpicture}
  }
  % From https://tex.stackexchange.com/a/7318
  \newcolumntype{C}{ >{\centering\arraybackslash} m{0.18\textwidth} }
  \newcolumntype{D}{ >{\centering\arraybackslash} m{0.27\textwidth} }
  \newcolumntype{E}{ >{\centering\arraybackslash} m{0.22\textwidth} }
  \begin{tabular}{CDEE}
    & \makecell{\textbf{Laplacian} \\ $(D=50)$}
    & \makecell{\textbf{Weighted Laplacian} \\ $(D=50)$}
    & \makecell{\textbf{Bi-harmonic} \\ $(D=5)$}
    \\[1ex]
    \makecell{\textbf{Exact} \\ \\ \\ \\ \\ \\ }
    & \includegraphics{../jet/exp/exp01_benchmark_laplacian/figures/architecture_tanh_mlp_768_768_512_512_1_device_cuda_dim_50_name_laplacian_vary_batch_size.pdf}
    % [trim={left bottom right top},clip]
    &  \includegraphics[trim={0.45cm 0 0 0},clip]{../jet/exp/exp01_benchmark_laplacian/figures/architecture_tanh_mlp_768_768_512_512_1_device_cuda_dim_50_name_weighted_laplacian_vary_batch_size.pdf}
    & \includegraphics[trim={0.45cm 0 0 0},clip]{../jet/exp/exp01_benchmark_laplacian/figures/architecture_tanh_mlp_768_768_512_512_1_device_cuda_dim_5_name_bilaplacian_vary_batch_size.pdf}
    \\[-6ex]
  \scalebox{0.85}{
    \begin{tikzpicture}
      \node {\usebox{\benchmarkLegend}};
    \end{tikzpicture}
  }
    & ($N=2048$)
    & ($N=2048$)
    & ($N=256$)
    \\[-6ex]
    \makecell{\\ \textbf{Stochastic}}
    & \includegraphics{../jet/exp/exp01_benchmark_laplacian/figures/architecture_tanh_mlp_768_768_512_512_1_batch_size_2048_device_cuda_dim_50_distribution_normal_name_laplacian_vary_num_samples.pdf}
    & \includegraphics[trim={0.45cm 0 0 0},clip]{../jet/exp/exp01_benchmark_laplacian/figures/architecture_tanh_mlp_768_768_512_512_1_batch_size_2048_device_cuda_dim_50_distribution_normal_name_weighted_laplacian_vary_num_samples.pdf}
    & \includegraphics[trim={0.45cm 0 0 0},clip]{../jet/exp/exp01_benchmark_laplacian/figures/architecture_tanh_mlp_768_768_512_512_1_batch_size_256_device_cuda_dim_5_distribution_normal_name_bilaplacian_vary_num_samples.pdf}
  \end{tabular}

  \caption{\textbf{\textcolor{tab-green}{Collapsed Taylor mode} is faster than \textcolor{tab-orange}{standard Taylor mode} and \textcolor{tab-blue}{nested first-order automatic differentiaion} while using less or the same memory, \textcolor{black!50!white}{opaque} memory consumptions are for non-differentiable computations.}
    Results are on GPU and we use a $D \to 768 \to 768 \to 512 \to 512 \to 1$ MLP with tanh activations.
    The exact computation varies the batch size, and the approximate computation fixes the batch size then varies the number of Monte-Carlo samples such that $S < D$ for the Laplacians, and $2 + 3S < 6D^2 - 3D + 6$ for the Bi-Laplacian (we can compute exactly otherwise).
    For each approach, we fit a line to the data and report the slope in \Cref{tab:benchmark}  to quantify the relative speedup and memory reduction.
  }
  \label{fig:benchmark}
\end{figure*}
%%% Local Variables:
%%% mode: LaTeX
%%% TeX-master: "../main"
%%% End:

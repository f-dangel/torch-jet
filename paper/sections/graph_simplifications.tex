In this section, we illustrate the two graph simplifications that are required to collapse Taylor mode.

We will consider collapsing the $f = \sin$ function's 2-jet as example.
Let's recall the vectorized-then-summed 2-jet of \Cref{eq:taylor-mode-scalar} for this example and assume that the Taylor coefficients are given by $\{\vx_{0,d} = \vx_0\}$ (\ie same primal value), $\{\vx_{1,d}\}$, and $\{\vx_{2,d}\}$ where $d$ indexes the directions along which we evaluate then sum:
\begin{align*}
  \begin{matrix}
    \vx_0
    \\
    \{\vx_{1,d}\}
    \\
    \{\vx_{2,d}\}
  \end{matrix}
  &\overset{\text{replicate $\vx_0$}}{\to}
    \begin{Bmatrix}
      \vx_{0,d} = \vx_0
      \\
      \vx_{1,d}
      \\
      \vx_{2,d}
    \end{Bmatrix}
    \overset{\text{2-jet}}{\to}
    \begin{Bmatrix}
      \vf_{0,d} = \sin(\vx_0)
      \\
      \vf_{1,d} = \cos(\vx_0) \odot \vx_{1,d}
      \\
      \vf_{2,d} = -\cos(\vx_0) \odot \vx_{2,d} + \cos(\vx_0) \odot \vx_{1,d} \odot \vx_{1,d}
    \end{Bmatrix}
  \\
  &\overset{\text{sum highest component}}{\to}
    \begin{matrix}
      \begin{Bmatrix}
        \vf_{0,d}
        \\
        \vf_{1,d}
      \end{Bmatrix}
      \\
      \sum_d \vf_{2,d}
    \end{matrix}
\end{align*}
where $\sin$ applies element-wise and $\odot$ denotes element-wise multiplication.
The computation graph for this procedure is displayed in the following diagram, with input and output nodes highlighted dark and light gray, and the convention that a suffix \texttt{\_d} means that a tensor collects the result of all the jets along its leading axis.
$\texttt{replicate}$ is a function that replicates a tensor $D$ times along a new leading axis, which can be basically for free and without additional memory overhead in PyTorch (using \texttt{torch.expand}). All other functions refer to those of the PyTorch API:

\begin{figure}[!h]
  \centering
  \tikzset{state/.style={rectangle, rounded corners, align=left, font=\ttfamily, draw=black, anchor=west, align=left, inner sep=5pt}}
  \tikzset{leaf/.style={fill=gray!50!white}}
  \tikzset{output/.style={fill=gray!25!white}}
  \scalebox{0.66}{
    \begin{tikzpicture}
      % draw the nodes
      \matrix [%
      ampersand replacement=\&,% to use inside a savebox
      column sep=4ex,%
      row sep=4ex,%
      ]{
        \node[state, leaf] (x0) {x0};
        \& \node[state] (x0-d) {x0\_d = replicate(x0)};
        \& \node[state, output] (f0-d) {f0\_d = sin(x0\_d)};
        \\
        \node[state, leaf] (x1-d) {x1\_d};
        \& \node[state] (df-d) {df\_d = cos(x0\_d)};
        \& \node[state, output] (f1-d) {f1\_d = einsum(``...,...\\->...'', x1\_d, df\_d)};
        \\
        \node[state, leaf] (x2-d) {x2\_d};
        \& \node[state] (d2f-d) {d2f\_d = -df\_d};
        \& \node[state] (temp1-d) {temp1\_d = einsum(``...,...,...\\->...'', d2f\_d, x1\_d, x1\_d)};
        \\
        \& \node[state] (temp2-d) {temp2\_d = einsum(``...,...\\->...'', df\_d, x2\_d)};
        \& \node[state] (f2-d) {f2\_d = temp1\_d + temp2\_d};
        \& \node[state, output] (f2) {f2 = sum(f2\_d, dim=0)};
        \\
      };

      % draw the relations
      \tikzset{arrow/.style={-Stealth}}
      \path
      (x0) edge [arrow] (x0-d)
      (x0-d) edge [arrow] (f0-d)
      (x0-d) edge [arrow] (df-d)
      (df-d) edge [arrow] (d2f-d)
      (df-d) edge [arrow] (f1-d)
      (x1-d) edge [arrow, out=25, in=170] (f1-d)
      (d2f-d) edge [arrow] (temp1-d)
      (df-d) edge [arrow, out=-25, in=65] (temp2-d)
      (x2-d) edge [arrow] (temp2-d)
      (x1-d) edge [arrow, out=-25,in=170] (temp1-d)
      (temp1-d) edge [arrow] (f2-d)
      (temp2-d) edge [arrow] (f2-d)
      (f2-d) edge [arrow] (f2)
      ;
    \end{tikzpicture}
  }
\end{figure}
%%% Local Variables:
%%% mode: LaTeX
%%% TeX-master: "../main"
%%% End:


Our simplification proceeds in two steps.
First, propagate \texttt{replicate} nodes down the graph to remove repeated computations on the same tensors (one forward traversal through the graph).
Second, propagate \texttt{sum} nodes up the graph (one backward traversal through the graph).
After applying both steps, the simplified looks as follows:

\begin{figure}[!h]
  \tikzset{state/.style={rectangle, rounded corners, align=left, font=\ttfamily, draw=black, anchor=west, align=left, inner sep=5pt}}
  \tikzset{leaf/.style={fill=gray!50!white}}
  \tikzset{output/.style={fill=gray!25!white}}
  \centering
  \scalebox{0.66}{
    \begin{tikzpicture}
      % draw the nodes
      \matrix [%
      ampersand replacement=\&,% to use inside a savebox
      column sep=4ex,%
      row sep=4ex,%
      ]{
        \node[state, leaf] (x0) {x0};
        \& \node[state] (f0) {f0 = sin(x0)};
        \& \node[state, output] (f0-d) {f0\_d = replicate(f0)};
        \\
        \node[state, leaf] (x1-d) {x1\_d};
        \& \node[state] (df) {df = cos(x0)};
        \& \node[state, output] (f1-d) {f1\_d = einsum(``d...,...\\->d...'', x1\_d, df)};
        \\
        \node[state, leaf] (x2-d) {x2\_d};
        \& \node[state] (d2f) {d2f = -df};
        \& \node[state] (temp1) {temp1 = einsum(``...,d...,d...\\->...'', d2f, x1\_d, x1\_d)};
        \\
        \node[state] (x2) {x2 = sum(x2\_d, dim=0)};
        \& \node[state] (temp2) {temp2 = einsum(``...,...\\->...'', df, x2)};
        \&
        \& \node[state, output] (f2) {f2 = temp1 + temp2};
        \\
      };

      % draw the relations
      \tikzset{arrow/.style={-Stealth}}
      \path
      (x0) edge [arrow] (f0)
      (f0) edge [arrow] (f0-d)
      (x0) edge [arrow] (df)
      (df) edge [arrow] (d2f)
      (df) edge [arrow] (f1-d)
      (x1-d) edge [arrow, out=15, in=170] (f1-d)
      (d2f) edge [arrow] (temp1)
      (df) edge [arrow, out=-45, in=65] (temp2)
      (x2-d) edge [arrow] (x2)
      (x2) edge [arrow] (temp2)
      (x1-d) edge [arrow, out=-25,in=170] (temp1)
      (temp1) edge [arrow] (f2)
      (temp2) edge [arrow] (f2)
      ;
    \end{tikzpicture}
  }
\end{figure}
%%% Local Variables:
%%% mode: LaTeX
%%% TeX-master: "../main"
%%% End:


Two important properties of the new graph are (i) the \texttt{replicate} node moved to an output node, hence all redundant computation was successfully removed (ii) The highest component \texttt{x2\_d} is immediately summed then propagated, \ie we collapsed Taylor mode.

We will now illustrate the two simplification steps in full detail.
The first stage starts from the original graph and pushes forward the replicate node, as illustrated step-by-step in \Cref{fig:push-replicate-simplification}.
The second stage starts from the graph produced by the replicate-push procedure, and propagates the final sum node up the graph, illustrated by \Cref{fig:pull-sum-simplification}.
This yields the final computation graph shown above.

\begin{figure}[!h]
  \centering
  \tikzset{state/.style={rectangle, rounded corners, align=left, font=\ttfamily, draw=black, anchor=west, align=left, inner sep=5pt}}
  \tikzset{leaf/.style={fill=gray!50!white}}
  \tikzset{output/.style={fill=gray!25!white}}
  \tikzset{replicate/.style={fill=blue!25!white}}
  \scalebox{0.58}{
    \begin{tikzpicture}
      % draw the nodes
      \matrix [%
      ampersand replacement=\&,% to use inside a savebox
      column sep=4ex,%
      row sep=4ex,%
      ]{
        \node[state, leaf] (x0) {x0};
        \& \node[state, replicate] (x0-d) {x0\_d = replicate(x0)};
        \& \node[state, output] (f0-d) {f0\_d = sin(x0\_d)};
        \\
        \node[state, leaf] (x1-d) {x1\_d};
        \& \node[state] (df-d) {df\_d = cos(x0\_d)};
        \& \node[state, output] (f1-d) {f1\_d = einsum(``...,...\\->...'', x1\_d, df\_d)};
        \\
        \node[state, leaf] (x2-d) {x2\_d};
        \& \node[state] (d2f-d) {d2f\_d = -df\_d};
        \& \node[state] (temp1-d) {temp1\_d = einsum(``...,...,...\\->...'', d2f\_d, x1\_d, x1\_d)};
        \\
        \& \node[state] (temp2-d) {temp2\_d = einsum(``...,...\\->...'', df\_d, x2\_d)};
        \& \node[state] (f2-d) {f2\_d = temp1\_d + temp2\_d};
        \& \node[state, output] (f2) {f2 = sum(f2\_d, dim=0)};
        \\
      };

      % draw the relations
      \tikzset{arrow/.style={-Stealth}}
      \path
      (x0) edge [arrow] (x0-d)
      (x0-d) edge [arrow] (f0-d)
      (x0-d) edge [arrow] (df-d)
      (df-d) edge [arrow] (d2f-d)
      (df-d) edge [arrow] (f1-d)
      (x1-d) edge [arrow, out=25, in=170] (f1-d)
      (d2f-d) edge [arrow] (temp1-d)
      (df-d) edge [arrow, out=-25, in=65] (temp2-d)
      (x2-d) edge [arrow] (temp2-d)
      (x1-d) edge [arrow, out=-25,in=170] (temp1-d)
      (temp1-d) edge [arrow] (f2-d)
      (temp2-d) edge [arrow] (f2-d)
      (f2-d) edge [arrow] (f2)
      ;
    \end{tikzpicture}
  }

  \vspace{-0.6ex}
  \noindent\rule{\textwidth}{1pt}
  \vspace{-0.6ex}

  \scalebox{0.58}{
    \begin{tikzpicture}
      % draw the nodes
      \matrix [%
      ampersand replacement=\&,% to use inside a savebox
      column sep=4ex,%
      row sep=4ex,%
      ]{
        \node[state, leaf] (x0) {x0};
        \& \node[state] (f0) {f0 = sin(x0)};
        \& \node[state, output, replicate] (f0-d) {f0\_d = replicate(f\_0)};
        \\
        \&
        \node[state, replicate] (x0-d) {x0\_d = replicate(x0)};
        \\
        \node[state, leaf] (x1-d) {x1\_d};
        \& \node[state] (df-d) {df\_d = cos(x0\_d)};
        \& \node[state, output] (f1-d) {f1\_d = einsum(``...,...\\->...'', x1\_d, df\_d)};
        \\
        \node[state, leaf] (x2-d) {x2\_d};
        \& \node[state] (d2f-d) {d2f\_d = -df\_d};
        \& \node[state] (temp1-d) {temp1\_d = einsum(``...,...,...\\->...'', d2f\_d, x1\_d, x1\_d)};
        \\
        \& \node[state] (temp2-d) {temp2\_d = einsum(``...,...\\->...'', df\_d, x2\_d)};
        \& \node[state] (f2-d) {f2\_d = temp1\_d + temp2\_d};
        \& \node[state, output] (f2) {f2 = sum(f2\_d, dim=0)};
        \\
      };

      % draw the relations
      \tikzset{arrow/.style={-Stealth}}
      \path
      (x0) edge [arrow] (f0)
      (f0) edge [arrow] (f0-d)
      (x0) edge [arrow] (x0-d)
      (x0-d) edge [arrow] (df-d)
      (df-d) edge [arrow] (d2f-d)
      (df-d) edge [arrow] (f1-d)
      (x1-d) edge [arrow, out=25, in=170] (f1-d)
      (d2f-d) edge [arrow] (temp1-d)
      (df-d) edge [arrow, out=-25, in=65] (temp2-d)
      (x2-d) edge [arrow] (temp2-d)
      (x1-d) edge [arrow, out=-25,in=170] (temp1-d)
      (temp1-d) edge [arrow] (f2-d)
      (temp2-d) edge [arrow] (f2-d)
      (f2-d) edge [arrow] (f2)
      ;
    \end{tikzpicture}
  }

  \vspace{-0.6ex}
  \noindent\rule{\textwidth}{1pt}
  \vspace{-0.6ex}

  \scalebox{0.58}{
    \begin{tikzpicture}
      % draw the nodes
      \matrix [%
      ampersand replacement=\&,% to use inside a savebox
      column sep=4ex,%
      row sep=4ex,%
      ]{
        \node[state, leaf] (x0) {x0};
        \& \node[state] (f0) {f0 = sin(x0)};
        \& \node[state, output, replicate] (f0-d) {f0\_d = replicate(f\_0)};
        \\
        \& \node[state] (df) {df = cos(x0)};
        \\
        \node[state, leaf] (x1-d) {x1\_d};
        \& \node[state, replicate] (df-d) {df\_d = replicate(df)};
        \& \node[state, output] (f1-d) {f1\_d = einsum(``...,...\\->...'', x1\_d, df\_d)};
        \\
        \node[state, leaf] (x2-d) {x2\_d};
        \& \node[state] (d2f-d) {d2f\_d = -df\_d};
        \& \node[state] (temp1-d) {temp1\_d = einsum(``...,...,...\\->...'', d2f\_d, x1\_d, x1\_d)};
        \\
        \& \node[state] (temp2-d) {temp2\_d = einsum(``...,...\\->...'', df\_d, x2\_d)};
        \& \node[state] (f2-d) {f2\_d = temp1\_d + temp2\_d};
        \& \node[state, output] (f2) {f2 = sum(f2\_d, dim=0)};
        \\
      };

      % draw the relations
      \tikzset{arrow/.style={-Stealth}}
      \path
      (x0) edge [arrow] (f0)
      (f0) edge [arrow] (f0-d)
      (x0) edge [arrow] (df)
      (df) edge [arrow] (df-d)
      (df-d) edge [arrow] (d2f-d)
      (df-d) edge [arrow] (f1-d)
      (x1-d) edge [arrow, out=25, in=170] (f1-d)
      (d2f-d) edge [arrow] (temp1-d)
      (df-d) edge [arrow, out=-25, in=65] (temp2-d)
      (x2-d) edge [arrow] (temp2-d)
      (x1-d) edge [arrow, out=-25,in=170] (temp1-d)
      (temp1-d) edge [arrow] (f2-d)
      (temp2-d) edge [arrow] (f2-d)
      (f2-d) edge [arrow] (f2)
      ;
    \end{tikzpicture}
  }

  \vspace{-0.6ex}
  \noindent\rule{\textwidth}{1pt}
  \vspace{-0.6ex}

  \scalebox{0.58}{
    \begin{tikzpicture}
      % draw the nodes
      \matrix [%
      ampersand replacement=\&,% to use inside a savebox
      column sep=4ex,%
      row sep=4ex,%
      ]{
        \node[state, leaf] (x0) {x0};
        \& \node[state] (f0) {f0 = sin(x0)};
        \& \node[state, output, replicate] (f0-d) {f0\_d = replicate(f\_0)};
        \\
        \node[state, leaf] (x1-d) {x1\_d};
        \& \node[state] (df) {df = cos(x0)};
        \& \node[state, output] (f1-d) {f1\_d = einsum(``d...,...\\->d...'', x1\_d, df)};
        \\
        \& \node[state] (d2f) {d2f = -df};
        \\
        \node[state, leaf] (x2-d) {x2\_d};
        \& \node[state, replicate] (d2f-d) {d2f\_d = replicate(d2f)};
        \& \node[state] (temp1-d) {temp1\_d = einsum(``...,...,...\\->...'', d2f\_d, x1\_d, x1\_d)};
        \\
        \& \node[state] (temp2-d) {temp2\_d = einsum(``...,d...\\->d...'', df, x2\_d)};
        \& \node[state] (f2-d) {f2\_d = temp1\_d + temp2\_d};
        \& \node[state, output] (f2) {f2 = sum(f2\_d, dim=0)};
        \\
      };

      % draw the relations
      \tikzset{arrow/.style={-Stealth}}
      \path
      (x0) edge [arrow] (f0)
      (x0) edge [arrow] (df)
      (f0) edge [arrow] (f0-d)
      (df) edge [arrow] (d2f)
      (d2f) edge [arrow] (d2f-d)
      (df) edge [arrow] (f1-d)
      (x1-d) edge [arrow, out=25, in=170] (f1-d)
      (d2f-d) edge [arrow] (temp1-d)
      (df) edge [arrow, out=-25, in=15] (temp2-d)
      (x2-d) edge [arrow] (temp2-d)
      (x1-d) edge [arrow, out=-20,in=160] (temp1-d)
      (temp1-d) edge [arrow] (f2-d)
      (temp2-d) edge [arrow] (f2-d)
      (f2-d) edge [arrow] (f2)
      ;
    \end{tikzpicture}
  }

  \vspace{-0.6ex}
  \noindent\rule{\textwidth}{1pt}
  \vspace{-0.6ex}

  \scalebox{0.58}{
    \begin{tikzpicture}
      % draw the nodes
      \matrix [%
      ampersand replacement=\&,% to use inside a savebox
      column sep=4ex,%
      row sep=4ex,%
      ]{
        \node[state, leaf] (x0) {x0};
        \& \node[state] (f0) {f0 = sin(x0)};
        \& \node[state, output, replicate] (f0-d) {f0\_d = replicate(f\_0)};
        \\
        \node[state, leaf] (x1-d) {x1\_d};
        \& \node[state] (df) {df = cos(x0)};
        \& \node[state, output] (f1-d) {f1\_d = einsum(``d...,...\\->d...'', x1\_d, df)};
        \\
        \node[state, leaf] (x2-d) {x2\_d};
        \& \node[state] (d2f) {d2f = -df};
        \& \node[state] (temp1-d) {temp1\_d = einsum(``...,d...,d...\\->d...'', d2f, x1\_d, x1\_d)};
        \\
        \& \node[state] (temp2-d) {temp2\_d = einsum(``...,d...\\->d...'', dg, x2\_d)};
        \& \node[state] (f2-d) {f2\_d = temp1\_d + temp2\_d};
        \& \node[state, output] (f2) {f2 = sum(f2\_d, dim=0)};
        \\
      };

      % draw the relations
      \tikzset{arrow/.style={-Stealth}}
      \path
      (x0) edge [arrow] (f0)
      (f0) edge [arrow] (f0-d)
      (x0) edge [arrow] (df)
      (df) edge [arrow] (d2f)
      (df) edge [arrow] (f1-d)
      (x1-d) edge [arrow, out=25, in=170] (f1-d)
      (d2f) edge [arrow] (temp1-d)
      (df) edge [arrow, out=-45, in=65] (temp2-d)
      (x2-d) edge [arrow] (temp2-d)
      (x1-d) edge [arrow, out=-25,in=170] (temp1-d)
      (temp1-d) edge [arrow] (f2-d)
      (temp2-d) edge [arrow] (f2-d)
      (f2-d) edge [arrow] (f2)
      ;
    \end{tikzpicture}
  }
  \caption{\textbf{Step-by-step illustration of pushing \texttt{replicate} nodes down a computation graph.}}
  \label{fig:push-replicate-simplification}
\end{figure}
%%% Local Variables:
%%% mode: LaTeX
%%% TeX-master: "../main"
%%% End:


\begin{figure}[!h]
  \centering
  % define node styles
  \tikzset{state/.style={rectangle, rounded corners, align=left, font=\ttfamily, draw=black, anchor=west, align=left, inner sep=5pt}}
  \tikzset{leaf/.style={fill=gray!50!white}}
  \tikzset{output/.style={fill=gray!25!white}}
  \tikzset{sum/.style={fill=red!25!white}}

  \scalebox{0.58}{
    \begin{tikzpicture}
      % draw the nodes
      \matrix [%
      ampersand replacement=\&,% to use inside a savebox
      column sep=4ex,%
      row sep=4ex,%
      ]{
        \node[state, leaf] (x0) {x0};
        \& \node[state] (f0) {f0 = sin(x0)};
        \& \node[state, output] (f0-d) {f0\_d = replicate(f0)};
        \\
        \node[state, leaf] (x1-d) {x1\_d};
        \& \node[state] (df) {df = cos(x0)};
        \& \node[state, output] (f1-d) {f1\_d = einsum(``d...,...\\->d...'', x1\_d, df)};
        \\
        \node[state, leaf] (x2-d) {x2\_d};
        \& \node[state] (d2f) {d2f = -df};
        \& \node[state] (temp1-d) {temp1\_d = einsum(``...,d...,d...\\->d...'', d2f, x1\_d, x1\_d)};
        \\
        \& \node[state] (temp2-d) {temp2\_d = einsum(``...,d...\\->d...'', df, x2\_d)};
        \& \node[state] (f2-d) {f2\_d = temp1\_d + temp2\_d};
        \& \node[state, output, sum] (f2) {f2 = sum(f2\_d, dim=0)};
        \\
      };

      % draw the relations
      \tikzset{arrow/.style={-Stealth}}
      \path
      (x0) edge [arrow] (f0)
      (x0) edge [arrow] (df)
      (f0) edge [arrow] (f0-d)
      (df) edge [arrow] (d2f)
      (df) edge [arrow] (f1-d)
      (x1-d) edge [arrow, out=25, in=170] (f1-d)
      (d2f) edge [arrow] (temp1-d)
      (df) edge [arrow, out=-45, in=65] (temp2-d)
      (x2-d) edge [arrow] (temp2-d)
      (x1-d) edge [arrow, out=-25,in=170] (temp1-d)
      (temp1-d) edge [arrow] (f2-d)
      (temp2-d) edge [arrow] (f2-d)
      (f2-d) edge [arrow] (f2)
      ;
    \end{tikzpicture}
  }

  \vspace{-0.6ex}
  \noindent\rule{\textwidth}{1pt}
  \vspace{-0.6ex}

  \scalebox{0.58}{
    \begin{tikzpicture}
      % draw the nodes
      \matrix [%
      ampersand replacement=\&,% to use inside a savebox
      column sep=4ex,%
      row sep=4ex,%
      ]{
        \node[state, leaf] (x0) {x0};
        \& \node[state] (f0) {f0 = sin(x0)};
        \& \node[state, output] (f0-d) {f0\_d = replicate(f0)};
        \\
        \node[state, leaf] (x1-d) {x1\_d};
        \& \node[state] (df) {df = cos(x0)};
        \& \node[state, output] (f1-d) {f1\_d = einsum(``d...,...\\->d...'', x1\_d, df)};
        \\
        \node[state, leaf] (x2-d) {x2\_d};
        \& \node[state] (d2f) {d2f = -df};
        \& \node[state] (temp1-d) {temp1\_d = einsum(``...,d...,d...\\->d...'', d2f, x1\_d, x1\_d)};
        \\
        \& \node[state] (temp2-d) {temp2\_d = einsum(``...,d...\\->d...'', df, x2\_d)};
        \& \node[state, sum] (temp1) {temp1 = sum(temp1\_d, dim=0)};
        \\
        \& \node[state, sum] (temp2) {temp2 = sum(temp2\_d, dim=0)};
        \& \node[state, output] (f2) {f2 = temp1 + temp2};
        \\
      };

      % draw the relations
      \tikzset{arrow/.style={-Stealth}}
      \path
      (x0) edge [arrow] (f0)
      (f0) edge [arrow] (f0-d)
      (x0) edge [arrow] (df)
      (df) edge [arrow] (d2f)
      (df) edge [arrow] (f1-d)
      (x1-d) edge [arrow, out=25, in=170] (f1-d)
      (d2f) edge [arrow] (temp1-d)
      (df) edge [arrow, out=-45, in=65] (temp2-d)
      (x2-d) edge [arrow] (temp2-d)
      (x1-d) edge [arrow, out=-25,in=170] (temp1-d)
      (temp1-d) edge [arrow] (temp1)
      (temp2-d) edge [arrow] (temp2)
      (temp1) edge [arrow] (f2)
      (temp2) edge [arrow] (f2)
      ;
    \end{tikzpicture}
  }

  \vspace{-0.6ex}
  \noindent\rule{\textwidth}{1pt}
  \vspace{-0.6ex}

  \scalebox{0.58}{
    \begin{tikzpicture}
      % draw the nodes
      \matrix [%
      ampersand replacement=\&,% to use inside a savebox
      column sep=4ex,%
      row sep=4ex,%
      ]{
        \node[state, leaf] (x0) {x0};
        \& \node[state, output] (f0) {f0 = sin(x0)};
        \\
        \node[state, leaf] (x1-d) {x1\_d};
        \& \node[state] (df) {df = cos(x0)};
        \& \node[state, output] (f1-d) {f1\_d = einsum(``d...,...\\->d...'', x1\_d, df)};
        \\
        \node[state, leaf] (x2-d) {x2\_d};
        \& \node[state] (d2f) {d2f = -df};
        \& \node[state] (temp1) {temp1 = einsum(``...,d...,d...\\->...'', d2f, x1\_d, x1\_d)};
        \\
        \& \node[state, sum] (temp2) {temp2 = einsum(``...,d...\\->...'', df, x2\_d)};
        \& \node[state, output] (f2) {f2 = temp1 + temp2};
        \\
      };

      % draw the relations
      \tikzset{arrow/.style={-Stealth}}
      \path
      (x0) edge [arrow] (f0)
      (x0) edge [arrow] (df)
      (df) edge [arrow] (d2f)
      (df) edge [arrow] (f1-d)
      (x1-d) edge [arrow, out=25, in=170] (f1-d)
      (d2f) edge [arrow] (temp1)
      (df) edge [arrow, out=-45, in=65] (temp2)
      (x2-d) edge [arrow] (temp2)
      (x1-d) edge [arrow, out=-25,in=170] (temp1)
      (temp1) edge [arrow] (f2)
      (temp2) edge [arrow] (f2)
      ;
    \end{tikzpicture}
  }

  \vspace{-0.6ex}
  \noindent\rule{\textwidth}{1pt}
  \vspace{-0.6ex}

  \scalebox{0.58}{
    \begin{tikzpicture}
      % draw the nodes
      \matrix [%
      ampersand replacement=\&,% to use inside a savebox
      column sep=4ex,%
      row sep=4ex,%
      ]{
        \node[state, leaf] (x0) {x0};
        \& \node[state, output] (f0) {f0 = sin(x0)};
        \\
        \node[state, leaf] (x1-d) {x1\_d};
        \& \node[state] (df) {df = cos(x0)};
        \& \node[state, output] (f1-d) {f1\_d = einsum(``d...,...\\->d...'', x1\_d, df)};
        \\
        \node[state, leaf] (x2-d) {x2\_d};
        \& \node[state] (d2f) {d2f = -df};
        \& \node[state] (temp1) {temp1 = einsum(``...,d...,d...\\->...'', d2f, x1\_d, x1\_d)};
        \\
        \node[state, sum] (x2) {x2 = sum(x2\_d, dim=0)};
        \& \node[state] (temp2) {temp2 = einsum(``...,...\\->...'', df, x2)};
        \& \node[state, output] (f2) {f2 = temp1 + temp2};
        \\
      };

      % draw the relations
      \tikzset{arrow/.style={-Stealth}}
      \path
      (x0) edge [arrow] (f0)
      (x0) edge [arrow] (df)
      (df) edge [arrow] (d2f)
      (df) edge [arrow] (f1-d)
      (x1-d) edge [arrow, out=15, in=170] (f1-d)
      (d2f) edge [arrow] (temp1)
      (df) edge [arrow, out=-45, in=65] (temp2)
      (x2-d) edge [arrow] (x2)
      (x2) edge [arrow] (temp2)
      (x1-d) edge [arrow, out=-25,in=170] (temp1)
      (temp1) edge [arrow] (f2)
      (temp2) edge [arrow] (f2)
      ;
    \end{tikzpicture}
  }
  \caption{\textbf{Step-by-step illustration of propagating \texttt{sum} nodes up a computation graph.}}
  \label{fig:pull-sum-simplification}
\end{figure}
%%% Local Variables:
%%% mode: LaTeX
%%% TeX-master: "../main"
%%% End:


%%% Local Variables:
%%% mode: LaTeX
%%% TeX-master: "../main"
%%% End:

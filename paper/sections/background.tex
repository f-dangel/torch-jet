\todo{Check if all bold notations are correct}


To motivate our approach, consider the one-dimensional case $f: \mathbb{R} \to \mathbb{R}$. To evaluate the 3-Jet $(J^3f)(t) = \sum_{k=1}^3 \frac{t^k}{k!} f_k$ at some $x_0$, we can expand the input variable $x_0$ to a path and build its 3-Jet $(J^3x)(t) = \sum_{k=1}^3 \frac{t^k}{k!} x_k$. Applying the chain rule \cref{eq:faa-di-bruno} provides the coefficients of $f$'s 3-Jet:
\begin{align}
\label{eq:taylor-mode-scalar}
\begin{split}
    f_0
    &=
    f(x_0)
    \\
    f_1
    &=
    \partial f(x_0) x_1
    \\
    f_2
    &=
    \partial^2 f(x_0)x_1^2
    +
    \partial f(x_0) x_2
    \\
    f_3
    &=
    \partial^3 f(x_0)x_1^3
    +
    3 \partial^2 f(x_0) x_1 x_2
    +
    \partial f(x_0) x_3.
    \end{split}
\end{align}
The general formula was originally given by Faa
di Bruno in 1856 and has since been extended to the multivariate case by Fraenkel in 1978 which serves as foundation for the Taylor-mode AD. Choosing $x_1 = 1, x_2 = x_3 = 0$ results in $f_1 = \partial f(x_0), f_2 = \partial^2 f(x_0)$ and $f_3 = \partial^3 f(x_0)$. Here, the key observation is that one obtains the identity $f_3 = \partial^3 f(x_0)$ with this initialization of the direction $(x_0,x_1,x_2)$.  To exploit this fact also in more general cases, we have to introduce some notation. We denote vectors, matrices, and tensors using boldface. Given $K$ vectors $\vv_1, \dots, \vv_K \in \mathbb{R}^D$, their tensor product is written as
\begin{equation}
    \otimes_{k=1}^K \vv_k = \vv_1 \otimes \ldots \otimes \vv_K 
    \in \left( \mathbb{R}^D \right)^{\otimes K},
\end{equation}
with components given by
\begin{equation}
     \left(\otimes_{k=1}^K \vv_k\right)_{d_1, \dots, d_K}
    = (\vv_1)_{d_1} \cdot \ldots \cdot (\vv_K)_{d_K}
    \quad \text{for } d_1, \dots, d_K \in \{1, \dots, D\}.
\end{equation}
There will be extensive use of the inner product
\begin{align}\label{eq:derivative-tensor-scalar-product}
    \left\langle
    \tA, \tB
    \right\rangle
    \coloneqq
    \sum_{d_1}
    \sum_{d_2}
    \dots
    \sum_{d_K}
    \tA_{d_1, d_2, \dots, d_K}
    \tB_{d_1, d_2, \dots, d_K},
\end{align}
where $\tA, \tB \in \left(\mathbb{R}^{D}\right)^{\otimes K}$. The $k$-th derivative of a function $\vf$ is denoted by $\partial^k \vf$.

One way to compute higher-order derivatives up to degree $K$ of a function $\vf : \mathbb{R}^D \to \mathbb{R}^m$ is to consider its (univariate) Taylor polynomial also called \emph{$K$-Jet}. To derive our approach, we provide an introduction of Taylor-mode AD ~\citep[][Chapter 13]{griewank2008evaluating}, which is the basis to compute $K$-Jets.

The evaluation of $\vf$'s $K$-Jet at an initial point $\vx_0 \in \mathbb{R}^n$ starts with the extension of $\vx_0$ to a smooth path $\vx: \mathbb{R} \to \mathbb{R}^D$ such that $\vx(0) = \vx_0$. Then, the $K$-Jet of $\vf$ is defined as $J^K \vf : \mathbb{R} \to \mathbb{R}^m$, $(J^K \vf)(t) := \sum_{k=0}^K \frac{t^k}{k!} \vf_k$ with $\vf_k := \left. \frac{d^k}{dt^k} \vf(\vx(t)) \right|_{t=0} $. The evaluation of $\vf_k$ requires the input $K$-Jet $(J^K \vx)(t) = \sum_{k=0}^K \frac{t^k}{k!} \vx_k$ as can be seen also in the simple example \eqref{eq:taylor-mode-scalar} above. The coefficients $\vx_1, \dots, \vx_K$ are selected problem specific as described later in detail. Hence, Taylor-mode AD maps $J^K\vx$ to $J^K\vf$ to compute all coefficients of $\vf$ as depicted below.

\begin{center}
  \begin{tikzpicture}
    \node[align=center] (topleft) {Extent input to smooth path \\
      $\vx(t)$};

    \node[align=center, right=4cm of topleft] (topright) {Path in output space \\ $\vf(\vx(t))$};
    \draw [-Latex] (topleft.east) to node [midway, above] {$f$} (topright.west);

    \node[align=center, below=1cm of topright] (bottomright) {Taylor polynomial of degree $K$
      \\
      $\sum_{k=0}^K \frac{t^k}{k!} \vf_k$
      \\
      {\color{maincolor}$(\vf_0, \dots, \vf_K)$}
    };
    \draw [-Latex] (topright.south) to node [midway, right] {$K$-jet} (bottomright.north);

    \node[align=center, below=1cm of topleft] (bottomleft) {Taylor polynomial of degree $K$
      \\
      $\sum_{k=0}^K \frac{t^k}{k!} \vx_k$
      \\
      {\color{maincolor} $(\vx_0, \dots, \vx_K)$}};
    \draw [-Latex] (topleft.south) to node [midway, left] {$K$-jet} (bottomleft.north);

    \draw [-Latex, maincolor] (bottomleft.east) to node [midway, above, maincolor] {Taylor mode} (bottomright.west);
  \end{tikzpicture}
\end{center}



As is common for AD methods, the main principles to obtain $J^K\vf$ rely on a decomposition of $\vf$ into elemental or primal functions and the application of the chain rule. For example, $\vf$ might be computed through an evaluation procedure or primal program that leverages the decomposition $\vf = \vg \circ \vh$. Having defined $J^K\vx$ and the decomposition of $\vf$, Taylor-mode AD calculates the $K$-Jet $\vf$ by successively evaluating all $\vf_k$ through the chain rule. 
For higher-order derivatives of multivariate functions, the chain rule is known as Faà di Bruno Formula ~\cite{arbogast1800calcul,hardy2006combinatorics,faa1857note}. The formula reads for $\vf = \vg \circ \vh$: 
\begin{align}
\label{eq:faa-di-bruno}
  \vf_k = \vg_k
  =
    \sum_{\sigma \in \partitioning(k)}
    \nu(\sigma)
    \left< 
    \partial^{|\sigma|} \vg,
    \tensorprod{s \in \sigma} \vh_s
    \right>,
\end{align}
with $\partitioning(k)$ being the integer partitioning of $k$ and multiplicity
\begin{equation}
    \nu(\sigma)
    = 
    \frac{k!}{
    \left(
    \prod_{s \in \sigma
    }
    n_s!
    \right) 
    \left(
    \prod_{s \in \sigma}
    s!
    \right)
     }.
\end{equation}
The parameter $n_s$ counts occurrences of $s$ in an integer partitioning $\sigma$, for instance $n_1(\{1,1,3\}) = 2$ and $n_3 = 1$.

There are explicit formulas (recurrence relations) for common primal functions that specify how to calculate the coefficients given an input polynomial (see ~\citep[][Chapter 13]{griewank2008evaluating}).


\paragraph{A Simple Propagation Scheme.}
Our contribution modifies Taylor-mode AD to incorporate structure of common differential operators used to train PINNs. 
To explain our approach systematically, we will use the following Taylor-mode AD scheme, which derives the $K$-Jet for a decomposed $\vf = \vg \circ \vh$ given $J^K\vx$:
\begin{align}\label{eq:taylor-mode-composition}
  \begin{split}
      \begin{pmatrix*}
        \vx_0
        \\
        \vx_1
        \\
        \vx_2
        \\
        \vdots
        \\
        \vx_K
      \end{pmatrix*}
      &\overset{\text{(\ref{eq:faa-di-bruno})}}{\to}
        \begin{pmatrix*}[l]
          \vh_0 =  \vh(\vx_0)
          \\
          \vh_1 =  \left< 
          \partial \vh(\vx_0),
          \vx_1 
          \right>
          \\
          \vh_2 = \left< 
          \partial^2 \vh(\vx_0),
          \vx_1 \otimes \vx_1 
          \right> 
          +
          \left <
          \partial \vh(\vx_0),
          \vx_2
          \right>
          \\
          \vdots
          \\
          \vh_K =
          \displaystyle \sum_{
          \mathclap{
          \sigma \in \partitioning(K)
          }
          }
          \nu(\sigma) \left<
          \partial^{|\sigma|} \vh(\vx_0),
          \tensorprod{s \in \sigma} \vx_s
          \right>
        \end{pmatrix*} 
        \\
        &\overset{\text{(\ref{eq:faa-di-bruno})}}{\to}
        \begin{pmatrix*}[l]
          \vg_0 =  \vg(\vh_0)
          \\
          \vg_1 = \left< 
          \partial \vg(\vh_0), 
          \vh_1
          \right>
          \\
          \vg_2 = \left<
          \partial^2 \vg(\vh_0), 
          \vh_1 \otimes \vh_1\right>
          + 
          \left< \partial \vg(\vh_0),
          \vh_2
          \right>
          \\
          \vdots
          \\
          \vg_K =
          \displaystyle\sum_{
          \mathclap{
          \sigma \in \partitioning(K)
          }
          } 
          \nu(\sigma) \left<
          \partial^{|\sigma|} \vg(\vh_0),
          \tensorprod{s \in \sigma} \vh_s
          \right>
        \end{pmatrix*}
        \\
        &\overset{\text{(\ref{eq:faa-di-bruno})}}{=}
        \begin{pmatrix*}[l]
          \vf_0 =  \vf(\vx_0)
          \\
          \vf_1 = \left<
          \partial \vf(\vx_0), 
          \vx_1
          \right>
          \\
          \vf_2 = \left<
          \partial^2 \vf(\vx_0), 
          \vx_1 \otimes \vx_1
          \right>
          + 
          \left< \partial \vf(\vx_0),
          \vx_2
          \right>
          \\
          \vdots
          \\
          \vf_K =
          \displaystyle\sum_{
          \mathclap{
          \sigma \in \partitioning(K)
          }
          }
          \nu(\sigma) \left< 
          \partial^{|\sigma|} \vf(\vx_0),
          \tensorprod{s \in \sigma} \vx_s
          \right>
        \end{pmatrix*}
    \end{split}
\end{align}
This procedure explains how to (i) propagate Taylor coefficients through the compute graph of $\vf$'s decomposition, and (ii) how the result relates to the input Jet.




%%% Local Variables:
%%% mode: LaTeX
%%% TeX-master: "../main"
%%% End:

\documentclass{article}

% use numbers for citations to save space
\PassOptionsToPackage{numbers, compress, sort}{natbib}

\def\status{preprint}
\usepackage[\status]{../paper/neurips_2025}

% Load the same preamble files as the paper
\input{../paper/preamble/custom_early.tex}
\input{../paper/preamble/neurips_2025.tex}
\input{../paper/preamble/goodfellow.tex}
% ===================================================================
% MATH
% ===================================================================
\usepackage{nicefrac} % fractions that fit into inline text
\usepackage{dsfont} % for \mathds command
\usepackage[%
exponent-product=\ensuremath{\cdot},%
group-minimum-digits={3}%
]{siunitx} % \num command for pretty-formatting large numbers
\newcommand{\mathemph}[1]{{\color{maincolor} #1}}

% ===================================================================
% REFERENCES
% ===================================================================
\usepackage{cleveref} % automatically adds type of reference, MUST BE LOADED AFTER AMSMATH
\crefname{section}{\S\!\!}{\S\!\!} % use paragraph symbol for Section
\crefname{appendix}{\S\!\!}{\S\!\!} % use paragraph symbol for Appendix

% ===================================================================
% TODOS, COMMENTS & WRITING
% ===================================================================
\usepackage{todonotes} % for TODOs, loads xcolor with []
\usepackage{comment} % for comment environment
\usepackage{xspace}
\newcommand*{\ie}{i.e.\@\xspace}
\newcommand*{\iid}{i.i.d.\@\xspace}
\newcommand*{\wrt}{w.r.t.\@\xspace}
\newcommand*{\eg}{e.g.\@\xspace}
\newcommand*{\Ie}{I.e.\@\xspace}
\newcommand*{\Eg}{E.g.\@\xspace}

% ===================================================================
% FIGURES & COLORS
% ===================================================================
\usepackage{wrapfig} % side-wrap text next to a figure
\usepackage{subcaption} % \subfigure environment
\usepackage{tabularx} % tables with automatic line break
\captionsetup[subfigure]{% subfigure captions are left-aligned
  justification=justified,%
  singlelinecheck=false,%
}%
\usepackage{tikz} % for drawings in LaTeX
\usetikzlibrary{
  arrows.meta, % for prettier arrows
  matrix, % for matrix of nodes
}

% VECTOR INSTITUTE PRIMARY COLORS
\definecolor{VectorBlack}{RGB}{34, 34, 34}
\definecolor{VectorGray}{RGB}{239, 238, 237}
% VECTOR INSTITUTE SECONDARY COLORS
\definecolor{VectorBlue}{RGB}{59, 69, 227}
\definecolor{VectorPink}{RGB}{253, 8, 238}
\definecolor{VectorOrange}{RGB}{250, 173, 26}
\definecolor{VectorTeal}{RGB}{82, 199, 222}

% PAPER COLOR THEME
\colorlet{maincolor}{VectorBlue}
\colorlet{secondcolor}{VectorPink}
\colorlet{thirdcolor}{VectorOrange}
\colorlet{fourthcolor}{VectorTeal}
\colorlet{fifthcolor}{VectorGray}

% MATPLOTLIB COLORS
\definecolor{tab:orange}{rgb}{1.0, 0.498, 0.055}
\definecolor{tab:blue}{rgb}{0.121, 0.466, 0.705}
\definecolor{tab:green}{rgb}{0.173, 0.627, 0.173}

% ===================================================================
% LINKS & REFERENCES
% ===================================================================
\hypersetup{%
  colorlinks,
  citecolor = maincolor,%
  linkcolor = maincolor,%
  urlcolor = secondcolor,%
}%

% ===================================================================
% SPECIAL SYMBOLS
% ===================================================================
\usepackage{pifont} % for check and cross marks
% commands from https://tex.stackexchange.com/a/42620
\newcommand{\cmark}{\ding{51}}
\newcommand{\xmark}{\ding{55}}

% ===================================================================
% ALGORITHMS
% ===================================================================
\usepackage{algorithm}
\usepackage{algpseudocode}

% ===================================================================
% TABLES
% ===================================================================
\usepackage{multirow}
\usepackage{array} % vertically centered table cells
\usepackage{makecell} % table cells with multiple lines of text

%%% Local Variables:
%%% mode: latex
%%% TeX-master: "../main"
%%% End:

\newcommand{\papertitle}{%
  Accelerating Differential Operators Through Linearity
}
\title{\papertitle}

% The \author macro works with any number of authors. There are two commands
% used to separate the names and addresses of multiple authors: \And and \AND.
%
% Using \And between authors leaves it to LaTeX to determine where to break the
% lines. Using \AND forces a line break at that point. So, if LaTeX puts 3 of 4
% authors names on the first line, and the last on the second line, try using
% \AND instead of \And before the third author name.

\author{%
  Felix Dangel\thanks{Equal contribution}\\
  Vector Institute \\
  Toronto \\ Canada \\
  \texttt{fdangel@vectorinstitute.ai} \\
  \And
  Marius Zeinhofer\\
  Seminar for Applied Mathematics, ETH Z\"urich, \\
  \texttt{marius.zeinhofer@uniklinik-freiburg.de}
}
%%% Local Variables:
%%% mode: latex
%%% TeX-master: "../main"
%%% End:


% Title and author
\renewcommand{\papertitle}{Reading Group: Future Research Ideas}

\begin{document}

\maketitle

\begin{abstract}
  This document contains a collection of related topics and future research ideas discussed in a reading group.
  The goal is to identify executable projects that we can tackle ourselves, or with students.
\end{abstract}

\tableofcontents

\section{Discussed Papers}

\subsection{[2025-06-20] Efficiently Access Diffusion Fisher: Within the Outer Product Span Space (ICML 2025, \url{https://www.arxiv.org/abs/2505.23264})}

\subsection{[2025-07-xxx]\url{https://arxiv.org/pdf/2506.05918} This paper discussed the artificial extension of the classical PINN PDE loss by terms that are obtain by differentiating the original PDE. Thus, we obtain e.g., derivatives of laplace. \textbf{Connection to our work: we could generate efficient schemes to compute the mixed-direction derivatives efficiently}}

\subsection{Candidates for future meetings}
\begin{itemize}
    \item A Physics Informed Neural Network Approach to Solution and Identification of Biharmonic Equations of Elasticity (\url{https://arxiv.org/abs/2108.07243})
    \item Identifying Memorization of Diffusion Models through p-Laplace Analysis (\url{https://arxiv.org/abs/2505.08246})
    \item Mollifier Layers: Enabling Efficient High-Order
Derivatives in Inverse PDE Learning (\url{https://arxiv.org/pdf/2505.11682})
    \item Arbitrary-Order Derivatives of Quantum Chemical Methods via Automatic Differentiation \textbf{Uses nested Jax derivative calls to evaluate arbitrary derivatives} 
\end{itemize}

\section{Brain Dump}

\begin{itemize}
    \item \textbf{[Makes no sense at the moment] Computing fractional derivatives with AD: are there applications?}
    This idea is a bit crazy. But assume we want to compute $\nicefrac{\partial^3}{\partial x^2 \partial y}$. We could do this by nesting $\nicefrac{\partial^{3/2}}{\partial x \partial y^{1/2}}$.
    Is there any literature on automatic fractional differentiation?
    Useful link could be \url{https://github.com/fracdiff/fracdiff}.

    \item \textbf{[Not promising after first exploration] Finding tensor factorizations that are easy to compute with Taylor mode}

    \vspace{1ex}
    
    \textbf{Goal:} We want to generalize the notion of a symmetric decomposition for a symmetric matrix to tensors. 
    We are interested in that because we know some nice computational tricks once we have such a factorization.
    \vspace{1ex}

    \textbf{Problem description:} Assume we have a rank-$K$ tensor $\tC \in \sR^{D \times \dots \times D}$. Assume further that this tensor is symmetric \wrt exchanging any two indices, \ie $[\tC]_{\dots,i,\dots,j,\dots} = [\tC]_{\dots,j,\dots,i,\dots}$.
    We want to find a symmetric decomposition of the following form:
    \begin{align*}
        \tC
        =
        \sum_i s_i \underbrace{\vv_i \otimes \vv_i \otimes \dots \otimes \vv_i}_{K \text{times}}
        =
        \sum_i s_i (\vv_i)^{\otimes K}
    \end{align*}
    with binary $s_i \in \{1, -1\}$, and $\vv_i \in \sR^D$.
    The integer $R$ is called the (symmetric) tensor rank (\url{https://epubs.siam.org/doi/10.1137/060661569}).
    
    \vspace{1ex}
    
    My questions are 
    
    (i) how is such a factorization called in the literature? $\to$ It is called symmetric tensor factorization (\url{https://link.springer.com/article/10.1007/s10915-020-01233-w}, \url{https://www.osti.gov/biblio/1725841}, \url{https://arxiv.org/pdf/2211.09121})
    
    (ii) does it always exist? $\to$ It depends on the tensor rank $R$. If $\mathrm{rank}(\mathbf{C})\leq R,$ then it always exists. But getting the tensor rank is an NP-hard problem. (\url{https://arxiv.org/abs/0911.1393})
    
    (iii) how would one compute it numerically? $\to$ We optimize the objective function $L(\bm{v}_1,\dots,\bm{v}_R)=| \mathbf{C} - \sum_{r=1}^R s_r \bm{v}_r \otimes \bm{v}_r \otimes \dots \otimes \bm{v}_r|^2_{F}$ where the integer $R$ is regarded as an hyper-parameter. Please check section 2.6 in this paper (\url{https://www.osti.gov/biblio/1725841}  
    
    \vspace{1ex}

    \textbf{Matrix case:} For $K=2$, \ie the matrix case $\tC \to \mC \in \sR^{D \times D}$ and $\mC$ symmetric, this is easy and we know that such a decomposition always exists: We can form the eigen-decomposition $\mC = \sum_i \lambda_i \vnu_i \vnu_i^\top$ with $i$th eigenpair $(\lambda_i, \vnu_i)$, then write
    \begin{align*}
        \mC
        =
        \sum_i \sign(\lambda_i) \vv_i \vv_i^\top
        =
        \sum_i \sign(\lambda_i) (\vv_i)^{\otimes 2}
    \end{align*}
    with $\vv_i = \sqrt{|\lambda_i|}\vnu_i$. 
    \vspace{1ex}
    
    But how does this generalize to arbitrary tensors with $K>2$?

\item \textbf{Use multivariate schemes + TTC to get better univariate schemes for mixed-directions:}
\textbf{Example}: If we search for $\langle \partial^2 \vg, \vh_i \otimes \vh_j \rangle$ we could use TTC to get 
\begin{equation}
    \langle \partial^2 \vg, \vh_i \otimes \vh_j \rangle = \sum_{q} \gamma_{pq} \langle \partial^2 \vg, \left([q]_1 \vh_i + [q]_2 \vh_j \right)^{\otimes 2} \rangle
\end{equation}
To compute the right-side we need the values of
\begin{align}
    &\langle \partial^2 \vg, \vh_i^{\otimes 2} \rangle \\
    &\langle \partial^2 \vg, \vh_j^{\otimes 2} \rangle \\
    &\langle \partial^2 \vg, \left(\vh_i + \vh_j \right)^{\otimes 2} \rangle,
\end{align}
which we get via jets. The main point I want to make is, we can have multivariate schemes where we can collapse more compared to univariate. After collapsing we can transform the terms that are only available in the multivariate case into jets using TTC formula. It could be that a lot of these sum of jets share common coefficients ($\gamma_{pq}$) and jets. 
\end{itemize}

\bibliography{../paper/references}
\bibliographystyle{../paper/icml2024}

\end{document}

%%% Local Variables:
%%% mode: LaTeX
%%% TeX-master: t
%%% End:

Computing differential operators of high-dimensional functions is a critical component in scientific machine learning, particularly for physics-informed neural networks and variational Monte Carlo. 
While Taylor-mode automatic differentiation promises efficient computation of higher-order derivatives, we found that vanilla implementations often underperform compared to nested backpropagation. 
Our work introduces collapsed Taylor-mode AD, a simple yet effective optimization that propagates the sum of highest-order coefficients directly through the computational graph. 
This approach (1) unifies and generalizes recent advances in forward-mode schemes, showing that the forward Laplacian emerges naturally from collapsing standard Taylor mode, while extending to other differential operators and stochastic variants, 
(2) demonstrates that such optimizations can be achieved through simple graph rewrites based on linearity, making it amenable to integration into existing just-in-time compilers without requiring specialized interfaces, and (3)
provides substantial empirical improvements over both vanilla Taylor mode and nested backpropagation, with up to 2x speedup for Laplacian operators and 9x for randomized Bi-harmonic operators, while often using less memory. 
Our PyTorch implementation and experiments confirm that these theoretical benefits translate into practical performance gains. 
The success of collapsed Taylor mode suggests that forward-mode AD schemes, when properly optimized, can outperform the traditional backpropagation approach for computing PDE operators. 
We believe this work takes an important step toward making Taylor mode a practical alternative in scientific machine learning, while maintaining ease of use through potential compiler integration. 
Future work could focus on integrating these optimizations directly into ML framework compilers, extending support to more primitive operations, and exploring additional graph-based optimizations for automatic differentiation. 
%%% Local Variables:
%%% mode: LaTeX
%%% TeX-master: "../main"
%%% End:

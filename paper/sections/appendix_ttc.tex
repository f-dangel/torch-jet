This section introduces the notation used in \cref{eq:ttc_general}. 
The left side of the equation, is defined as the derivative of $\vf$ with respect to an auxiliary variable $\vz = (z_1, \dots, z_p) \in \mathbb{R}^p$ of order $\vk = (k_1, \dots, k_p) \in \mathbb{N}^p$, $p \in \mathbb{N}$. In this context, we understand $\mathbb{N}^p$ as multi-index set, and the derivative of order $\vk$ is defined as 
\begin{equation}
    \frac{\partial^{\vk}}{\partial \vz^\vk} :=  \frac{\partial^{k_1} \dots \partial^{k_p}}{\partial z_1^{k_1} \dots \partial z_p^{k_p}}.
\end{equation}
Corresponding to the auxiliary variables $\vz$ there is the matrix $\boldsymbol{S} = (\vs_1, \dots, \vs_p) \in \mathbb{R}^{n \times p}$, which is used to encode the combination of the partial derivatives. Using $|\vk| = \sum_{n=1}^p k_n$, the left side of the equation computes the derivatives $\partial^{|\vk|} \vf(\vx)[\underbrace{\vs_1, \dots, \vs_1}_{k_1}, \dots, \underbrace{\vs_p, \dots, \vs_p}_{k_p}]$. The right side of the formula sums over all multi-indices $\vl \in \mathbb{N}^p$ such that $|\vl| = |\vk|$. If $p=2$ and $|\vk| = 4$ the index-set consists of $\left\{ (4,0), (0, 4), (3, 1), (1, 3),  (2, 2)\right\}$. 

The coefficient $\gamma_{\vk \vl}$ is given as \todo{explain set of indices for the summation}
\begin{equation}
\label{eq:ttc_coeff}
    \gamma_{\vk \vl} = \sum_{0 < \vm \leq \vk} (-1)^{|\vk - \vm|} 
    \left(
    \begin{matrix}
        \vk \\
        \vm
    \end{matrix}
    \right) 
    \left( 
    \begin{matrix}
        |\vk|\frac{\vm}{|\vm|} \\
        \vl
    \end{matrix} 
    \right)  
    \left( 
    \frac{|\vm|}{|\vk|}
    \right)^{|\vk|}.
\end{equation}
The summation ranges over the set $\left\{ \vm = (m_1, \dots, m_p) \in \mathbb{N}^p \mid m_1 \leq k_1, \dots, m_p \leq k_p, |m| > 0 \right\}$. Furthermore, we utilize the generalized binomial coefficient
\begin{equation}
    \left(
       \begin{matrix}
        a \\
        b
    \end{matrix}
    \right) = \prod_{i=0}^{b-1} \frac{a - i}{b - i}
\end{equation}
to allow the computation for all $a \in \mathbb{R}$ and $b \in \mathbb{N}$, which is defined to be $1$ if $b=0$. The generalized binomial coefficient of multi-indices, is the product of all generalized binomial coefficients of the components of the indices:
$
\left(
    \begin{matrix}
        \vk \\
        \vl
    \end{matrix}
    \right) 
    = 
    \prod_{i=1}^p 
    \left(
       \begin{matrix}
        k_i \\
        l_i
    \end{matrix}
    \right).
$ 
This notation also includes cases where the multi-index has components of $\mathbb{R}$.
The notation $\boldsymbol{S}\vl$ is understood as a matrix-vector product.


\subsection{Examples} \label{sec:biharm_felix_approach}
\paragraph{Felix's approach to computing the biharmonic operator.}
In the paragraph above, we learned that if we can write our coefficient tensor as $\tC = \sum_d (\vv_d)^{\otimes K}$, then we can use $K$-th order Taylor mode to efficiently compute it, while collapsing the sum.
Let's look at the biharmonic operator now.
It reads $\Delta^2 f(\vx) =\sum_{i=1}^D \sum_{j=1}^D \partial^4 f(\vx) [\ve_i, \ve_i, \ve_j, \ve_j]$. 
Let's write down the coefficient tensor for the biharmonic.
It is $\tC = \sum_d \sum_{d'} (\ve_d)^{\otimes 2} (\ve_{d'})^{\otimes 2}$.
What do we see? This coefficient cannot be written in our desired form; it is slightly more general: We have a sum, but instead of a 4-th tensor power of the same vector, we get the product of two 2-nd tensor powers (note that in the most general case we have four different vectors). Since we now have to collapse the fourth derivative tensor along \emph{different} dimensions, we cannot use 4-th order Taylor mode to compute it!

Since we cannot use 4-th order Taylor mode, let's increase $K$ to $>4$ and look for terms of the form $\partial^4 f(\vx) [\bullet, \bullet, \blacktriangle, \blacktriangle]$.
After starring at the Faa di Bruno cheatsheet (\Cref{sec:faa-di-bruno-cheatsheet}) for a bit, we can see that these terms show up in the $K=6$-th order Taylor mode.
Let's write down how the 6-th derivative looks like if we set the Taylor coefficients to $\vx_3 = \vx_4 = \vx_5 = \vx_6 = \vzero$.
The output of Taylor mode is then
\begin{subequations}
\begin{align}\label{eq:felix-biharmonic-jet1}
    \vf_6
    =
    \partial^{6} f[\vx_1, \vx_1, \vx_1, \vx_1, \vx_1, \vx_1]
    +
    15 \partial^5 f[\vx_1, \vx_1, \vx_1, \vx_1, \vx_2]
    +
    {\color{blue}
    45 \partial^4 f[\vx_1, \vx_1, \vx_2, \vx_2]
    }
    +
    15 \partial^3 f[\vx_2, \vx_2, \vx_2]
\end{align}
Notice the \textcolor{blue}{blue term}, which has the same structure as the summands we want to compute for the biharmonic.
Now we have to find a way to kill the other terms.
We can kill the $\partial^5$ and $\partial^3$ term by noting that they contain $\vx_2$ an odd number of times.
We can use this to evaluate the negative of those terms by evaluating the Taylor mode from \Cref{eq:felix-biharmonic-jet1}, but using $-\vx_2$ instead of $\vx_2$.
This gives
\begin{align}\label{eq:felix-biharmonic-jet2}
    \vf_6
    =
    \partial^{6} f[\vx_1, \vx_1, \vx_1, \vx_1, \vx_1, \vx_1]
    -
    15 \partial^5 f[\vx_1, \vx_1, \vx_1, \vx_1, \vx_2]
    +
    {\color{blue}
    45 \partial^4 f[\vx_1, \vx_1, \vx_2, \vx_2]
    }
    -
    15 \partial^3 f[\vx_2, \vx_2, \vx_2]
\end{align}
To kill the $\partial^6$ term, we evaluate the same jet as in \Cref{eq:felix-biharmonic-jet1}, but set $\vx_2 = \vzero$. This gives
\begin{align}\label{eq:felix-biharmonic-jet3}
    \vf_6
    =
    \partial^{6} f[\vx_1, \vx_1, \vx_1, \vx_1, \vx_1, \vx_1]
\end{align}
Now, let's put things together.
Let's evaluate  \Cref{eq:felix-biharmonic-jet1} $+$ \Cref{eq:felix-biharmonic-jet2} $- 2 $ \Cref{eq:felix-biharmonic-jet3}, which gives
\begin{align}
    {\color{blue}
    90 \partial^4 f[\vx_1, \vx_1, \vx_2, \vx_2]
    }
\end{align}
\end{subequations}
Cool! \textbf{We now found a way to compute one summand of the biharmonic operator by computing three $6$-jets and then linearly combining their outputs.}
(NOTE: I think this is somewhat similar to Chapter 13.3 in Andrea's book). These jets only differ in how we set the second-order Taylor coefficient: $+1 \vx_2, -1 \vx_2, 0 \vx_2$.
We should see plenty of cancellations from the $\pm \vx_2$ in the compute graph.
It might be a headache to write the simplifications that take care of this.

Let's do a quick analysis of this approach: For each jet, we have to forward-propagate $1 + 6$ vectors.
That's $3 + 18$ vectors in total.
With a fully-naive approach, we can now compute the biharmonic by repeating this procedure $D^2$ times (once per element). 
That would be $21 D^2$ vectors in total.

However, we may hope to reduce this by removing some redundancies and collapsing sums.

First, note that all three jets share the same $\vx_0$.
Hence, we actually need $1 + 18$ rather than $3 + 18$ vectors for one element of the biharmonic; and we can also share this computation over all summands.
This gets us down to $1 + 18 D^2$ vectors total.

Second, note that we can pull the sum from the biharmonic inside each of the three jets.
This brings us down from $1 + 18$ to $1 + 3 + 15$ where only the $15$ scales with $D^2$.
Overall, collapsing the sum brings us down to $1 + 3 + 15D^2$.

I think there is more to collapse, but for that we need to write out all sums.
I believe we could get something that looks very similar to what Tim wrote down below.

\documentclass{article}

% use numbers for citations to save space
\PassOptionsToPackage{numbers, compress}{natbib}

% either empty (for submission), 'preprint', or 'final'
\def\status{}
\usepackage[\status]{neurips_2024}

\input{preamble/custom_early.tex}
\input{preamble/neurips_2024.tex}
% follow DL notation from the Goodfellow book
\input{preamble/goodfellow.tex}
% ===================================================================
% MATH
% ===================================================================
\usepackage{nicefrac} % fractions that fit into inline text
\usepackage{dsfont} % for \mathds command
\usepackage[%
exponent-product=\ensuremath{\cdot},%
group-minimum-digits={3}%
]{siunitx} % \num command for pretty-formatting large numbers
\newcommand{\mathemph}[1]{{\color{maincolor} #1}}

% ===================================================================
% REFERENCES
% ===================================================================
\usepackage{cleveref} % automatically adds type of reference, MUST BE LOADED AFTER AMSMATH
\crefname{section}{\S\!\!}{\S\!\!} % use paragraph symbol for Section
\crefname{appendix}{\S\!\!}{\S\!\!} % use paragraph symbol for Appendix

% ===================================================================
% TODOS, COMMENTS & WRITING
% ===================================================================
\usepackage{todonotes} % for TODOs, loads xcolor with []
\usepackage{comment} % for comment environment
\usepackage{xspace}
\newcommand*{\ie}{i.e.\@\xspace}
\newcommand*{\iid}{i.i.d.\@\xspace}
\newcommand*{\wrt}{w.r.t.\@\xspace}
\newcommand*{\eg}{e.g.\@\xspace}
\newcommand*{\Ie}{I.e.\@\xspace}
\newcommand*{\Eg}{E.g.\@\xspace}

% ===================================================================
% FIGURES & COLORS
% ===================================================================
\usepackage{wrapfig} % side-wrap text next to a figure
\usepackage{subcaption} % \subfigure environment
\usepackage{tabularx} % tables with automatic line break
\captionsetup[subfigure]{% subfigure captions are left-aligned
  justification=justified,%
  singlelinecheck=false,%
}%
\usepackage{tikz} % for drawings in LaTeX
\usetikzlibrary{
  arrows.meta, % for prettier arrows
  matrix, % for matrix of nodes
}

% VECTOR INSTITUTE PRIMARY COLORS
\definecolor{VectorBlack}{RGB}{34, 34, 34}
\definecolor{VectorGray}{RGB}{239, 238, 237}
% VECTOR INSTITUTE SECONDARY COLORS
\definecolor{VectorBlue}{RGB}{59, 69, 227}
\definecolor{VectorPink}{RGB}{253, 8, 238}
\definecolor{VectorOrange}{RGB}{250, 173, 26}
\definecolor{VectorTeal}{RGB}{82, 199, 222}

% PAPER COLOR THEME
\colorlet{maincolor}{VectorBlue}
\colorlet{secondcolor}{VectorPink}
\colorlet{thirdcolor}{VectorOrange}
\colorlet{fourthcolor}{VectorTeal}
\colorlet{fifthcolor}{VectorGray}

% MATPLOTLIB COLORS
\definecolor{tab:orange}{rgb}{1.0, 0.498, 0.055}
\definecolor{tab:blue}{rgb}{0.121, 0.466, 0.705}
\definecolor{tab:green}{rgb}{0.173, 0.627, 0.173}

% ===================================================================
% LINKS & REFERENCES
% ===================================================================
\hypersetup{%
  colorlinks,
  citecolor = maincolor,%
  linkcolor = maincolor,%
  urlcolor = secondcolor,%
}%

% ===================================================================
% SPECIAL SYMBOLS
% ===================================================================
\usepackage{pifont} % for check and cross marks
% commands from https://tex.stackexchange.com/a/42620
\newcommand{\cmark}{\ding{51}}
\newcommand{\xmark}{\ding{55}}

% ===================================================================
% ALGORITHMS
% ===================================================================
\usepackage{algorithm}
\usepackage{algpseudocode}

% ===================================================================
% TABLES
% ===================================================================
\usepackage{multirow}
\usepackage{array} % vertically centered table cells
\usepackage{makecell} % table cells with multiple lines of text

%%% Local Variables:
%%% mode: latex
%%% TeX-master: "../main"
%%% End:

\newcommand{\papertitle}{%
  Accelerating Differential Operators Through Linearity
}
\title{\papertitle}

% The \author macro works with any number of authors. There are two commands
% used to separate the names and addresses of multiple authors: \And and \AND.
%
% Using \And between authors leaves it to LaTeX to determine where to break the
% lines. Using \AND forces a line break at that point. So, if LaTeX puts 3 of 4
% authors names on the first line, and the last on the second line, try using
% \AND instead of \And before the third author name.

\author{%
  Felix Dangel\thanks{Equal contribution}\\
  Vector Institute \\
  Toronto \\ Canada \\
  \texttt{fdangel@vectorinstitute.ai} \\
  \And
  Marius Zeinhofer\\
  Seminar for Applied Mathematics, ETH Z\"urich, \\
  \texttt{marius.zeinhofer@uniklinik-freiburg.de}
}
%%% Local Variables:
%%% mode: latex
%%% TeX-master: "../main"
%%% End:


\definecolor{darkgreen}{rgb}{0,0.6,0}
%\newcommand{\AK}[2][]{\alternatingtodo[color=red!40, #1]{#2 }}
\newcommand{\temp}[1]{\textcolor{blue}{#1}}
\newcommand{\AW}[1]{\textcolor{darkgreen}{#1}}

\begin{document}

\maketitle

\begin{abstract}
  For the accelerated evaluation differential operators, we explore automated computational graph simplifications based on the concept of linearity.
  These simplification can be used for many common differential operators like the Laplacian that computes the sum of diagonal elements of the Hessian using Taylor mode automatic differentiation (\texttt{jet}s).
  We show that the required Taylor coefficients can first be summed, then propagated, which reduces the overall computational cost.
  Due to the simplicity of this simplification (propagating a sum up a computational graph), we argue that it could (or should) be performed by the just-in-time (\texttt{jit}) compiler in machine learning frameworks.
  Our preliminary experiments achieve promising, fully automated, speed-ups, which we believe can easily be integrated into automatic differentiation libraries.
\end{abstract}

\AW{comments
 \begin{itemize}
     \item compute graph, computation graph $\Rightarrow$ computational graph?
     \item Fig. 1 wird nicht referenziert, ich teile die Bedenken, dass die Graphik sehr ähnlich ist, 
     \item ich mache mal für Sec. 2 einen neuen Vorschlag
     \item univariant Taylor polynomial $\Rightarrow$ $K$-jet, ich würde $K$-jet mit kleinem j schreiben, wie ist denn da so die Meinung?
 \end{itemize}
}\section{Introduction}\label{sec:introduction}
\todo{Biharmonic Operator instead of Bi-laplace!}\todo{Make clear that we consider PDE operators that consist of sums}Using neural networks to learn functions constrained by physical laws is a popular trend in scientific machine learning \cite{carleo2017solving, pfau2020ab, hermann2020deep, karniadakis2021physics, raissi2019physics, hu2023hutchinson, sun2020global}.
Typically, the physics is encoded through partial differential equations (PDEs) that the neural net must satisfy.
The associated loss functions require evaluating differential operators \wrt the network's input, rather than weights.
Evaluating differential operators remains a computational challenge, especially if they contain high-order derivatives.

\paragraph{Computing PDE operators.} Two important fields that require the evaluation of PDE operators are variational Monte-Carlo (VMC) simulations and physics-informed neural networks (PINNs).
VMC employs neural networks as an ansatz for the Schr\"odinger equation \cite{carleo2017solving, pfau2020ab, hermann2020deep} and demands computing the net's Laplacian, i.e., the Hessian trace usually in a non-differentiable fashion.
PINNs represent PDE solutions as a neural network and train the network via minimizing the residuals of the governing equations \cite{raissi2019physics, karniadakis2021physics}. For instance, Kolmogorov-type equations—including the Fokker-Planck and Black-Scholes equation—requires the differentiable evaluation of weighted second-order derivatives on high-dimensional spatial domains \cite{hu2023hutchinson, sun2024dynamical}. For PINN applications in elasticity, the Biharmonic operator \cite{vahab_physics-informed_2022} requires the computation of fourth-order derivatives, making it a commonly considered a benchmarking problem for computing high-order derivatives in PINNs \cite{hu2023hutchinson, vikas_biharm, shi2024stochastic}.

\paragraph{Is backpropagation all we need?}
% Alternative: The gap between theory and practice
Although nesting first-order automatic differentiation (AD) to compute high-order derivatives scales poorly, this approach is common practice.
\todo{Felix, M: still needs citation.}
A promising alternative to nested backpropagation, especially for higher-order derivatives is \emph{Taylor-mode AD}~\cite[or simply \emph{Taylor-mode},][\S13]{griewank2008evaluating}, introduced to the machine learning community in \citeyear{bettencourt2019taylor} and the JAX ecosystem in \cite{bradbury2018jax}.
However, we observe empirically that vanilla Taylor-mode is often not enough to improve upon nested backpropagation: Evaluating the Laplacian of a 5-layer $\tanh$ activated MLP using JAX' \emph{Taylor mode is 50\% slower} than nested backpropagation via computing, then tracing, the Hessian via Hessian-vector products \cite{pearlmutter1994fast,dagreou2024how}. This calls into question the relevance of Taylor mode for computing common PDE operators.

\paragraph{Improved Taylor mode schemes}
However, recent works have successfully demonstrated the potential of modified forward propagation schemes.
For the (weighted) Laplacian, \citet{li2023forward, li2024dof} developed a special forward propagation framework called the \emph{forward Laplacian}.
Relating to our example above, JAX' forward Laplacian \cite{gao2023folx} is roughly twice as fast as nested backpropagation at reduced memory costs.
While the forward Laplacian does not rely on Taylor mode, recent work pointed out a connection \cite{dangel2024kroneckerfactored}; however, it remains unclear if efficient forward schemes can be derived for other differential operators. Another line of work concerns the stochastic approximation of differential operators in high-dimensions \citet{shi2024stochastic}, relying on Taylor mode with suitably sampled random directions. In this work we identify a mechanism to rewrite the computational graph of standard Taylor mode, applicable to general PDE operators and stochastic Taylor mode: Precisely, our contributions are:
\todo{Felix: Make it more explicit that we evaluate Taylor mode in multiple directions and then sum the results. The sum can be pulled inside.}

\begin{figure*}[!t]
  \centering
  \begin{minipage}[b]{0.42\linewidth}
    \centering
    \input{figures/vanilla_taylor_not_enough.tex}

    \caption{\textbf{$\blacktriangle$ Vanilla Taylor mode is not enough to beat nested 1\textsuperscript{st}-order AD.}
      Illustrated for computing the Laplacian of a $\mathrm{tanh}$-activated $50 \!\to\! 768 \!\to\! 768 \!\to\! 512 \!\to\! 512 \!\to\! 1$
      MLP with JAX (+ \texttt{jit}) on GPU (details in \Cref{sec:jax-benchmark}).
      We show how to automatically obtain the specialized forward Laplacian through simple graph transformations of vanilla Taylor mode.
    }\label{fig:vanilla-taylor-not-enough}

    \vspace{0.25ex}
    \caption{\textbf{$\blacktriangleright$ Our collapsed Taylor mode directly propagates the sum of highest degree coefficients.}
      Visualized for propagating four $K$-jets through a $\sR^5 \!\to\! \sR^3 \!\to\! \sR$ function ($K=2$ yields the forward Laplacian).
      \Cref{sec:background} introduces the notation.}\label{fig:visual-abstract}
  \end{minipage}
  \hfill
  \begin{minipage}[b]{0.57\linewidth}
    \centering
    \newcommand{\drawgridrectangle}[4]{%
  \begin{tikzpicture}[scale=#4]
    \pgfmathsetmacro{\ymax}{#1}
    \pgfmathsetmacro{\xmax}{#2}

    % Fill the rectangle
    \fill[#3] (0,0) rectangle (\xmax,\ymax);

    % Draw the border
    \draw[white, line width=#4*3pt] (0,0) rectangle (\xmax,\ymax);


    % Draw vertical grid lines
    \pgfmathsetmacro{\xsteps}{#2}
    \foreach \x in {1,...,\xsteps} {
      \draw[white, line width=#4*3pt] (\x,0) -- (\x,\ymax);
    }

    % Draw horizontal grid lines
    \pgfmathsetmacro{\ysteps}{#1}
    \foreach \y in {1,...,\ysteps} {
      \draw[white, line width=#4*3pt] (0,\y) -- (\xmax,\y);
    }
  \end{tikzpicture}%
}

\newsavebox{\taylorStandard}
\savebox{\taylorStandard}{
  \begin{tikzpicture}
    \matrix [%
    matrix of nodes,%
    ampersand replacement=\&,% to use inside a savebox
    nodes={anchor=center, align=center},%
    column sep=4ex,%
    row sep=1ex,%
    ] (taylor)
    {
      \drawgridrectangle{1}{3}{blue!30}{0.33} \& \drawgridrectangle{1}{2}{blue!30}{0.33} \& \drawgridrectangle{1}{1}{blue!30}{0.33}
      \\[-1.5ex]
      $\vx_0$ \& $\vh_0$ \& $\vg_0$
      \\
      \drawgridrectangle{3}{3}{green!30}{0.33} \& \drawgridrectangle{3}{2}{green!30}{0.33} \& \drawgridrectangle{3}{3}{green!30}{0.33}
      \\[-1.5ex]
      $\{\vx_{1,d}\}$ \& $\{\vh_{1,d}\}$ \& $\{\vg_{1,d}\}$
      \\
      \drawgridrectangle{3}{3}{red!30}{0.33} \& \drawgridrectangle{3}{2}{red!30}{0.33} \& \drawgridrectangle{3}{1}{red!30}{0.33} \& \drawgridrectangle{1}{1}{red!60}{0.33}
      \\[-1.5ex]
      $\{\vx_{2,d}\}$ \& $\{\vh_{2,d}\}$ \& $\{\vg_{2,d}\}$ \& $\sum_d \vg_{2,d}$
      \\
    };

    % draw dependencies
    \pgfmathsetmacro{\K}{3}
    \pgfmathsetmacro{\L}{2}

    \foreach \l in {1,...,\L}{
      \pgfmathsetmacro{\lother}{int(\l+1)}
      \foreach \k in {1,...,\K} {
        \pgfmathsetmacro{\row}{int(2*\k-1)}
        \foreach \kother in {\k,...,\K} {
          \pgfmathsetmacro{\rowother}{int(2*\kother-1)}
          \draw[-Stealth, line width=1pt, white!50!black] (taylor-\row-\l.east) -- (taylor-\rowother-\lother.west);
        }
      }
    }
    \pgfmathsetmacro{\Lstart}{int(\L + 1)}
    \pgfmathsetmacro{\Lend}{int(\L + 2)}
    \pgfmathsetmacro{\rowfinal}{int(2*\K - 1)}
    \draw[-Stealth, line width=1pt, white!50!black] (taylor-\rowfinal-\Lstart.east) -- (taylor-\rowfinal-\Lend.west);
  \end{tikzpicture}
}

\newsavebox{\taylorCollapsed}
\savebox{\taylorCollapsed}{
  \begin{tikzpicture}
    \matrix [%
    matrix of nodes,%
    ampersand replacement=\&,% to use inside a savebox
    nodes={anchor=center, align=center},%
    column sep=4ex,%
    row sep=1ex,%
    ] (taylor)
    {
      \drawgridrectangle{1}{3}{blue!30}{0.33} \& \drawgridrectangle{1}{2}{blue!30}{0.33} \& \drawgridrectangle{1}{1}{blue!30}{0.33}
      \\[-1.5ex]
      $\vx_0$ \& $\vh_0$ \& $\vg_0$
      \\
      \drawgridrectangle{3}{3}{green!30}{0.33} \& \drawgridrectangle{3}{2}{green!30}{0.33} \& \drawgridrectangle{3}{3}{green!30}{0.33}
      \\[-1.5ex]
      $\{\vx_{1,d}\}$ \& $\{\vh_{1,d}\}$ \& $\{\vg_{1,d}\}$
      \\[2ex]
      \drawgridrectangle{1}{3}{red!60}{0.33} \& \drawgridrectangle{1}{2}{red!60}{0.33} \& \drawgridrectangle{1}{1}{red!60}{0.33}
      \\[-1.5ex]
      $\sum_d \vx_{2,d}$ \& $\sum_d \vh_{2,d}$ \& $\sum_d \vg_{2,d}$
      \\[1.1ex]
      \& \&
      \\
    };

    % draw dependencies
    \pgfmathsetmacro{\K}{3}
    \pgfmathsetmacro{\L}{2}

    \foreach \l in {1,...,\L}{
      \pgfmathsetmacro{\lother}{int(\l+1)}
      \foreach \k in {1,...,\K} {
        \pgfmathsetmacro{\row}{int(2*\k-1)}
        \foreach \kother in {\k,...,\K} {
          \pgfmathsetmacro{\rowother}{int(2*\kother-1)}
          \draw[-Stealth, line width=1pt, white!50!black] (taylor-\row-\l.east) -- (taylor-\rowother-\lother.west);
        }
      }
    }
  \end{tikzpicture}
}

\begin{figure*}[!t]
  \centering
  \resizebox{\linewidth}{!}{%
    \begin{tikzpicture}
      \node (standard) [fill=black!5!white, draw=black, rounded corners]{\usebox{\taylorStandard}};
      \node [anchor=north east, align=center, inner sep=10pt] at (standard.north east) {\textbf{Standard} \\ \textbf{Taylor mode}};
      \node (collapsed) [fill=black!5!white, draw=black, rounded corners, anchor=north west, xshift=5pt] at (standard.north east) {\usebox{\taylorCollapsed}};
      \node [anchor=south, align=center, inner sep=3pt] at (collapsed.south) {\textbf{Collapsed Taylor mode (ours)}};
    \end{tikzpicture}
  }
  \caption{\textbf{Visual comparison of standard Taylor mode and our proposed collapsed Taylor mode.}}\label{fig:comparison-standard-vs-collapsed}
\end{figure*}

\begin{figure*}
  \centering
  \newsavebox{\taylorStandardNew}
  \savebox{\taylorStandardNew}{
    \begin{tikzpicture}
      \matrix [%
      matrix of nodes,%
      ampersand replacement=\&,% to use inside a savebox
      nodes={anchor=center, align=center},%
      column sep=5ex,%
      row sep=1ex,%
      ] (taylor)
      {
        \drawgridrectangle{1}{5}{gray!25!white}{0.33}
        \& \drawgridrectangle{1}{3}{gray!25!white}{0.33}
        \& \drawgridrectangle{1}{1}{gray!25!white}{0.33}
        \\[-1.5ex]
        $\vx_0$ \& $\vh_0$ \& $\vg_0$
        \\
        \drawgridrectangle{4}{5}{gray!50!white}{0.33}
        \& \drawgridrectangle{4}{3}{gray!50!white}{0.33}
        \& \drawgridrectangle{4}{1}{gray!50!white}{0.33}
        \\[-1.5ex]
        $\{\vx_{1,d}\}$ \& $\{\vh_{1,d}\}$ \& $\{\vg_{1,d}\}$
        \\[0.5ex]
        \drawgridrectangle{1}{5}{white}{0.33}
        \& \drawgridrectangle{1}{3}{white}{0.33}
        \& \drawgridrectangle{1}{1}{white}{0.33}
        \\[0.5ex]
        \\
        \drawgridrectangle{4}{5}{gray}{0.33}
        \& \drawgridrectangle{4}{3}{gray}{0.33}
        \& \drawgridrectangle{4}{1}{gray}{0.33}
        \\[-1.5ex]
        $\{\vx_{K-1,d}\}$ \& $\{\vh_{K-1,d}\}$ \& $\{\vg_{K-1,d}\}$
        \\[0.5ex]
        \drawgridrectangle{4}{5}{red!50}{0.33}
        \&
        \drawgridrectangle{4}{3}{red!50}{0.33}
        \&
        \drawgridrectangle{4}{1}{red!50}{0.33}
        \\[-1.5ex]
        $\{\vx_{K,d}\}$ \& $\{\vh_{K,d}\}$ \& $\{\vg_{K,d}\}$
        \\[-1.5ex]
        \textcolor{purple!50!red}{$\sum_d \vx_{K,d}$}
        \& \textcolor{purple!50!red}{$\sum_d \vh_{K,d}$}
        \& \textcolor{purple!50!red}{$\sum_d \vg_{K,d}$}
        \\
      };

      \node[xshift=-1pt, yshift=-14pt] at (taylor-9-1) {\drawgridrectangle{1}{5}{purple!50!red}{0.33}};
      \node[xshift=-1pt, yshift=-14pt] at (taylor-9-2) {\drawgridrectangle{1}{3}{purple!50!red}{0.33}};
      \node[xshift=-1pt, yshift=-14pt] at (taylor-9-3) {\drawgridrectangle{1}{1}{purple!50!red}{0.33}};

      \node at (taylor-5-1) {\vdots};
      \node at (taylor-5-2) {\vdots};
      \node at (taylor-5-3) {\vdots};

      % draw dependencies
      \pgfmathsetmacro{\K}{5}
      \pgfmathsetmacro{\L}{2}

      \foreach \l in {1,...,\L}{
        \pgfmathsetmacro{\lother}{int(\l+1)}
        \foreach \k in {1,...,\K} {
          \pgfmathsetmacro{\row}{int(2*\k-1)}
          \foreach \kother in {\k,...,\K} {
            \pgfmathsetmacro{\rowother}{int(2*\kother-1)}
            \draw[-Stealth, line width=1pt, gray] (taylor-\row-\l.east) -- (taylor-\rowother-\lother.west);
          }
        }
      }

      \coordinate (arrowStart) at ($(taylor-1-1.north)+(0,3.5ex)$);
      \coordinate (arrowEnd) at ($(taylor-1-3.north east)+(0,3.5ex)$);
      \draw[-Stealth, line width=2pt, black] (arrowStart) to node [midway, fill=white, align=center] {\textbf{Taylor forward} \\ \textbf{propagation}} (arrowEnd);

      \node [left=1.5ex of taylor-1-1] (zero) {0};
      \node [left=1.5ex of taylor-3-1] {1};
      \node [left=1.5ex of taylor-5-1] {$\vdots$};
      \node [left=1.5ex of taylor-7-1] {$K-1$};
      \node [left=1.5ex of taylor-9-1] {$K$};

      \node [align=center] (coefficientLabel) at ($(zero)+(0, 5.5ex)$) {\textbf{Derivative}\\\textbf{degree}};

      \draw[rounded corners] (taylor-9-1.north west) rectangle (taylor.south east);
    \end{tikzpicture}
  }

  \begin{tikzpicture}
    \node {\usebox{\taylorStandardNew}};
  \end{tikzpicture}
\end{figure*}

%%% Local Variables:
%%% mode: LaTeX
%%% TeX-master: "../main"
%%% End:

  \end{minipage}
\end{figure*}

\begin{enumerate}[leftmargin=0.5cm]
\item \textbf{We propose optimizing standard Taylor mode by collapsing the highest Taylor coefficients} by directly propagating their sum, rather than propagating then summing (\cref{fig:visual-abstract}). This recovers the forward Laplacian and is applicable to randomized Taylor mode. Furthermore, relying on a result of Griewank et al.\, \cite{griewank_evaluating_1999}, we show how to transform general PDE operators into a form amenable to our setting.

%and, relying on [...] can be generalized to 

  %Collapsing standard Taylor mode for the Laplacian yields the forward Laplacian \cite{li2023forward}, but we show that this optimization is applicable to many other differential operators, and stochastic Taylor mode \cite{shi2024stochastic}.
  \todo{Felix: Phrase this so that it becomes clear that this is a theoretical contribution.}

\item \textbf{We show how to collapse standard Taylor mode by simple graph rewrites based on linearity.}
  This leads to a clean separation of concepts:
  Users can build their computational graph using standard Taylor mode, then use graph rewrites to collapse it.
  Due to the simple nature of our proposed rewrites, this feature could easily be absorbed into the just-in-time (JIT) compilation of ML frameworks, without introducing a new interface.

\item \textbf{We empirically demonstrate the performance improvements of collapsed Taylor mode.} We introduce a Taylor mode library for PyTorch \cite{paszke2019pytorch} that realizes the graph simplifications with \texttt{torch.fx} \cite{reed2022torch}. Furthemore, we show that collapsing Taylor mode yields the theoretically expected improvements in run time and memory consumption. Due to the simple nature of our proposed rewrites, this feature could easily be integrated into the just-in-time (JIT) compilation of ML frameworks, without introducing a new interface.

\end{enumerate}

Our work takes an important step towards the broader adoption of Taylor mode as viable alternative to backpropagation for computing PDE operators, while being as easy to use. 

%%% Local Variables:
%%% mode: LaTeX
%%% TeX-master: "../main"
%%% End:


\section{Background on AD for Higher-Order Derivatives}\label{sec:background}
Taylor-mode AD (or, simply, Taylor mode) computes higher-order derivatives---as needed, \eg, for PDE operators---through propagation of Taylor coefficients according to the chain rule.

\begin{figure}[!t]
  \centering
  \begin{minipage}[t]{0.7\linewidth}
    \centering
   \begin{tikzpicture}
    \tikzset{box/.style={rectangle, rounded corners, draw=black, inner sep=3pt}}
    \node[align=center, box] (topleft) {Extent input to \\ smooth path
      $\vx(t)$};

    \node[align=center, right=2.5cm of topleft, box] (topright) {Path in output \\ space $f(\vx(t))$};
    \draw [-Latex] (topleft.east) to node [midway, above] {$f$} (topright.west);

    \node[align=center, below=1cm of topright, box, draw=tab-orange] (bottomright) {%Taylor polynomial of degree $K$
      % \\
      $K$-jet \; $\sum_{k=0}^K \frac{t^k}{k!} \vf_k$
      % \\
      % {\color{maincolor}$(\vf_0, \dots, \vf_K)$}
    };
    \draw [-Latex] (topright.south) to node [midway, right] {$J^K$}
    (bottomright.north);

    \node[align=center, below=1cm of topleft, box, draw=tab-orange] (bottomleft) {{%Taylor polynomial of degree $K$
        % \\
        $K$-jet \; $\sum_{k=0}^K \frac{t^k}{k!} \vx_k$
        % \\
        % {\color{maincolor} $(\vx_0, \dots, \vx_K)$}
      }};
    \draw [-Latex] (topleft.south) to node [midway, left] {$J^K$} (bottomleft.north);

    \draw [-Latex, \colorTM, align=center] (bottomleft.east) to node [midway, above, tab-orange] {\color{\colorTM}Taylor-mode\\ AD} (bottomright.west);
  \end{tikzpicture}
  \end{minipage}
  \hfill
  \begin{minipage}[b]{0.29\linewidth}
    \caption{The Taylor-mode AD: From the $K$-jet of the input to the $K$-jet of the output.}
  \label{fig:utp}
  \end{minipage}
\end{figure}

%%% Local Variables:
%%% mode: LaTeX
%%% TeX-master: "../main"
%%% End:


\paragraph{Scalar case.}
To illustrate Taylor-mode, consider the scalar function $f: \sR \to \sR$ and extend the input variable $x$ to a path $x(t)$ whose form is a univariate Taylor polynomial of degree $K$, $\smash{x(t) = \sum_{k=0}^K \frac{t^k}{k!} x_k}$ with $x_k$ the $k$-th Taylor coefficient.
If $f$ is smooth enough, we can evaluate Taylor coefficients of the transformed path $\smash{f(x(t)) = \sum_{k=0}^K \frac{t^k}{k!} f_k}$ with $\smash{f_k \coloneqq \frac{\mathrm{d}^k}{\mathrm{d}t^k} f(x(t)) |_{t=0}}$.
The chain rule provides the coefficients' propagation rules.
\Eg, for degree $K=3$ we get
\begin{align}
  \label{eq:taylor-mode-scalar}
  \begin{matrix*}[l]
    f_0 = f(x_0)\,,
    \\[0.75ex]
    f_1 = \partial f(x_0) x_1\,,
  \end{matrix*}
  \qquad
  \begin{matrix*}[l]
    f_2 = \partial^2 f(x_0)x_1^2 + \partial f(x_0) x_2\,,
    \\[0.75ex]
    f_3
    = \partial^3 f(x_0)x_1^3 + 3 \partial^2 f(x_0) x_1 x_2 + \partial f(x_0) x_3\,.
  \end{matrix*}
\end{align}
\citet{faa1857note} provided the general formula for $f_k$, and \citet{fraenkel1978formulae} extended it to the multivariate case \cite[see also][]{arbogast1800calcul,hardy2006combinatorics}.
It serves as foundation for the Taylor mode to compute higher-order derivatives, e.g.,  \citep[\S13]{griewank2008evaluating}:
setting $x_1 = 1, x_2 = x_3 = 0$ yields $f_1 = \partial f(x_0), f_2 = \partial^2 f(x_0), f_3 = \partial^3 f(x_0)$.
We call the univariate Taylor polynomial of a function $x(t)$ of degree $K$, represented by the coefficients $(x_0, \dots, x_K)$, the \emph{$K$-jet of $x$}, following the terminology of JAX's Taylor mode.

\paragraph{Notation for multivariate case.}
We consider the general case of computing higher-order derivatives, \eg PDE operators, of a vector-to-vector function $\vf: \sR^D \to \sR^C$.
This requires additional notation to generalize \cref{eq:taylor-mode-scalar}.
Given $K$ vectors $\vv_1, \dots, \vv_K \in \sR^D$, we write their tensor product as
\begin{equation*}
  \otimes_{k=1}^K \vv_k = \vv_1 \otimes \ldots \otimes \vv_K
  \in ( \sR^D )^{\otimes K}
  \quad
  \text{with entries}
  \quad
  \left[\otimes_{k=1}^K \vv_k\right]_{d_1, \dots, d_K}
  = [\vv_1]_{d_1} \cdots [\vv_K]_{d_K}
\end{equation*}
for $d_1, \dots, d_K \in \{1, \dots, D\}$, and compactly write $\vv^{\otimes K} = \otimes_{k=1}^K \vv$.
We define the inner product of two tensors $\smash{\tA, \tB \in (\sR^{D})^{\otimes K}}$ as the Euclidean inner product of their flattened versions
\begin{align}\label{eq:derivative-tensor-scalar-product}
  \textstyle % Comment out this line if we have enough space
  \left\langle
  \tA, \tB
  \right\rangle
  \coloneqq
  \sum_{d_1}
  \sum_{d_2}
  \dots
  \sum_{d_K}
  \tA_{d_1, d_2, \dots, d_K}
  \tB_{d_1, d_2, \dots, d_K} \in \sR\,.
\end{align}
We allow broadcasting in \cref{eq:derivative-tensor-scalar-product}: if one tensor has more dimensions but matching trailing dimensions, we take the inner product for each component of the leading dimensions.
This allows to express contractions with derivative tensors of vector-valued functions, \eg,
contracting the $k$-th derivative tensor $\partial^k \vf(\vx_0) \in \sR^C \otimes (\sR^D)^k$, such that $\langle \tA, \partial^k \vf(\vx_0) \rangle \in \sR^C$. 

\paragraph{Multivariate case \& composition.}
Evaluating the $K$-jet of $\vf$ at an argument $\vx_0 \in \sR^D$ starts with the extension of $\vx_0$ to a smooth path $\vx: \sR \to \sR^D$ with $\vx(0) = \vx_0$.
The $K$-jet of $\vf$ is defined as \todo{Felix@Tim: Wouldn't it be better to define the $K$-jet as mapping from $\times_{k=0}^K \sR^D \to \times_{k=0}^K \sR^C$ ?
  This would be more aligned with the propagation in \cref{eq:taylor-mode-composition}
  T: We could, this is just notation, I thought it would be better because this is the usual motivation of Taylor-mode.
  AW: I think to be precise it should be a mapping from $\times_{k=0}^K \sR^D$ to the polynomials of degree $\le K$ in the sense of Fig. 3 ;-), but that might be too complicated. For me the current statement is fine, since $(J^K \vf)$ indeed maps from $\R \to \R^C$ as stated
  T: The mapping is actually precise in the sense that you could interpret everything as the vectors of the coefficients and use this as the jet definition instead of the sum. Then, we would have indeed the mapping between $\times_{k=0}^K \sR^D$ and $\times_{k=0}^K \sR^C$. Ive seen this in other ML papers about taylor mode. But as I said, for me it doesnt matter :D}
\begin{align*}
  \textstyle % Comment out this line if we have enough space
  J^K \vf : \sR \to \sR^C\,,
  \quad (J^K \vf)(t) := \sum_{k=0}^K \frac{t^k}{k!} \vf_k
  \quad \text{with} \quad
  \vf_k := \left. \frac{\mathrm{d}^k}{\mathrm{d}t^k} \vf(\vx(t)) \right|_{t=0}
\end{align*}
and requires the $K$-jet of $\vx$, $(J^K \vx)(t) := \sum_{k=0}^K \frac{t^k}{k!} \vx_k$ (illustrated in \cref{fig:utp}). \todo{mention the jet form (x1, ... xk)}

As is common for AD, the computation of the $\vf_k$ known as Taylor mode relies on composing $\vf$ of elemental functions with known derivatives, and the chain rule.
In the simplest case, let $\vf$ be given by $\vf = \vg \circ \vh: \sR^D \to \sR^I \to \sR^C$ for two elemental functions $\vg$ and $\vh$. Given the input $K$-jet for $\vx$, the coefficients $\vh_k = \left. \frac{\mathrm{d}^k}{\mathrm{d}t^k}\vh(\vx(t)) \right|_{t=0}$ are computed using the generalized Faà di Bruno formula \eqref{eq:faa-di-bruno}:
\begin{align}
  \label{eq:faa-di-bruno}
  % \textstyle % Comment out this line if we have enough space
  \vh_k
  =
  \sum_{\sigma \in \partitioning(k)}
  \nu(\sigma)
  \left<
  \partial^{|\sigma|} \vh,
  \tensorprod{s \in \sigma} \vx_s
  \right>
  \quad
  \text{with}
  \quad
  \nu(\sigma)
  =
  \frac{k!}{
    \left(
      \prod_{s \in \sigma
      }
      n_s!
    \right)
    \left(
      \prod_{s \in \sigma}
      s!
    \right)
  }\,
\end{align}
Here, $\partitioning(k)$ is the integer partitioning of $k$ (a set of sets), $\nu$ is a multiplicity function, and $n_s$ counts occurrences of $s$ in an integer partitioning $\sigma$ (\eg $n_1(\{1,1,3\}) = 2$ and $n_3 = 1$).
The $\vh_k$'s are propagated through $\vg$ resulting in the $K$-jet for $\vf$.
In summary, this yields the propagation scheme (where $\vx_k \in \sR^D$, $\vh_k \in \sR^I$, $\vf_k \in \sR^C$)
\begin{align}\label{eq:taylor-mode-composition}
  \begin{split}
    &\begin{pmatrix*}
      \vx_0
      \\
      \vx_1
      \\
      \vx_2
      \\
      \vdots
      \\
      \vx_K
    \end{pmatrix*}
      \!\overset{\text{(\ref{eq:faa-di-bruno})}}{\to}\!
      \begin{pmatrix*}[l]
        \vh_0 \!=\!  \vh(\vx_0)
        \\
        \vh_1 \!=\!  \left<
        \partial \vh(\vx_0),
        \vx_1
        \right>
        \\
        \vh_2 \!=\! \left<
        \partial^2 \vh(\vx_0),
        \vx_1 \otimes \vx_1
        \right>
        \!+\!
        \left <
        \partial \vh(\vx_0),
        \vx_2
        \right>
        \\
        \vdots
        \\
        \vh_K \!=\!
        \displaystyle \sum_{
        \mathclap{
        \sigma \in \partitioning(K)
        }
        }
        \nu(\sigma) \left<
        \partial^{|\sigma|} \vh(\vx_0),
        \tensorprod{s \in \sigma} \vx_s
        \right>\!\!\!
      \end{pmatrix*}
    \\
    &\!\!\overset{\text{(\ref{eq:faa-di-bruno})}}{\to}\!
      \left(\!\!\!
      \begin{array}{l}
        \vg_0 \!=\!  \vg(\vh_0)
        \\
        \vg_1 \!=\! \left<
        \partial \vg(\vh_0),
        \vh_1
        \right>
        \\
        \vg_2 \!=\! \left<
        \partial^2 \vg(\vh_0),
        \vh_1 \!\otimes\! \vh_1\right>
        \!+\!
        \left< \partial \vg(\vh_0),
        \vh_2
        \right>
        \\
        \vdots
        \\
        \vg_K \!=\!
        \displaystyle\sum_{
        \mathclap{
        \sigma \in \partitioning(K)
        }
        }
        \nu(\sigma) \left<
        \partial^{|\sigma|} \vg(\vh_0),
        \tensorprod{s \in \sigma} \vh_s
        \right>
      \end{array}
      \!\!\!\!
      \right)
      \!\!=\!\!
      \left(\!\!\!
      \begin{array}{l}
        \vf_0 \!=\!  \vf(\vx_0)
        \\
        \vf_1 \!=\! \left<
        \partial \vf(\vx_0),
        \vx_1
        \right>
        \\
        \vf_2 \!=\! \left<
        \partial^2 \vf(\vx_0),
        \vx_1 \!\otimes\! \vx_1
        \right>
        \!+\!
        \left< \partial \vf(\vx_0),
        \vx_2
        \right>
        \\
        \vdots
        \\
        \vf_K \!=\!
        \displaystyle\sum_{
        \mathclap{
        \sigma \in \partitioning(K)
        }
        }
        \nu(\sigma) \left<
        \partial^{|\sigma|} \vf(\vx_0),
        \tensorprod{s \in \sigma} \vx_s
        \right>
      \end{array}
      \!\!\!\!
      \right)
  \end{split}
\end{align}
which describes the forward propagation of a single $K$-jet.
\citet[][\S13]{griewank2008evaluating} presents explicit formulas of common elemental functions that have quadratic complexity in $K$.
However, computing PDE operators requires propagating \emph{multiple} $K$-jets in parallel accumulating their results.
We propose to pull this accumulation inside Taylor mode's propagation scheme, thereby collapsing it.

%%% Local Variables:
%%% mode: LaTeX
%%% TeX-master: "../main"
%%% End:


\section{The Collapsed Taylor-Mode}\label{sec:methodology}
\todo{improve intro text}

We introduce a unified framework for handling differential operators efficiently. In the next subsection, a linearity trick is applied to sums of Jets. Those jets already unify the settings of Forward-Laplace \cite{li2023forward}, Randomized Taylor-Mode ~\cite{shi2024stochastic, hu2023hutchinson} and the generalized trace operator. Afterwards, we show that all differential operators, including those not directly representable through jets because of mixed-partial derivatives, can be computed via a suitable combination of Jets. This is based on a classical AD result of Griewank et al. \cite{griewank_evaluating_1999}. The result will also include the approach of \cite{shi2024stochastic} for computing arbitrary differential operators. However, our approach only needs Jets of the highest derivative degree. In addition, it has to be emphasized, our proposed method is purely automatic, meaning it can be implemented as part of a JIT-toolchain.


\subsection{Linearity for Taylor-mode AD}
The aim is to compute the $K$-Jet 
\begin{align}\label{eq:sum-k-jet}
  \sum_{n=1}^N\left< 
  \partial^{K} f(\vx_0),
  \otimes_{k=1}^K \vv_n
  \right>.
\end{align}
Inserting this into the propagation scheme \cref{eq:taylor-mode-composition} gives the Taylor-mode presented in \cref{eq:sum-taylor-mode-naive}, which would propagate $1 + KN$ vectors. 

Our efficient approach is based on the observation that there is a special element in $\partitioning(K)$, namely the trivial partition $\sigma_{t} = \{K\}$. This partition corresponds to the term 
\begin{equation}
    \nu(\sigma_{t}) \left<
    \partial \vg(\vh),
    \vh_{K, n}
    \right>, 
\end{equation}
of the Faà di Bruno formula. Using the definitions of \cite{hardy2006combinatorics} it is $\nu(\sigma_{t}) = 1$. 
Therefore,
\begin{align}
\label{eq:faa-di-bruno-expanded}
     \vf_{K, n} 
     = 
     \vg_{K,n} 
     &= 
     \displaystyle\sum_{
      \sigma \in \partitioning(K) \setminus \{\sigma_t\}
     } 
     \nu(\sigma) \left<
     \partial^{|\sigma|} \vg(\vh_0),
     \tensorprod{s \in \sigma} \vh_{K, n}
     \right>
     \\
     &+
     \left<
     \partial \vg(\vh),
     \vh_{K, n}
     \right>.
\end{align}


Looking at \cref{eq:faa-di-bruno-expanded} together with \cref{eq:sum-taylor-mode-naive}, it can be seen that it is not necessary to propagate the $K$-th Taylor coefficients $\vx_{K,n}, \{\vh_{K,n}\}_n, \{\vg_{K, n}\}_n$ to compute the $K$-Jet \cref{eq:sum-k-jet}. Instead, it is sufficient to propagate the sums of these coefficients directly by using linearity of $\left<\partial \vh, \bullet \right>$ and $\left< \partial g, \bullet \right>$. \Cref{eq:sum-taylor-mode-efficient} shows this scheme (changes highlighted in \textcolor{maincolor}{color}. This scheme \cref{eq:sum-taylor-mode-efficient}, only propagates $1 + (K-1)N + 1$ vectors. 

To emphasize that the saving of $N-1$ vectors in the propagation scheme is indeed significant for common applications, important operators are discussed below. 


\subsection{Examples}
A highlight of our general approach \cref{eq:sum-k-jet} is that it naturally includes common important differential operators or their approximation.

\paragraph{Laplace Operator} The Laplace Operator occurs in a variety of differential equations and is thus a worthwhile goal for PINNs.  \todo{need citations here}
For a function $f: \sR^D \to \sR$ the Laplacian is given as
\begin{align}\label{eq:laplacian}
  \Delta f(\vx_0)
  :=
  \sum_{d=1}^D \frac{\partial^2 f(\vx_0)}{\partial x_d^2}
  =
  \sum_{d=1}^D \left< 
  \partial^2 f(\vx_0),
  \ve_d \otimes \ve_d
  \right>
\end{align}
Assume that $f$ is again decomposed into $g \circ \vh$, where $\vh$ could have potentially multiple output dimensions is it is for a hidden layer. Selecting $K = 2$, $N=D$ and $\vv_n = \ve_n$ lifts \cref{eq:laplacian} into the setting of \cref{eq:sum-k-jet}.
Naively applying the scheme \cref{eq:sum-taylor-mode-efficient} would lead to $1 + 2D$ (see \cref{eq:laplacian-naive}). If we instead apply our proposed efficient scheme \cref{eq:sum-taylor-mode-efficient}, only $1 + D + 1$ vectors are propagated. The scheme is presented in \cref{eq:laplacian-efficient} (changes highlighted in \textcolor{maincolor}{color})

Note that this scheme is aligned with the findings of \cite{li2023forward}, but here it is embedded into a general framework.



\paragraph{Weighted sums of second-order derivatives.} \todo{We can make the connection \cite{hu2023hutchinson} equation (5) clear and discuss the case of $\mC$ being a function of the input} Another important differential operator is represented by a symmetric positive semi-definite (PSD) matrix $\boldsymbol{C} \in \left(\mathbb{R}^D\right)^{\otimes 2}$:
\begin{align}\label{eq:weighted-laplacian}
  \sum_{i,j} \boldsymbol{C}_{i,j} \frac{\partial^2 \vf(\vx_0)}{\partial x_i \partial x_j}.
\end{align}
Those operators occur, for example, as 
\begin{equation*}
    \Tr (\msigma \msigma^\top \partial^2 \vf(\vx_0)
\end{equation*}
in Fokker-Planck and Hamiltonian-Jacobi Bellmann equations. \todo{need citations} Since $\boldsymbol{C}$ is PSD it can be expressed as $\mC = \sum_{n=1}^{\rank(\mC)} \vc_n \vc_n^\top$ for suitable $\vc_n \in \sR^D$. Therefore, it fits the setting of \cref{eq:sum-k-jet} with $K = 2, N = \rank(\mC)$ and $\vv_n = \vc_n$, and can be compute using \cref{eq:sum-taylor-mode-efficient}. This requires the propagation of $1 + \rank(\mC) + 1$ vectors instead of $1 + 2\rank(\mC)$.

\todo{mention randomized laplacians!}


\paragraph{Traces of higher-order derivative tensors}
\todo{cite papers that consider this} The higher-order Laplacian, an often theoretically considered example, is also easily expressible in our framework:
\begin{align}\label{eq:trace-differential-operator}
    \Tr( \partial^K f(\vx) )
    \coloneqq
    \sum_{d=1}^D
    \frac{\partial^K f(\vx_0)}{\partial \evx_d^K}
    =
    \sum_{d=1}^D
    \left< 
    \partial^K f(\vx_0),
    \otimes_{i=1}^K \ve_d
    \right>,
\end{align}
with $N = D$ and $\vv_n = \ve_n$.

\subsection{Operators with mixed-partial Derivative}
Despite being already powerful enough to unify various approaches for the computation of important differential operators, our framework cannot handle operators that include arbitrary mixed-partial derivatives.
In the following, we explain how to transform a differential operator containing arbitrary mixed-partial derivatives into a combination of Jets using a well-known result from the AD community.


Consider the operator
\begin{equation}
   \mA = \sum_{n=1}^{N} \frac{\partial^{\vk_n}}{\partial \vx^{\vk_n}},
\end{equation}
where $\vk_n = (k_1, \dots, k_D) \in \mathbb{N}^D$ is a multi-index. Further information about the multi-index notation can be found in \cref{sec:appendix_ttc}.

A technique to reconstruct mixed-partials of arbitrary order from jets has already been proposed in 1999 by Griewank et al. in \cite{griewank_evaluating_1999}. The authors summarized their method in the following formula
\begin{equation}
\label{eq:ttc_general}
    \frac{\partial^{\vk}}{\partial \vz^\vk} \vf(\vx + \boldsymbol{S} \vz)\Big|_{\vz = \boldsymbol{0}} 
    = 
    \sum_{\underset{\vl \in \mathbb{N}^p}{|\vl| = |\vk|}}
    \gamma_{\vk \vl}
    \frac{1}{\vk!}
    \left<
    \partial^{K}\vf,
    \otimes_{k=1}^K
    \boldsymbol{S}\vl
    \right>.
\end{equation}
\cref{sec:appendix_ttc} provides supplementary details to the formula. From \cref{eq:ttc_general} we conclude
\begin{align} \label{eq:ttc_general_operator}
    \mA \vf(\vx_0) =  \sum_{n=1}^N \sum_{\underset{\vl \in \mathbb{N}^p}{|\vl| = |\vk_n|}}
    \gamma_{\vk_n \vl}
    \frac{1}{\vk_n!}
    \left<
    \partial^{|\vk_n|}\vf,
    \otimes_{l=1}^{|\vk_n|}
    \boldsymbol{S}\vl
    \right>.
\end{align}
Therefore, a general differential operator with arbitrary mixed-partial derivatives $\mA\vf(\vx_0)$ is expressible in our setting \cref{eq:sum-k-jet}. Thus, for every $\mA$ we can apply our efficient propagation scheme and have to propagate the different $|\vk_n|$-Jets with directions $\mS \vl$ \todo{maybe be a bit more precise here} instead of depending on expensive nested derivative calls. 

As the inventors of the formula already recognized, the coefficients $\gamma_{\vk\vl}$ only depend on the problem structure and are independent of the function $\vf$. As we will show, this allows to exploit symmetry that is often inherent to differential operators.
\todo{are there symmetries directly exploitable?}

\subsection{Examples}

\paragraph{Biharmonic Operator}
The Biharmonic Operator is a fourth-order differential operator with mixed-partial derivatives that is interesting for some practitioners. \todo{cite}
It is defined as
\begin{align}
\label{eq:biharm}
    \Delta^2 \vf(\vx)
    \coloneqq
    \sum_{i=1}^D \sum_{j=1}^D
    \frac{\partial^4 \vf(\vx)}{\partial \evx_i^2 \partial \evx_j^2}
    =
    \sum_{i=1}^D \sum_{j=1}^D
    \left<
    \partial^4 \vf(\vx),
    \ve_i \otimes \ve_i \otimes \ve_j \otimes \ve_j
    \right>.
\end{align}

The classical way to compute such derivatives is to nest all four derivatives, which grows exponentially in run-time and memory with the derivative order. A comparison with this approach is given in \cref{sec:experiments}.

Another way is naturally given by considering the Biharmonic Operator as the second-order Laplacian. Then, \cref{eq:laplacian-naive} could be applied twice. Instead of propagating fourth-order derivative tensors, holding $\left( \begin{matrix} D + 3 \\ 4 \end{matrix} \right) \in O(D^4)$ distinct elements, the twice-applied Laplacian would have to push $(2 + D) + D * (2 + D) + D * (2 + D) = 2D^2 + 5D + 2$ vectors.
Using \cref{eq:laplacian-efficient} we reduce this further to $(2 + D) + D * (2 + D) + 1 * (2 + D) = D^2 + 4D + 5$ vectors. 

A different approach was proposed in \cite{shi2024stochastic}, which relies on the hand-selection of $6$-Jets to extract the searched derivatives. See the appendix for more details: \cref{sec:biharm_felix_approach}. 

\todo{compare with \cite{hu2023hutchinson}}

Our approach first applies \cref{eq:ttc_general} to the Biharmonic Operator \cref{eq:biharm}. To this end, parameters $\vk, p$, auxiliary variables $\vz$, and matrices $\boldsymbol{S}_{ij}$ have to be selected for all occurring mixed-partial derivatives. The parameters are grouped into the case where all four derivatives directions coincide, i.e, $i = j$ and the case where $i \neq j$. The selected parameters are depicted in \cref{tab:params_ttc_biharm}. 

\begin{table}[]
    \centering
    \begin{tabular}{|c|c|c|c|c|}
\toprule
             & $p$ & $\vk$    & $\vz$        & $\boldsymbol{S}_{ij}$  \\
\midrule
     $i = j$ & $1$ & $(4)$    & $(z_1)$      & $(\ve_i)$                \\
\midrule
  $i \neq j$ & $2$ & $(2, 2)$ & $(z_1, z_2)$ & $\left( \ve_i \; \ve_j \right)$ \\
\bottomrule
    \end{tabular}
    \caption{Parameters to apply \cref{eq:ttc_general} to the Biharmonic Operator \cref{eq:biharm}.}
    \label{tab:params_ttc_biharm}
\end{table}


Having defined the parameters, \cref{eq:ttc_general} is utilized to map the Biharmonic Operator \cref{eq:biharm} of a function $\vf$ to a collection of fourth-order jets: \todo{check coefficients} 
\begin{align}
    \Delta^2 \vf(\vx)
    &=
    \sum_{i=1}^D
    \gamma_{(4)(4)}
    \frac{1}{4!}
    \left<
    \partial^{4}\vf(\vx_0),
    \otimes_{i=1}^4 4 \ve_i
    \right>
    \\
    &+
    \sum_{i=1}^D \sum_{\underset{j \neq i}{j=1}}^D
    \sum_{\underset{\vl \in \mathbb{N}^2}{|\vl| = 4}}
    \gamma_{(2, 2) \vl}
    \frac{1}{(2, 2)!}
    \left<
    \partial^{4} \vf(\vx_0),
    \otimes_{i=1}^4
    \boldsymbol{S}_{ij} \vl
    \right>.
\end{align}
As mentioned before, the coefficients $\gamma_{\vk \vl}$ capture the symmetric structure of the differential operator. A close look into the definition \cref{eq:ttc_coeff} shows the equality of $\vl = (4,0)$ with $\vl=(0, 4)$ and $\vl = (3, 1)$ with $\vl = (3, 1)$. Exploiting those symmetries the formula boils down to \todo{think we need 1/4!}
\begin{align}
   \Delta^2 \vf(\vx)
   &=
    \left(
   \gamma_{(4)(4)} + 2 (D - 1) \gamma_{(2,2)(4, 0)}
   \right)
   \sum_{i=1}^D
    \left< \partial^4 \vf(\vx_0),
    \otimes_{i=1}^4
    4\ve_i
    \right>
    \\
    &+
    2 \gamma_{(2, 2)(3, 1)} 
    \sum_{i=1}^D 
    \sum_{\underset{j \neq i}{j=1}}^D
    \left< \partial^4 \vf(\vx_0),
    \otimes_{i=1}^4 3 \ve_i+ \ve_j
    \right>
    \\
    &+ 
    \gamma_{(2, 2)(2, 2)}
    \sum_{i=1}^D 
    \sum_{\underset{j \neq i}{j=1}}^D
    \left< \partial^4 \vf(\vx_0),
    \otimes_{i=1}^4
    2 \ve_i + 2 \ve_j
    \right>.
\end{align}
\Cref{tab:ttc_biharm_coeffs} lists the required coefficients.

\begin{table}[]
    \centering
    \begin{tabular}{|c|c|c|c|}
        \toprule
        $\gamma_{(4)(4)}$ & $\gamma_{(2, 2) (4, 0)}$ & $\gamma_{(2, 2) (3, 1)}$ & $\gamma_{(2, 2) (2, 2)}$ \\
        \midrule
        $\frac{3}{32}$ & $\frac{13}{192}$ &  -$\frac{1}{3}$ & $\frac{5}{8}$  \\
        \bottomrule
    \end{tabular}
    \caption{Coefficients to compute the Biharmonic Operator by \Crefrange{eq:ttc_biharm_full1}{eq:ttc_biharm_full3}.}
    \label{tab:ttc_biharm_coeffs}
\end{table}

The last term can be further simplified by leveraging the symmetry of the direction $2 \ve_i + 2 \ve_j$. This simplification leads to the following efficient propagation scheme for \cref{eq:biharm}:
\begin{align}
   \Delta^2 \vf(\vx)
   &= \label{eq:ttc_biharm_full1}
    \left(
   \gamma_{(4)(4)} + 2 (D - 1) \gamma_{(2,2)(4, 0)}
   \right)
   \sum_{i=1}^D
    \left<\partial^4 \vf(\vx_0),
    \otimes_{i=1}^4 4\ve_i
    \right>
    \\
    &+ \label{eq:ttc_biharm_full2}
    2 \gamma_{(2, 2)(3, 1)} 
    \sum_{i=1}^D 
    \sum_{\underset{j \neq i}{j=1}}^D
    \left< \partial^4 \vf(\vx_0), 
    \otimes_{i=1}^4 3 \ve_i + \ve_j
    \right>
    \\
    &+ \label{eq:ttc_biharm_full3}
    2 \gamma_{(2, 2)(2, 2)}
    \sum_{i=1}^{D-1} 
    \sum_{j=i+1}^D
    \left< \partial^4 \vf(\vx_0),
    \otimes_{i=1}^4 2 \ve_i + 2 \ve_j
    \right>.
\end{align} 

The \Crefrange{eq:ttc_biharm_full1}{eq:ttc_biharm_full3} are computed through $D + D(D-1) + \frac{D(D-1)}{2}$ fourth-order jets. These fourth-order jets exhibit the structure required to apply our proposed graph simplifications, which makes the computations even more efficient. \todo{align with higher order laplacian} Thus, the summation can be pulled inside the highest coefficients in the propagation scheme, like for the Laplacian. From this we conclude that our propagation scheme for the Biharmonic Operator has the effort of propagating $1 + 3D + 1 + 1 + 3 \cdot D(D-1) + 1 + 1 + 3 \frac{D(D-1)}{2} + 1 = 9\frac{D^2}{2} - 3\frac{D}{2} + 6$ vectors. It should be noted, that all jets are independent and can be computed in parallel, which provides further benefits for the run-time complexity.

\todo{in comparison to laplace over laplace, we are worse in theory?}




\paragraph{[Not done] Stochastic Taylor Derivative Estimator}
\todo{consider this for mixed-partials, like estimator of biharmonic, maybe mention this also in weighted sum of derivatives}
The approach of \citep{shi2024stochastic} lacks the possibility to handle differential operators with arbitrary mixed-partials. With our framework however, it is easy to transform a differential operator with mixed-partials into a collection of Jets. Therefore, the randomization approach of their work is automatically applicable in our framework. 

The sampled directions to approximate a differential operator 
This lowers the computational cost when $D$ is prohibitively large for exact evaluation of differential operators like the Laplacian.
Take the weighted Laplacian from \Cref{eq:weighted-laplacian} and assume we have access to random vectors $\rvv$ with and $\E[\vv \vv^{\top}] = \mC$.
Then we can draw $S \ll D$ random vectors $\vv_1, \vv_2, \dots, \vv_S \overset{\text{i.i.d.}}{\sim} \rvv$ and compute
\begin{align}
  \begin{split}
    \E[\partial^2f[\rvv, \rvv]]
    &=
      \E \left[
      \rvv^{\top} \frac{\partial^2 f}{\partial \vx \partial \vx} \rvv
      \right]
      =
      % \E \left[
      %   \sum_{i,j} \ervv_i \frac{\partial^2 f}{\partial x_i \partial x_j} \ervv_j
      % \right]
      % =
      \sum_{i,j} \frac{\partial^2 f}{\partial x_i \partial x_j}
      \E \left[
      \ervv_i  \ervv_j
      \right]
      =
      \sum_{i,j} \frac{\partial^2 f}{\partial x_i \partial x_j} C_{i,j}
    \\
    &\approx
      \frac{1}{S}
      \sum_{s=1}^S
      \partial^2 f[\vv_s, \vv_s]\,.
  \end{split}
\end{align}
Again, we see that the second-order coefficients can be collapsed when setting $\vx_{1,d} = \vv_d, \vx_{2,d} = \vzero$ in \cref{eq:laplacian-naive,eq:laplacian-efficient}.


% \section{Kronecker-Factored Approximate Curvature for
% PINNs}\label{sec:kfac_pinns}
% \input{sections/contribution.tex}

\section{Experiments}\label{sec:experiments}
\paragraph{Design decisions and limitations.} Although JAX already offers an experimental Taylor mode implementation, we re-implemented Taylor mode in PyTorch, taking heavy inspiration from the JAX implementation for the interface.
We decided so because PyTorch's \texttt{fx} library provides a user-friendly official interface to capture and transform compute graphs of functions to apply our proposed collapsing.
While we believe that such graph transformations should in principle be feasible in JAX as well (and could \eg be integrated into its \texttt{jit} compiler), it seems that this can currently only be achieved with (potentially fragile) internal APIs and requires a deep understanding of its internal tracing mechanisms.
Main advantage seems to be that we can clearly disentangle building the graph from simplifying it using linearity, whereas in JAX we could write a custom interpreter that would have to re-implement the logic of summing the $K$-th component for abitrary $K$.

\paragraph{Usage.} Our implementation takes a PyTorch function (\eg a neural net) and first captures its compute graph using \texttt{torch.fx}'s symbolic tracing mechanism, then replaces each operation with its Taylor arithmetic.
This yields the compute graph of the function's $K$-jet.
Users can use this vanilla Taylor mode to define their differential operator's computation. 
The collapsing is done by again tracing said computation with \texttt{torch.fx} and rewriting the resulting graph, propagating the summation of highest coefficient up.
The resulting graph performs the same computation, but uses our collapsed Taylor mode.
See \Cref{fig:interface-overview} for a visual walk-through of the procedure.

TODO Limitations.

\paragraph{Experimental procedure.} We compare our collapsed Taylor mode with vanilla Taylor mode and nested first-order AD on the previously discussed differential operators in PyTorch on an NVIDIA RTX 6000 with .




\begin{figure*}[!t]
  \centering
  % From https://tex.stackexchange.com/a/7318
  \newcolumntype{C}{ >{\centering\arraybackslash} m{0.12\textwidth} }
  \newcolumntype{D}{ >{\centering\arraybackslash} m{0.27\textwidth} }
  \begin{tabular}{CDDD}
    & \textbf{Laplacian $(D=50)$}
    & \textbf{Weighted Laplacian $(D=50)$}
    & \textbf{Bi-harmonic $(D=5)$}
    \\
    \textbf{Exact}
    & \includegraphics{../jet/exp/exp01_benchmark_laplacian/figures/architecture_tanh_mlp_768_768_512_512_1_device_cuda_dim_50_name_laplacian_vary_batch_size.pdf}
    &  \includegraphics{../jet/exp/exp01_benchmark_laplacian/figures/architecture_tanh_mlp_768_768_512_512_1_device_cuda_dim_50_name_weighted_laplacian_vary_batch_size.pdf}
    & \includegraphics{../jet/exp/exp01_benchmark_laplacian/figures/architecture_tanh_mlp_768_768_512_512_1_device_cuda_dim_5_name_bilaplacian_vary_batch_size.pdf}
    \\
    \textbf{Stochastic}
    & \includegraphics{../jet/exp/exp01_benchmark_laplacian/figures/architecture_tanh_mlp_768_768_512_512_1_batch_size_2048_device_cuda_dim_50_distribution_normal_name_laplacian_vary_num_samples.pdf}
    & \includegraphics{../jet/exp/exp01_benchmark_laplacian/figures/architecture_tanh_mlp_768_768_512_512_1_batch_size_2048_device_cuda_dim_50_distribution_normal_name_weighted_laplacian_vary_num_samples.pdf}
    & \includegraphics{../jet/exp/exp01_benchmark_laplacian/figures/architecture_tanh_mlp_768_768_512_512_1_batch_size_256_device_cuda_dim_5_distribution_normal_name_bilaplacian_vary_num_samples.pdf}
  \end{tabular}

  \newsavebox{\benchmarkLegend}
  \savebox{\benchmarkLegend}{
    \begin{tikzpicture}[font=\small]
      \matrix [%
      matrix of nodes,%
      ampersand replacement=\&,% to use inside a savebox
      nodes={anchor=west, align=left, inner sep=1pt},%
      column sep=1ex,%
      row sep=0ex,%
      ] (legend)
      {
        \& \draw[tab-blue] plot[mark=*] coordinates {(0,0)};
        \& Nested first-order AD\phantom{y}\quad
        \& \draw[tab-orange] plot[mark=triangle*, rotate=270] coordinates {(0,0)};
        \& Standard Taylor mode\quad
        \& \draw[tab-green] plot[mark=triangle*, rotate=90] coordinates {(0,0)};
        \& Collapsed Taylor mode (ours)
        \\
        \& \node[anchor=center]{\tikz\draw[thick] (0, 0) to ++(2.5ex, 0);};
        \& Differentiable
        \& \node[anchor=center, opacity=0.5]{\tikz\draw[thick, dashed] (0, 0) to ++(2.5ex, 0);};
        \& Non-differentiable
        \& \node[anchor=center, opacity=0.0]{\tikz\draw[thick] (0, 0) to ++(2.5ex, 0);};
        \\
      };

      \draw[gray, rounded corners] (current bounding box.north west) rectangle (current bounding box.south east);
    \end{tikzpicture}
  }
  \begin{tikzpicture}
    \node {\usebox{\benchmarkLegend}};
  \end{tikzpicture}

  \caption{\textbf{\textcolor{tab-green}{Collapsed Taylor mode} is faster than \textcolor{tab-orange}{standard Taylor mode} and \textcolor{tab-blue}{nested first-order automatic differentiaion} while using less or the same memory, \textcolor{black!50!white}{opaque} memory consumptions are for non-differentiable computations.}
    Results are on GPU and we use a $D \to 768 \to 768 \to 512 \to 512 \to 1$ MLP with tanh activations with $D=50$ for the Laplacians and $D=5$ for the Bi-harmonic operator.
    The exact computation varies the batch size, and the approximate computation fixes $N=2048$ for the Laplacians, and $N=256$ for the Bi-Laplacian, and varies the number of Monte-Carlo samples such that $S < D$ for the Laplacians, and $2 + 3S < 6D^2 - 3D + 6$ for the Bi-Laplacian (we can compute exactly otherwise).
    For each approach, we fit a line to the data and report the slope in \Cref{tab:benchmark}  to quantify the relative speedup and memory reduction.
  }
  \label{fig:benchmark}
\end{figure*}

\begin{table}[!t]
  \centering
  \caption{\textbf{Benchmark from \Cref{fig:benchmark} in numbers.}
    We fit linear functions and report their slopes, \ie how much run time and memory increase when incrementing the batch size or number of Monte-Carlo samples.
    Our collapsed Taylor mode is up to two times faster than nested first-order autodiff, while using 80\% of memory in the differentiable, and 70\% in the non-differentiable, setting.
    All numbers are shown with two significant digits and bold values are best according to parenthesized values.}
  \label{tab:benchmark}
  \vspace{1.5ex}
  % paths where the performances are stored
  \def\datapathLaplacianExact{../jet/exp/exp01_benchmark_laplacian/performance/architecture_tanh_mlp_768_768_512_512_1_device_cuda_dim_50_name_laplacian_vary_batch_size}
  \def\datapathLaplacianStochastic{../jet/exp/exp01_benchmark_laplacian/performance/architecture_tanh_mlp_768_768_512_512_1_batch_size_2048_device_cuda_dim_50_distribution_normal_name_laplacian_vary_num_samples}
  \def\datapathWeightedLaplacianExact{../jet/exp/exp01_benchmark_laplacian/performance/architecture_tanh_mlp_768_768_512_512_1_device_cuda_dim_50_name_weighted_laplacian_vary_batch_size}
  \def\datapathWeightedLaplacianStochastic{../jet/exp/exp01_benchmark_laplacian/performance/architecture_tanh_mlp_768_768_512_512_1_batch_size_2048_device_cuda_dim_50_distribution_normal_name_weighted_laplacian_vary_num_samples}
  \def\datapathBilaplacianExact{../jet/exp/exp01_benchmark_laplacian/performance/architecture_tanh_mlp_768_768_512_512_1_device_cuda_dim_5_name_bilaplacian_vary_batch_size}
  \def\datapathBilaplacianStochastic{../jet/exp/exp01_benchmark_laplacian/performance/architecture_tanh_mlp_768_768_512_512_1_batch_size_256_device_cuda_dim_5_distribution_normal_name_bilaplacian_vary_num_samples}
  \resizebox{\linewidth}{!}{%
    % configuration options for the \num command
    \sisetup{%
      % scientific-notation=true,%
      round-mode=figures,%
      round-precision=2,%
      detect-weight, % for bolding to work
      tight-spacing=true, % less space around \cdot
    }
    \begin{tabular}{ccc|cccc}
      \toprule
      \textbf{Mode}
      & \makecell{\textbf{Per-datum or } \\ \textbf{-sample cost}}
      & \textbf{Implementation}
      & \textbf{Laplacian}
      & \makecell{\textbf{Weighted} \\ \textbf{Laplacian}}
      & \textbf{Bi-harmonic}
      \\
      \midrule
      \multirow{9}{*}{\textbf{Exact}}
      & \multirow{3}{*}{Time [ms]}
      & \textcolor{tab-blue}{Nested first-order}
      & \input{\datapathLaplacianExact/hessian_trace_best.txt}
      & \input{\datapathWeightedLaplacianExact/hessian_trace_best.txt}
      & \input{\datapathBilaplacianExact/hessian_trace_best.txt}
      \\
      &
      & \textcolor{tab-orange}{Standard Taylor}
      & \input{\datapathLaplacianExact/jet_naive_best.txt}
      & \input{\datapathWeightedLaplacianExact/jet_naive_best.txt}
      & \input{\datapathBilaplacianExact/jet_naive_best.txt}
      \\
      &
      & \textcolor{tab-green}{Collapsed (ours)}
      & \textbf{\input{\datapathLaplacianExact/jet_simplified_best.txt}}
      & \textbf{\input{\datapathWeightedLaplacianExact/jet_simplified_best.txt}}
      & \textbf{\input{\datapathBilaplacianExact/jet_simplified_best.txt}}
      \\ \cmidrule{2-6}
      & \multirow{3}{*}{\makecell{Mem.\,[MiB] \\ (differentiable)}}
      & \textcolor{tab-blue}{Nested first-order}
      & \input{\datapathLaplacianExact/hessian_trace_peakmem.txt}
      & \input{\datapathWeightedLaplacianExact/hessian_trace_peakmem.txt}
      & \textbf{\input{\datapathBilaplacianExact/hessian_trace_peakmem.txt}}
      \\
      &
      & \textcolor{tab-orange}{Standard Taylor}
      & \input{\datapathLaplacianExact/jet_naive_peakmem.txt}
      & \input{\datapathWeightedLaplacianExact/jet_naive_peakmem.txt}
      & \input{\datapathBilaplacianExact/jet_naive_peakmem.txt}
      \\
      &
      & \textcolor{tab-green}{Collapsed (ours)}
      & \textbf{\input{\datapathLaplacianExact/jet_simplified_peakmem.txt}}
      & \textbf{\input{\datapathWeightedLaplacianExact/jet_simplified_peakmem.txt}}
      & \input{\datapathBilaplacianExact/jet_simplified_peakmem.txt}
      \\ \cmidrule{2-6}
      & \multirow{3}{*}{\makecell{Mem.\,[MiB] \\ (non-diff.)}}
      & \textcolor{tab-blue}{Nested first-order}
      & \input{\datapathLaplacianExact/hessian_trace_peakmem_nondifferentiable.txt}
      & \input{\datapathWeightedLaplacianExact/hessian_trace_peakmem_nondifferentiable.txt}
      & \input{\datapathBilaplacianExact/hessian_trace_peakmem_nondifferentiable.txt}
      \\
      &
      & \textcolor{tab-orange}{Standard Taylor}
      & \textbf{\input{\datapathLaplacianExact/jet_naive_peakmem_nondifferentiable.txt}}
      & \textbf{\input{\datapathWeightedLaplacianExact/jet_naive_peakmem_nondifferentiable.txt}}
      & \input{\datapathBilaplacianExact/jet_naive_peakmem_nondifferentiable.txt}
      \\
      &
      & \textcolor{tab-green}{Collapsed (ours)}
      & \textbf{\input{\datapathLaplacianExact/jet_simplified_peakmem_nondifferentiable.txt}}
      & \textbf{\input{\datapathWeightedLaplacianExact/jet_simplified_peakmem_nondifferentiable.txt}}
      & \textbf{\input{\datapathBilaplacianExact/jet_simplified_peakmem_nondifferentiable.txt}}
      \\
      \midrule
      \multirow{9}{*}{\textbf{Stochastic}}
      & \multirow{3}{*}{Time [ms]}
      & \textcolor{tab-blue}{Nested first-order}
      & \input{\datapathLaplacianStochastic/hessian_trace_best.txt}
      & \input{\datapathWeightedLaplacianStochastic/hessian_trace_best.txt}
      & \input{\datapathBilaplacianStochastic/hessian_trace_best.txt}
      \\
      &
      & \textcolor{tab-orange}{Standard Taylor}
      & \input{\datapathLaplacianStochastic/jet_naive_best.txt}
      & \input{\datapathWeightedLaplacianStochastic/jet_naive_best.txt}
      & \input{\datapathBilaplacianStochastic/jet_naive_best.txt}
      \\
      &
      & \textcolor{tab-green}{Collapsed (ours)}
      & \textbf{\input{\datapathLaplacianStochastic/jet_simplified_best.txt}}
      & \textbf{\input{\datapathWeightedLaplacianStochastic/jet_simplified_best.txt}}
      & \textbf{\input{\datapathBilaplacianStochastic/jet_simplified_best.txt}}
      \\ \cmidrule{2-6}
      & \multirow{3}{*}{\makecell{Mem.\,[MiB]\\(differentiable)}}
      & \textcolor{tab-blue}{Nested first-order}
      & \input{\datapathLaplacianStochastic/hessian_trace_peakmem.txt}
      & \input{\datapathWeightedLaplacianStochastic/hessian_trace_peakmem.txt}
      & \input{\datapathBilaplacianStochastic/hessian_trace_peakmem.txt}
      \\
      &
      & \textcolor{tab-orange}{Standard Taylor}
      & \input{\datapathLaplacianStochastic/jet_naive_peakmem.txt}
      & \input{\datapathWeightedLaplacianStochastic/jet_naive_peakmem.txt}
      & \input{\datapathBilaplacianStochastic/jet_naive_peakmem.txt}
      \\
      &
      & \textcolor{tab-green}{Collapsed (ours)}
      & \textbf{\input{\datapathLaplacianStochastic/jet_simplified_peakmem.txt}}
      & \textbf{\input{\datapathWeightedLaplacianStochastic/jet_simplified_peakmem.txt}}
      & \textbf{\input{\datapathBilaplacianStochastic/jet_simplified_peakmem.txt}}
      \\ \cmidrule{2-6}
      & \multirow{3}{*}{\makecell{Mem.\,[MiB]\\(non-diff.)}}
      & \textcolor{tab-blue}{Nested first-order}
      & \input{\datapathLaplacianStochastic/hessian_trace_peakmem_nondifferentiable.txt}
      & \input{\datapathWeightedLaplacianStochastic/hessian_trace_peakmem_nondifferentiable.txt}
      & \input{\datapathBilaplacianStochastic/hessian_trace_peakmem_nondifferentiable.txt}
      \\
      &
      & \textcolor{tab-orange}{Standard Taylor}
      & \textbf{\input{\datapathLaplacianStochastic/jet_naive_peakmem_nondifferentiable.txt}}
      & \textbf{\input{\datapathWeightedLaplacianStochastic/jet_naive_peakmem_nondifferentiable.txt}}
      & \input{\datapathBilaplacianStochastic/jet_naive_peakmem_nondifferentiable.txt}
      \\
      &
      & \textcolor{tab-green}{Collapsed (ours)}
      & \textbf{\input{\datapathLaplacianStochastic/jet_simplified_peakmem_nondifferentiable.txt}}
      & \textbf{\input{\datapathWeightedLaplacianStochastic/jet_simplified_peakmem_nondifferentiable.txt}}
      & \textbf{\input{\datapathBilaplacianStochastic/jet_simplified_peakmem_nondifferentiable.txt}}
      \\
      \bottomrule
    \end{tabular}
  }
\end{table}

%%% Local Variables:
%%% mode: LaTeX
%%% TeX-master: "../main"
%%% End:


% \section{Discussion and Conclusion}\label{sec:conclusion}
% Computing differential operators of high-dimensional functions is a critical component in scientific machine learning, particularly for physics-informed neural networks and variational Monte Carlo. 
While Taylor-mode automatic differentiation promises efficient computation of higher-order derivatives, we found that vanilla implementations often underperform compared to nested backpropagation. 
Our work introduces collapsed Taylor-mode AD, a simple yet effective optimization that propagates the sum of highest-order coefficients directly through the computational graph. 
This approach (1) unifies and generalizes recent advances in forward-mode schemes, showing that the forward Laplacian emerges naturally from collapsing standard Taylor mode, while extending to other differential operators and stochastic variants, 
(2) demonstrates that such optimizations can be achieved through simple graph rewrites based on linearity, making it amenable to integration into existing just-in-time compilers without requiring specialized interfaces, and (3)
provides substantial empirical improvements over both vanilla Taylor mode and nested backpropagation, with up to 2x speedup for Laplacian operators and 9x for randomized Bi-harmonic operators, while often using less memory. 
Our PyTorch implementation and experiments confirm that these theoretical benefits translate into practical performance gains. 
The success of collapsed Taylor mode suggests that forward-mode AD schemes, when properly optimized, can outperform the traditional backpropagation approach for computing PDE operators. 
We believe this work takes an important step toward making Taylor mode a practical alternative in scientific machine learning, while maintaining ease of use through potential compiler integration. 
Future work could focus on integrating these optimizations directly into ML framework compilers, extending support to more primitive operations, and exploring additional graph-based optimizations for automatic differentiation. 
%%% Local Variables:
%%% mode: LaTeX
%%% TeX-master: "../main"
%%% End:

% \begin{ack}
  Funding statements and acknowledgements go here.
\end{ack}
%%% Local Variables:
%%% mode: LaTeX
%%% TeX-master: "../main"
%%% End:


\bibliography{references}
\bibliographystyle{icml2024.bst}

\clearpage
\appendix

% Label appendix equations as (A1), (B10) etc.
\renewcommand\theequation{\thesection\arabic{equation}}
\renewcommand\thefigure{\thesection\arabic{figure}}
\renewcommand\thetable{\thesection\arabic{table}}

\section{Visual Tour: From Function to Collapsed Taylor Mode}\label{sec:appendix-visual-tour}
\input{figures/interface_overview}
\clearpage

\section{Linearity for Taylor-mode AD}


The following Taylor-mode AD scheme results from inserting \cref{eq:sum-k-jet} into the \cref{eq:taylor-mode-composition}.
\begin{align}\label{eq:sum-taylor-mode-naive}
  \begin{split}
      \begin{pmatrix*}
        \vx_0
        \\
       \{\vx_{1,n}\}
        \\
        \{\vx_{2,n}\}
        \\
        \vdots
        \\
        \{\vx_{K, n}\}
      \end{pmatrix*}
      &\overset{\text{(\ref{eq:faa-di-bruno})}}{\to}
        \begin{pmatrix*}[l]
          \vh_0 =  \vh(\vx_0)
          \\
          \{\vh_{1,n}\} =  
          \left\{
          \left< 
          \partial \vh(\vx_0),
          \vx_{1, n} 
          \right>
          \right\}
          \\
          \{\vh_{2, n}\} = 
          \left\{
          \left< 
          \partial^2 \vh(\vx_0),
          \vx_{1, n} \otimes \vx_{1, n} 
          \right>
          +
          \left<
          \partial \vh(\vx_0),
          \vx_{2, n}
          \right>
          \right\}
          \\
          \vdots
          \\
          \{\vh_{K, n}\} =
          \left\{
          \displaystyle \sum_{
          \sigma \in \partitioning(K)
          }
          \nu(\sigma) \left<
          \partial^{|\sigma|} \vh(\vx_0),
          \tensorprod{s \in \sigma} \vx_{s, n}
          \right>
          \right\}
        \end{pmatrix*} 
        \\
        &\overset{\text{(\ref{eq:faa-di-bruno})}}{\to}
        \begin{pmatrix*}[l]
          \vg_0 =  \vg(\vh_0)
          \\
          \{\vg_{1,n}\} = 
          \left\{
          \left< 
          \partial \vg(\vh_0), 
          \vh_{1, n}
          \right>
          \right\}
          \\
          \{\vg_{2,n}\} = 
          \left\{
          \left<
          \partial^2 \vg(\vh_0), 
          \vh_{1, n} \otimes \vh_{1, n}\right>
          + 
          \left< \partial \vg(\vh_0),
          \vh_{2, n}
          \right>
          \right\}
          \\
          \vdots
          \\
          \{\vg_{K,n}\} =
          \left\{
          \displaystyle\sum_{
          \sigma \in \partitioning(K)
          } 
          \nu(\sigma) \left<
          \partial^{|\sigma|} \vg(\vh_0),
          \tensorprod{s \in \sigma} \vh_{s, n}
          \right>
          \right\}
        \end{pmatrix*}
        \\
        &\overset{\text{(\ref{eq:faa-di-bruno})}}{=}
        \begin{pmatrix*}[l]
          \vf_0 =  \vf(\vx_0)
          \\
          \{\vf_{1,n}\} = \left\{
          \left<
          \partial \vf(\vx_0), 
          \vx_{1, n}
          \right>
          \right\}
          \\
          \{\vf_{2,n}\} = \left\{
          \left<
          \partial^2 \vf(\vx_0), 
          \vx_{1, n} \otimes \vx_{1, n}
          \right>
          + 
          \left< \partial \vf(\vx_0),
          \vx_{2, n}
          \right>
          \right\}
          \\
          \vdots
          \\
          \{\vf_{K, n}\} =
          \left\{
          \displaystyle\sum_{
          \sigma \in \partitioning(K)
          }
          \nu(\sigma) \left< 
          \partial^{|\sigma|} \vf(\vx_0),
          \tensorprod{s \in \sigma} \vx_{s, n}
          \right>
          \right\}
        \end{pmatrix*}
        \\
        &\overset{\text{slice}}{\to} \{ \vg_{K,n} \}
        \\
        &\overset{\text{sum}}{\to} \sum_{n=1}^N \vg_{K,n} 
          \overset{\{\vx_{1,n} = \vv_n, \vx_{2,n} = \vzero, \dots, \vx_{K, n} = \vzero \}}{=} 
          \sum_{n=1}^N \left<
          \partial^K \vf(\vx_0), 
          \otimes_{k=1}^K \vv_n
          \right>
    \end{split}
\end{align}


Can we do better?

\begin{align}\label{eq:sum-taylor-mode-efficient}
  \begin{split}
      \begin{pmatrix*}
        \vx_0
        \\
       \{\vx_{1,n}\}
        \\
        \{\vx_{2,n}\}
        \\
        \vdots
        \\
        \textcolor{\colorcTM}{\displaystyle\sum_{n = 1}^N\vx_{K, n}}
      \end{pmatrix*}
      &\overset{\text{(\ref{eq:faa-di-bruno})}}{\to}
        \begin{pmatrix*}[l]
          \vh_0 &=  \vh(\vx_0)
          \\
          \{\vh_{1,n}\} &=  
          \left\{
          \left< 
          \partial \vh(\vx_0),
          \vx_{1, n} 
          \right>
          \right\}
          \\
          \{\vh_{2, n}\} &= 
          \left\{
          \left< 
          \partial^2 \vh(\vx_0),
          \vx_{1, n} \otimes \vx_{1, n} 
          \right>
          +
          \left<
          \partial \vh(\vx_0),
          \vx_{2, n}
          \right>
          \right\}
          \\
          \vdots
          \\
          \textcolor{\colorcTM}{ \displaystyle \sum_{n = 1}^N \vh_{K, n}} &=
          \displaystyle \sum_{n=1}^N \sum_{
          \sigma \in \partitioning(K) \setminus \{\sigma_t\}
          }
          \nu(\sigma) \left<
          \partial^{|\sigma|} \vh(\vx_0),
          \tensorprod{s \in \sigma} \vx_{s, n}
          \right>
          \\
          &+ 
          \left<
          \partial \vh(\vx_0),
          \textcolor{\colorcTM}{
          \sum_{n=1}^N \vx_{K, n}
          }
          \right>
        \end{pmatrix*} 
        \\
        &\overset{\text{(\ref{eq:faa-di-bruno})}}{\to}
        \begin{pmatrix*}[l]
          \vg_0 &=  \vg(\vh_0)
          \\
          \{\vg_{1,n}\} &= 
          \left\{
          \left< 
          \partial \vg(\vh_0), 
          \vh_{1, n}
          \right>
          \right\}
          \\
          \{\vg_{2,n}\} &= 
          \left\{
          \left<
          \partial^2 \vg(\vh_0), 
          \vh_{1, n} \otimes \vh_{1, n}\right>
          + 
          \left< \partial \vg(\vh_0),
          \vh_{2, n}
          \right>
          \right\}
          \\
          \vdots
          \\
          \textcolor{\colorcTM}{\displaystyle\sum_{n=1}^N\vg_{K,n}} &=
          \displaystyle \sum_{n=1}^N \sum_{
          \sigma \in \partitioning(K) \setminus \{\sigma_t\}
          } 
          \nu(\sigma) \left<
          \partial^{|\sigma|} \vg(\vh_0),
          \tensorprod{s \in \sigma} \vh_{s, n}
          \right>
          \\
          &+
          \left<
          \partial \vg(\vh_0),
          \textcolor{\colorcTM}{\displaystyle \sum_{n=1}^N\vh_{K, n}}
          \right>
        \end{pmatrix*}
        \\
        &\overset{\text{(\ref{eq:faa-di-bruno})}}{=}
        \begin{pmatrix*}[l]
          \vf_0 &=  \vf(\vx_0)
          \\
          \{\vf_{1,n}\} &= \left\{
          \left<
          \partial \vf(\vx_0), 
          \vx_{1, n}
          \right>
          \right\}
          \\
          \{\vf_{2,n}\} &= \left\{
          \left<
          \partial^2 \vf(\vx_0), 
          \vx_{1, n} \otimes \vx_{1, n}
          \right>
          + 
          \left< \partial \vf(\vx_0),
          \vx_{2, n}
          \right>
          \right\}
          \\
          \vdots
          \\
          \textcolor{\colorcTM}{\displaystyle \sum_{n=1}^N \vf_{K, n}} &=
          \displaystyle \sum_{n=1}^N\sum_{
          \sigma \in \partitioning(K) \setminus \{ \sigma_t \}
          }
          \nu(\sigma) \left< 
          \partial^{|\sigma|} \vf(\vx_0),
          \tensorprod{s \in \sigma} \vx_{s, n}
          \right> 
          \\
          &+
          \left<
          \partial \vf(\vx_0),
          \textcolor{\colorcTM}{\displaystyle \sum_{n=1}^N\vx_{K, n}}
          \right>
        \end{pmatrix*}
        \\
        &\overset{\text{slice}}{\to} \textcolor{\colorcTM}{\sum_{n=1}^N \vg_{K,n}}
          \overset{\{\vx_{1,n} = \vv_n, \vx_{2,n} = \vzero, \dots, \vx_{K, n} = \vzero \}}{=} 
          \sum_{n=1}^N \left<
          \partial^K \vf(\vx_0), 
          \otimes_{k=1}^K \vv_n
          \right>
    \end{split}
\end{align}

\subsection{Examples -- Laplace Operator}

\paragraph{Naive Laplacian}
The propagation scheme below is obtained, by applying \cref{eq:sum-taylor-mode-naive} to \cref{eq:laplacian}.
\begin{align}\label{eq:laplacian-naive}
  \begin{split}
    \begin{pmatrix*}
      \vx_0
      \\
      \{\vx_{1,d} \}
      \\
      \{\vx_{2,d} \}
    \end{pmatrix*}
    &\overset{\text{(\ref{eq:taylor-mode-scalar})}}{\to}
      \begin{pmatrix*}[l]
        \vh_0 =  \vh(\vx_0)
        \\
        \{\vh_{1,d}\} = \{
        \left<
        \partial \vh(\vx_0),
        \vx_{1,d}
        \right>
        \}
        \\
        \{\vh_{2,d}\} = 
        \{
        \left<
        \partial^2 \vh(\vx_0),
        \vx_{1,d} \otimes \vx_{1,d} 
        \right>
        + 
        \left<
        \partial \vh(\vx_0),
        \vx_{2,d}
        \right>
        \}
      \end{pmatrix*}
    \\
    &\overset{\text{(\ref{eq:taylor-mode-scalar})}}{\to}
      \begin{pmatrix*}[l]
        g_0 =  g(\vh_0)
        \\
        \{g_{1,d}\} = \{
        \left<\partial g(\vh_0), 
        \vh_{1,d}
        \right>
        \}
        \\
        \{g_{2,d}\} = \{
        \left< 
        \partial^2 g(\vh_0),
        \vh_{1,d} \otimes \vh_{1,d} 
        \right>
        + 
        \left<
        \partial g(\vh_0), 
        \vh_{2,d}
        \right>
        \}
      \end{pmatrix*}
      \\
      &\overset{\text{(\ref{eq:taylor-mode-scalar})}}{=}
      \begin{pmatrix*}[l]
        f_0 =  \vf(\vx_0)
        \\
        \{f_{1,d} \} = \{
        \left<
        \partial f(\vx_0),
        \vx_{1,d}
        \right>
        \}
        \\
        \{ f_{2,d} \} = \{
        \left<
        \partial^2 f(\vx_0),
        \vx_{1,d} \otimes \vx_{1,d}
        \right>
        + 
        \left<
        \partial f(\vx_0), 
        \vx_{2,d}
        \right>
        \}
      \end{pmatrix*}
    \\
    &\overset{\text{slice}}{\to} \{ g_{2,d} \}
    \\
    &\overset{\text{sum}}{\to} \sum_{d=1}^D \{ g_{2,d} \}
      \overset{\{\vx_{1,d} = \ve_d, \vx_{2,d} = \vzero\}}{=} \Delta f(\vx_0)
  \end{split}
\end{align}


\paragraph{Efficient Laplacian}
Instead of naively applying \cref{eq:sum-taylor-mode-naive}, \cref{eq:sum-taylor-mode-efficient} is used to obtain a efficient Laplacian propagation approach:
\begin{align}\label{eq:laplacian-efficient}
  \begin{split}
    \begin{pmatrix*}
      \vx_0
      \\
      \{\vx_{1,d} \}
      \\
      \textcolor{\colorcTM}{\displaystyle\sum_{d=1}^D \vx_{2,d}}
    \end{pmatrix*}
    &\overset{\text{(\ref{eq:taylor-mode-scalar})}}{\to}
      \begin{pmatrix*}[l]
        \vh_0 =  \vh(\vx_0)
        \\
        \{\vh_{1,d}\} = \{
        \left<
        \partial \vh(\vx_0),
        \vx_{1,d}
        \right>
        \}
        \\
        \textcolor{\colorcTM}{\displaystyle\sum_{d=1}^D \vh_{2,d}} = \displaystyle\sum_{d=1}^D 
        \left< \partial^2 \vh(\vx_0), 
        \vx_{1,d} \otimes \vx_{1,d} 
        \right> 
        + 
        \left<
        \partial \vh(\vx_0),
        \textcolor{\colorcTM}{\displaystyle\sum_{d=1}^D\vx_{2,d}}
        \right>
      \end{pmatrix*}
    \\
    &\overset{\text{(\ref{eq:taylor-mode-scalar})}}{\to}
      \begin{pmatrix*}[l]
        \vg_0 =  \vg(\vh_0)
        \\
        \{\vg_{1,d}\} = \{
        \left<
        \partial \vg(\vh_0), 
        \vh_{1,d}
        \right>
        \}
        \\
        \textcolor{\colorcTM}{\displaystyle\sum_{d=1}^D\vg_{2,d}} 
        = 
        \displaystyle\sum_{d=1}^D
        \left< 
        \partial^2 \vg(\vh_0),
        \vh_{1,d} \otimes \vh_{1,d}
        \right>
        + 
        \left< 
        \partial \vg(\vh_0), 
        \textcolor{\colorcTM}{\displaystyle\sum_{d=1}^D\vh_{2,d}}
        \right>
      \end{pmatrix*} 
      \\
      &\overset{\text{(\ref{eq:taylor-mode-scalar})}}{=}
      \begin{pmatrix*}[l]
        \vf_0 =  \vf(\vx_0)
        \\
        \{\vf_{1,d} \} = \{ 
        \left<
        \partial \vf(\vx_0),
        \vx_{1,d}
        \right>
        \}
        \\
        \textcolor{\colorcTM}{\displaystyle\sum_{d=1}^D \vf_{2,d}}
        =
        \displaystyle\sum_{d=1}^D 
        \left<
        \partial^2 \vf(\vx_0),
        \vx_{1,d} \otimes \vx_{1,d}
        \right>
        +
        \left<
        \partial \vf(\vx_0),
        \textcolor{\colorcTM}{\displaystyle\sum_{i=1}^D\vx_{2,d}}
        \right>
      \end{pmatrix*}
    \\
    &\overset{\text{slice}}{\to} \textcolor{\colorcTM}{\sum_{d=1}^D \{ \vg_{2,d} \}}
      \overset{\{(\vx_{1,d} = \ve_d, \vx_{2,d} = \vzero)\}}{=}
      \Delta f(\vx_0)
  \end{split}
\end{align}

\section{Fa\`a Di Bruno Formula Cheat Sheet}\label{sec:faa-di-bruno-cheatsheet}
\begin{tiny}
  \begin{align*}
    \vx_{0}
    &\to&
          \vh_{0}
          =
          h(\vx_{0})
    &\to&
          \vg_{0}
          =
          g(\vh_{0})
          =
    &\vf_{0}
      =
      f(\vx_{0})
    \\
    \vx_{1}
    &\to&
          \vh_{1}
          =
          \partial h [\vx_{1}]
    &\to&
          \vg_{1}
          =
          \partial g [\vh_{1}]
          =
    &\vf_{1}
      =
      \partial f [\vx_{1}]
    \\
    \vx_{2}
    &\to&
          \vh_{2}
          =
          \begin{matrix}
            \partial^{2} h [\vx_{1}, \vx_{1}]
            \\
            +
            \partial h [\vx_{2}]
          \end{matrix}
    &\to&
          \vg_{2}
          =
          \begin{matrix}
            \partial^{2} g [\vh_{1}, \vh_{1}]
            \\
            +
            \partial g [\vh_{2}]
          \end{matrix}
          =
    &\vf_{2}
      =
      \begin{matrix}
        \partial^{2} f [\vx_{1}, \vx_{1}]
        \\
        +
        \partial f [\vx_{2}]
      \end{matrix}
    \\
    \vx_{3}
    &\to&
          \vh_{3}
          =
          \begin{matrix}
            \partial^{3} h [\vx_{1}, \vx_{1}, \vx_{1}]
            \\
            +
            3 \partial^{2} h [\vx_{1}, \vx_{2}]
            \\
            +
            \partial h [\vx_{3}]
          \end{matrix}
    &\to&
          \vg_{3}
          =
          \begin{matrix}
            \partial^{3} g [\vh_{1}, \vh_{1}, \vh_{1}]
            \\
            +
            3 \partial^{2} g [\vh_{1}, \vh_{2}]
            \\
            +
            \partial g [\vh_{3}]
          \end{matrix}
          =
    &\vf_{3}
      =
      \begin{matrix}
        \partial^{3} f [\vx_{1}, \vx_{1}, \vx_{1}]
        \\
        +
        3 \partial^{2} f [\vx_{1}, \vx_{2}]
        \\
        +
        \partial f [\vx_{3}]
      \end{matrix}
    \\
    \vx_{4}
    &\to&
          \vh_{4}
          =
          \begin{matrix}
            \partial^{4} h [\vx_{1}, \vx_{1}, \vx_{1}, \vx_{1}]
            \\
            +
            6 \partial^{3} h [\vx_{1}, \vx_{1}, \vx_{2}]
            \\
            +
            4 \partial^{2} h [\vx_{1}, \vx_{3}]
            \\
            +
            3 \partial^{2} h [\vx_{2}, \vx_{2}]
            \\
            +
            \partial h [\vx_{4}]
          \end{matrix}
    &\to&
          \vg_{4}
          =
          \begin{matrix}
            \partial^{4} g [\vh_{1}, \vh_{1}, \vh_{1}, \vh_{1}]
            \\
            +
            6 \partial^{3} g [\vh_{1}, \vh_{1}, \vh_{2}]
            \\
            +
            4 \partial^{2} g [\vh_{1}, \vh_{3}]
            \\
            +
            3 \partial^{2} g [\vh_{2}, \vh_{2}]
            \\
            +
            \partial g [\vh_{4}]
          \end{matrix}
          =
    &\vf_{4}
      =
      \begin{matrix}
        \partial^{4} f [\vx_{1}, \vx_{1}, \vx_{1}, \vx_{1}]
        \\
        +
        6 \partial^{3} f [\vx_{1}, \vx_{1}, \vx_{2}]
        \\
        +
        4 \partial^{2} f [\vx_{1}, \vx_{3}]
        \\
        +
        3 \partial^{2} f [\vx_{2}, \vx_{2}]
        \\
        +
        \partial f [\vx_{4}]
      \end{matrix}
    \\
    \vx_{5}
    &\to&
          \vh_{5}
          =
          \begin{matrix}
            \partial^{5} h [\vx_{1}, \vx_{1}, \vx_{1}, \vx_{1}, \vx_{1}]
            \\
            +
            10 \partial^{4} h [\vx_{1}, \vx_{1}, \vx_{1}, \vx_{2}]
            \\
            +
            10 \partial^{3} h [\vx_{1}, \vx_{1}, \vx_{3}]
            \\
            +
            15 \partial^{3} h [\vx_{1}, \vx_{2}, \vx_{2}]
            \\
            +
            5 \partial^{2} h [\vx_{1}, \vx_{4}]
            \\
            +
            10 \partial^{2} h [\vx_{2}, \vx_{3}]
            \\
            +
            \partial h [\vx_{5}]
          \end{matrix}
    &\to&
          \vg_{5}
          =
          \begin{matrix}
            \partial^{5} g [\vh_{1}, \vh_{1}, \vh_{1}, \vh_{1}, \vh_{1}]
            \\
            +
            10 \partial^{4} g [\vh_{1}, \vh_{1}, \vh_{1}, \vh_{2}]
            \\
            +
            10 \partial^{3} g [\vh_{1}, \vh_{1}, \vh_{3}]
            \\
            +
            15 \partial^{3} g [\vh_{1}, \vh_{2}, \vh_{2}]
            \\
            +
            5 \partial^{2} g [\vh_{1}, \vh_{4}]
            \\
            +
            10 \partial^{2} g [\vh_{2}, \vh_{3}]
            \\
            +
            \partial g [\vh_{5}]
          \end{matrix}
          =
    &\vf_{5}
      =
      \begin{matrix}
        \partial^{5} f [\vx_{1}, \vx_{1}, \vx_{1}, \vx_{1}, \vx_{1}]
        \\
        +
        10 \partial^{4} f [\vx_{1}, \vx_{1}, \vx_{1}, \vx_{2}]
        \\
        +
        10 \partial^{3} f [\vx_{1}, \vx_{1}, \vx_{3}]
        \\
        +
        15 \partial^{3} f [\vx_{1}, \vx_{2}, \vx_{2}]
        \\
        +
        5 \partial^{2} f [\vx_{1}, \vx_{4}]
        \\
        +
        10 \partial^{2} f [\vx_{2}, \vx_{3}]
        \\
        +
        \partial f [\vx_{5}]
      \end{matrix}
    \\
    \vx_{6}
    &\to&
          \vh_{6}
          =
          \begin{matrix}
            \partial^{6} h [\vx_{1}, \vx_{1}, \vx_{1}, \vx_{1}, \vx_{1}, \vx_{1}]
            \\
            +
            15 \partial^{5} h [\vx_{1}, \vx_{1}, \vx_{1}, \vx_{1}, \vx_{2}]
            \\
            +
            20 \partial^{4} h [\vx_{1}, \vx_{1}, \vx_{1}, \vx_{3}]
            \\
            +
            45 \partial^{4} h [\vx_{1}, \vx_{1}, \vx_{2}, \vx_{2}]
            \\
            +
            15 \partial^{3} h [\vx_{1}, \vx_{1}, \vx_{4}]
            \\
            +
            60 \partial^{3} h [\vx_{1}, \vx_{2}, \vx_{3}]
            \\
            +
            15 \partial^{3} h [\vx_{2}, \vx_{2}, \vx_{2}]
            \\
            +
            6 \partial^{2} h [\vx_{1}, \vx_{5}]
            \\
            +
            15 \partial^{2} h [\vx_{2}, \vx_{4}]
            \\
            +
            10 \partial^{2} h [\vx_{3}, \vx_{3}]
            \\
            +
            \partial h [\vx_{6}]
          \end{matrix}
    &\to&
          \vg_{6}
          =
          \begin{matrix}
            \partial^{6} g [\vh_{1}, \vh_{1}, \vh_{1}, \vh_{1}, \vh_{1}, \vh_{1}]
            \\
            +
            15 \partial^{5} g [\vh_{1}, \vh_{1}, \vh_{1}, \vh_{1}, \vh_{2}]
            \\
            +
            20 \partial^{4} g [\vh_{1}, \vh_{1}, \vh_{1}, \vh_{3}]
            \\
            +
            45 \partial^{4} g [\vh_{1}, \vh_{1}, \vh_{2}, \vh_{2}]
            \\
            +
            15 \partial^{3} g [\vh_{1}, \vh_{1}, \vh_{4}]
            \\
            +
            60 \partial^{3} g [\vh_{1}, \vh_{2}, \vh_{3}]
            \\
            +
            15 \partial^{3} g [\vh_{2}, \vh_{2}, \vh_{2}]
            \\
            +
            6 \partial^{2} g [\vh_{1}, \vh_{5}]
            \\
            +
            15 \partial^{2} g [\vh_{2}, \vh_{4}]
            \\
            +
            10 \partial^{2} g [\vh_{3}, \vh_{3}]
            \\
            +
            \partial g [\vh_{6}]
          \end{matrix}
          =
    &\vf_{6}
      =
      \begin{matrix}
        \partial^{6} f [\vx_{1}, \vx_{1}, \vx_{1}, \vx_{1}, \vx_{1}, \vx_{1}]
        \\
        +
        15 \partial^{5} f [\vx_{1}, \vx_{1}, \vx_{1}, \vx_{1}, \vx_{2}]
        \\
        +
        20 \partial^{4} f [\vx_{1}, \vx_{1}, \vx_{1}, \vx_{3}]
        \\
        +
        45 \partial^{4} f [\vx_{1}, \vx_{1}, \vx_{2}, \vx_{2}]
        \\
        +
        15 \partial^{3} f [\vx_{1}, \vx_{1}, \vx_{4}]
        \\
        +
        60 \partial^{3} f [\vx_{1}, \vx_{2}, \vx_{3}]
        \\
        +
        15 \partial^{3} f [\vx_{2}, \vx_{2}, \vx_{2}]
        \\
        +
        6 \partial^{2} f [\vx_{1}, \vx_{5}]
        \\
        +
        15 \partial^{2} f [\vx_{2}, \vx_{4}]
        \\
        +
        10 \partial^{2} f [\vx_{3}, \vx_{3}]
        \\
        +
        \partial f [\vx_{6}]
      \end{matrix}
    \\
    \vx_{7}
    &\to&
          \vh_{7}
          =
          \begin{matrix}
            \partial^{7} h [\vx_{1}, \vx_{1}, \vx_{1}, \vx_{1}, \vx_{1}, \vx_{1}, \vx_{1}]
            \\
            +
            21 \partial^{6} h [\vx_{1}, \vx_{1}, \vx_{1}, \vx_{1}, \vx_{1}, \vx_{2}]
            \\
            +
            35 \partial^{5} h [\vx_{1}, \vx_{1}, \vx_{1}, \vx_{1}, \vx_{3}]
            \\
            +
            105 \partial^{5} h [\vx_{1}, \vx_{1}, \vx_{1}, \vx_{2}, \vx_{2}]
            \\
            +
            35 \partial^{4} h [\vx_{1}, \vx_{1}, \vx_{1}, \vx_{4}]
            \\
            +
            210 \partial^{4} h [\vx_{1}, \vx_{1}, \vx_{2}, \vx_{3}]
            \\
            +
            105 \partial^{4} h [\vx_{1}, \vx_{2}, \vx_{2}, \vx_{2}]
            \\
            +
            21 \partial^{3} h [\vx_{1}, \vx_{1}, \vx_{5}]
            \\
            +
            105 \partial^{3} h [\vx_{1}, \vx_{2}, \vx_{4}]
            \\
            +
            70 \partial^{3} h [\vx_{1}, \vx_{3}, \vx_{3}]
            \\
            +
            105 \partial^{3} h [\vx_{2}, \vx_{2}, \vx_{3}]
            \\
            +
            7 \partial^{2} h [\vx_{1}, \vx_{6}]
            \\
            +
            21 \partial^{2} h [\vx_{2}, \vx_{5}]
            \\
            +
            35 \partial^{2} h [\vx_{3}, \vx_{4}]
            \\
            +
            \partial h [\vx_{7}]
          \end{matrix}
    &\to&
          \vg_{7}
          =
          \begin{matrix}
            \partial^{7} g [\vh_{1}, \vh_{1}, \vh_{1}, \vh_{1}, \vh_{1}, \vh_{1}, \vh_{1}]
            \\
            +
            21 \partial^{6} g [\vh_{1}, \vh_{1}, \vh_{1}, \vh_{1}, \vh_{1}, \vh_{2}]
            \\
            +
            35 \partial^{5} g [\vh_{1}, \vh_{1}, \vh_{1}, \vh_{1}, \vh_{3}]
            \\
            +
            105 \partial^{5} g [\vh_{1}, \vh_{1}, \vh_{1}, \vh_{2}, \vh_{2}]
            \\
            +
            35 \partial^{4} g [\vh_{1}, \vh_{1}, \vh_{1}, \vh_{4}]
            \\
            +
            210 \partial^{4} g [\vh_{1}, \vh_{1}, \vh_{2}, \vh_{3}]
            \\
            +
            105 \partial^{4} g [\vh_{1}, \vh_{2}, \vh_{2}, \vh_{2}]
            \\
            +
            21 \partial^{3} g [\vh_{1}, \vh_{1}, \vh_{5}]
            \\
            +
            105 \partial^{3} g [\vh_{1}, \vh_{2}, \vh_{4}]
            \\
            +
            70 \partial^{3} g [\vh_{1}, \vh_{3}, \vh_{3}]
            \\
            +
            105 \partial^{3} g [\vh_{2}, \vh_{2}, \vh_{3}]
            \\
            +
            7 \partial^{2} g [\vh_{1}, \vh_{6}]
            \\
            +
            21 \partial^{2} g [\vh_{2}, \vh_{5}]
            \\
            +
            35 \partial^{2} g [\vh_{3}, \vh_{4}]
            \\
            +
            \partial g [\vh_{7}]
          \end{matrix}
          =
    &\vf_{7}
      =
      \begin{matrix}
        \partial^{7} f [\vx_{1}, \vx_{1}, \vx_{1}, \vx_{1}, \vx_{1}, \vx_{1}, \vx_{1}]
        \\
        +
        21 \partial^{6} f [\vx_{1}, \vx_{1}, \vx_{1}, \vx_{1}, \vx_{1}, \vx_{2}]
        \\
        +
        35 \partial^{5} f [\vx_{1}, \vx_{1}, \vx_{1}, \vx_{1}, \vx_{3}]
        \\
        +
        105 \partial^{5} f [\vx_{1}, \vx_{1}, \vx_{1}, \vx_{2}, \vx_{2}]
        \\
        +
        35 \partial^{4} f [\vx_{1}, \vx_{1}, \vx_{1}, \vx_{4}]
        \\
        +
        210 \partial^{4} f [\vx_{1}, \vx_{1}, \vx_{2}, \vx_{3}]
        \\
        +
        105 \partial^{4} f [\vx_{1}, \vx_{2}, \vx_{2}, \vx_{2}]
        \\
        +
        21 \partial^{3} f [\vx_{1}, \vx_{1}, \vx_{5}]
        \\
        +
        105 \partial^{3} f [\vx_{1}, \vx_{2}, \vx_{4}]
        \\
        +
        70 \partial^{3} f [\vx_{1}, \vx_{3}, \vx_{3}]
        \\
        +
        105 \partial^{3} f [\vx_{2}, \vx_{2}, \vx_{3}]
        \\
        +
        7 \partial^{2} f [\vx_{1}, \vx_{6}]
        \\
        +
        21 \partial^{2} f [\vx_{2}, \vx_{5}]
        \\
        +
        35 \partial^{2} f [\vx_{3}, \vx_{4}]
        \\
        +
        \partial f [\vx_{7}]
      \end{matrix}
    \\
    \vx_{8}
    &\to&
          \vh_{8}
          =
          \begin{matrix}
            \partial^{8} h [\vx_{1}, \vx_{1}, \vx_{1}, \vx_{1}, \vx_{1}, \vx_{1}, \vx_{1}, \vx_{1}]
            \\
            +
            28 \partial^{7} h [\vx_{1}, \vx_{1}, \vx_{1}, \vx_{1}, \vx_{1}, \vx_{1}, \vx_{2}]
            \\
            +
            56 \partial^{6} h [\vx_{1}, \vx_{1}, \vx_{1}, \vx_{1}, \vx_{1}, \vx_{3}]
            \\
            +
            210 \partial^{6} h [\vx_{1}, \vx_{1}, \vx_{1}, \vx_{1}, \vx_{2}, \vx_{2}]
            \\
            +
            70 \partial^{5} h [\vx_{1}, \vx_{1}, \vx_{1}, \vx_{1}, \vx_{4}]
            \\
            +
            560 \partial^{5} h [\vx_{1}, \vx_{1}, \vx_{1}, \vx_{2}, \vx_{3}]
            \\
            +
            420 \partial^{5} h [\vx_{1}, \vx_{1}, \vx_{2}, \vx_{2}, \vx_{2}]
            \\
            +
            56 \partial^{4} h [\vx_{1}, \vx_{1}, \vx_{1}, \vx_{5}]
            \\
            +
            420 \partial^{4} h [\vx_{1}, \vx_{1}, \vx_{2}, \vx_{4}]
            \\
            +
            280 \partial^{4} h [\vx_{1}, \vx_{1}, \vx_{3}, \vx_{3}]
            \\
            +
            840 \partial^{4} h [\vx_{1}, \vx_{2}, \vx_{2}, \vx_{3}]
            \\
            +
            105 \partial^{4} h [\vx_{2}, \vx_{2}, \vx_{2}, \vx_{2}]
            \\
            +
            28 \partial^{3} h [\vx_{1}, \vx_{1}, \vx_{6}]
            \\
            +
            168 \partial^{3} h [\vx_{1}, \vx_{2}, \vx_{5}]
            \\
            +
            280 \partial^{3} h [\vx_{1}, \vx_{3}, \vx_{4}]
            \\
            +
            210 \partial^{3} h [\vx_{2}, \vx_{2}, \vx_{4}]
            \\
            +
            280 \partial^{3} h [\vx_{2}, \vx_{3}, \vx_{3}]
            \\
            +
            8 \partial^{2} h [\vx_{1}, \vx_{7}]
            \\
            +
            28 \partial^{2} h [\vx_{2}, \vx_{6}]
            \\
            +
            56 \partial^{2} h [\vx_{3}, \vx_{5}]
            \\
            +
            35 \partial^{2} h [\vx_{4}, \vx_{4}]
            \\
            +
            \partial h [\vx_{8}]
          \end{matrix}
    &\to&
          \vg_{8}
          =
          \begin{matrix}
            \partial^{8} g [\vh_{1}, \vh_{1}, \vh_{1}, \vh_{1}, \vh_{1}, \vh_{1}, \vh_{1}, \vh_{1}]
            \\
            +
            28 \partial^{7} g [\vh_{1}, \vh_{1}, \vh_{1}, \vh_{1}, \vh_{1}, \vh_{1}, \vh_{2}]
            \\
            +
            56 \partial^{6} g [\vh_{1}, \vh_{1}, \vh_{1}, \vh_{1}, \vh_{1}, \vh_{3}]
            \\
            +
            210 \partial^{6} g [\vh_{1}, \vh_{1}, \vh_{1}, \vh_{1}, \vh_{2}, \vh_{2}]
            \\
            +
            70 \partial^{5} g [\vh_{1}, \vh_{1}, \vh_{1}, \vh_{1}, \vh_{4}]
            \\
            +
            560 \partial^{5} g [\vh_{1}, \vh_{1}, \vh_{1}, \vh_{2}, \vh_{3}]
            \\
            +
            420 \partial^{5} g [\vh_{1}, \vh_{1}, \vh_{2}, \vh_{2}, \vh_{2}]
            \\
            +
            56 \partial^{4} g [\vh_{1}, \vh_{1}, \vh_{1}, \vh_{5}]
            \\
            +
            420 \partial^{4} g [\vh_{1}, \vh_{1}, \vh_{2}, \vh_{4}]
            \\
            +
            280 \partial^{4} g [\vh_{1}, \vh_{1}, \vh_{3}, \vh_{3}]
            \\
            +
            840 \partial^{4} g [\vh_{1}, \vh_{2}, \vh_{2}, \vh_{3}]
            \\
            +
            105 \partial^{4} g [\vh_{2}, \vh_{2}, \vh_{2}, \vh_{2}]
            \\
            +
            28 \partial^{3} g [\vh_{1}, \vh_{1}, \vh_{6}]
            \\
            +
            168 \partial^{3} g [\vh_{1}, \vh_{2}, \vh_{5}]
            \\
            +
            280 \partial^{3} g [\vh_{1}, \vh_{3}, \vh_{4}]
            \\
            +
            210 \partial^{3} g [\vh_{2}, \vh_{2}, \vh_{4}]
            \\
            +
            280 \partial^{3} g [\vh_{2}, \vh_{3}, \vh_{3}]
            \\
            +
            8 \partial^{2} g [\vh_{1}, \vh_{7}]
            \\
            +
            28 \partial^{2} g [\vh_{2}, \vh_{6}]
            \\
            +
            56 \partial^{2} g [\vh_{3}, \vh_{5}]
            \\
            +
            35 \partial^{2} g [\vh_{4}, \vh_{4}]
            \\
            +
            \partial g [\vh_{8}]
          \end{matrix}
          =
    &\vf_{8}
      =
      \begin{matrix}
        \partial^{8} f [\vx_{1}, \vx_{1}, \vx_{1}, \vx_{1}, \vx_{1}, \vx_{1}, \vx_{1}, \vx_{1}]
        \\
        +
        28 \partial^{7} f [\vx_{1}, \vx_{1}, \vx_{1}, \vx_{1}, \vx_{1}, \vx_{1}, \vx_{2}]
        \\
        +
        56 \partial^{6} f [\vx_{1}, \vx_{1}, \vx_{1}, \vx_{1}, \vx_{1}, \vx_{3}]
        \\
        +
        210 \partial^{6} f [\vx_{1}, \vx_{1}, \vx_{1}, \vx_{1}, \vx_{2}, \vx_{2}]
        \\
        +
        70 \partial^{5} f [\vx_{1}, \vx_{1}, \vx_{1}, \vx_{1}, \vx_{4}]
        \\
        +
        560 \partial^{5} f [\vx_{1}, \vx_{1}, \vx_{1}, \vx_{2}, \vx_{3}]
        \\
        +
        420 \partial^{5} f [\vx_{1}, \vx_{1}, \vx_{2}, \vx_{2}, \vx_{2}]
        \\
        +
        56 \partial^{4} f [\vx_{1}, \vx_{1}, \vx_{1}, \vx_{5}]
        \\
        +
        420 \partial^{4} f [\vx_{1}, \vx_{1}, \vx_{2}, \vx_{4}]
        \\
        +
        280 \partial^{4} f [\vx_{1}, \vx_{1}, \vx_{3}, \vx_{3}]
        \\
        +
        840 \partial^{4} f [\vx_{1}, \vx_{2}, \vx_{2}, \vx_{3}]
        \\
        +
        105 \partial^{4} f [\vx_{2}, \vx_{2}, \vx_{2}, \vx_{2}]
        \\
        +
        28 \partial^{3} f [\vx_{1}, \vx_{1}, \vx_{6}]
        \\
        +
        168 \partial^{3} f [\vx_{1}, \vx_{2}, \vx_{5}]
        \\
        +
        280 \partial^{3} f [\vx_{1}, \vx_{3}, \vx_{4}]
        \\
        +
        210 \partial^{3} f [\vx_{2}, \vx_{2}, \vx_{4}]
        \\
        +
        280 \partial^{3} f [\vx_{2}, \vx_{3}, \vx_{3}]
        \\
        +
        8 \partial^{2} f [\vx_{1}, \vx_{7}]
        \\
        +
        28 \partial^{2} f [\vx_{2}, \vx_{6}]
        \\
        +
        56 \partial^{2} f [\vx_{3}, \vx_{5}]
        \\
        +
        35 \partial^{2} f [\vx_{4}, \vx_{4}]
        \\
        +
        \partial f [\vx_{8}]
      \end{matrix}
  \end{align*}
\end{tiny}
%%% Local Variables:
%%% mode: LaTeX
%%% TeX-master: "../main"
%%% End:




\section{TTC}\label{sec:appendix_ttc}
This section introduces the notation we used in \cref{eq:ttc_general}. 
The right side of the formula sums over all $\vj
\in \mathbb{N}^I$ such that $\lVert \vj \rVert_1 := \sum_i [\vj]_i = K$. If $I=2$ and $\lVert \vj \rVert_1 = 4$, this index-set consists of $\left\{ (4,0), (0, 4), (3, 1), (1, 3),  (2, 2)\right\}$. 

The coefficient $\gamma_{\vi, \vj}$ is defined as
\begin{equation}
\label{eq:ttc_coeff}
    \gamma_{\vi, \vj} := \sum_{0 < \vm \leq \vi} (-1)^{\lVert \vi - \vm \rVert_1} 
    \left(
    \begin{matrix}
        \vi \\
        \vm
    \end{matrix}
    \right) 
    \left( 
    \begin{matrix}
        \lVert \vi \rVert_1 \frac{\vm}{\lVert \vm \rVert_1} \\
        \vj
    \end{matrix} 
    \right)  
    \left( 
    \frac{\lVert \vm \rVert_1}{\lVert \vi \rVert_1}
    \right)^{\lVert \vi \rVert_1}.
\end{equation}
The summation ranges over the set $\left\{ \vm \in \mathbb{N}^I \mid [\vm]_1 \leq [\vi]_1, \dots, [\vm]_I \leq [\vi]_I, \lVert \vm \rVert_1 > 0 \right\}$. Furthermore, we utilize the generalized binomial coefficient
\begin{equation}
    \left(
       \begin{matrix}
        a \\
        b
    \end{matrix}
    \right) := \prod_{l=0}^{b-1} \frac{a - l}{b - l}
\end{equation}
to allow the computation for all $a \in \mathbb{R}$ and $b \in \mathbb{N}$, which is defined to be $1$ if $b=0$. The generalized binomial coefficient of vectors is the product of all generalized binomial coefficients of the components:
$
\left(
    \begin{matrix}
        \va \\
        \vb
    \end{matrix}
    \right) 
    := 
    \prod_{i=1}^I
    \left(
       \begin{matrix}
        [\va]_i \\
        [\vb]_i
    \end{matrix}
    \right).
$ 
This notation also includes cases where the vector has components of $\mathbb{R}$.

\paragraph{Example coefficient computation}
Below we compute the coefficient $\gamma_{(2, 2),(3, 1)}$ as used to compute the biharmonic operator. This coefficient reads
\begin{align}
    \gamma_{(2, 2),(3, 1)} = \sum_{
    \substack{
    \vm \in \mathbb{N}, \; ||\vm||_1 > 0 
    \\
    [\vm]_1 \leq 2, \; [\vm]_2 \leq 2
    } 
    }
    (-1)^{ 2 - [\vm]_1 + 2 - [\vm]_2 } 
    \left(
    \begin{matrix}
        2 \\
        [\vm]_1
    \end{matrix}
    \right) 
    \left(
    \begin{matrix}
        2 \\
        [\vm]_2
    \end{matrix}
    \right) 
    \left( 
    \begin{matrix}
        4 \frac{[\vm]_1}{\lVert \vm \rVert_1} \\
        3
    \end{matrix} 
    \right) 
    \left( 
    \begin{matrix}
        4 \frac{[\vm]_2}{\lVert \vm \rVert_1} \\
        1
    \end{matrix} 
    \right)
    \left( 
    \frac{\lVert \vm \rVert_1}{4}
    \right)^4.
\end{align}
We have $\vm \in \{(1, 0), (2, 0), (1, 1), (2, 1), (2, 2), (1, 2), (0, 1), (0, 2)\}$, which results in the terms
\begin{align}
    &(-1)^{ 2 - 1 + 2 - 0} 
    \left(
    \begin{matrix}
        2 \\
        1
    \end{matrix}
    \right) 
    \left(
    \begin{matrix}
        2 \\
        0
    \end{matrix}
    \right) 
    \left( 
    \begin{matrix}
        4 \frac{1}{1} \\
        3
    \end{matrix} 
    \right) 
    \left( 
    \begin{matrix}
        4 \frac{0}{1} \\
        1
    \end{matrix} 
    \right)
    \left( 
    \frac{1}{4}
    \right)^4
    \\
    +&
    (-1)^{ 2 - 2 + 2 - 0 } 
    \left(
    \begin{matrix}
        2 \\
        2
    \end{matrix}
    \right) 
    \left(
    \begin{matrix}
        2 \\
        0
    \end{matrix}
    \right) 
    \left( 
    \begin{matrix}
        4 \frac{2}{2} \\
        3
    \end{matrix} 
    \right) 
    \left( 
    \begin{matrix}
        4 \frac{0}{2} \\
        1
    \end{matrix} 
    \right)
    \left( 
    \frac{2}{4}
    \right)^4
    \\
    +&
    (-1)^{ 2 - 1 + 2 - 1 } 
    \left(
    \begin{matrix}
        2 \\
        1
    \end{matrix}
    \right) 
    \left(
    \begin{matrix}
        2 \\
        1
    \end{matrix}
    \right) 
    \left( 
    \begin{matrix}
        4 \frac{1}{2} \\
        3
    \end{matrix} 
    \right) 
    \left( 
    \begin{matrix}
        4 \frac{1}{2} \\
        1
    \end{matrix} 
    \right)
    \left( 
    \frac{2}{4}
    \right)^4
    \\
    +&
    (-1)^{ 2 - 2 + 2 - 1 } 
    \left(
    \begin{matrix}
        2 \\
        2
    \end{matrix}
    \right) 
    \left(
    \begin{matrix}
        2 \\
        1
    \end{matrix}
    \right) 
    \left( 
    \begin{matrix}
        4 \frac{2}{3} \\
        3
    \end{matrix} 
    \right) 
    \left( 
    \begin{matrix}
        4 \frac{1}{3} \\
        1
    \end{matrix} 
    \right)
    \left( 
    \frac{3}{4}
    \right)^4
    \\
    +&
    (-1)^{ 2 - 2 + 2 - 2} 
    \left(
    \begin{matrix}
        2 \\
        2
    \end{matrix}
    \right) 
    \left(
    \begin{matrix}
        2 \\
        2
    \end{matrix}
    \right) 
    \left( 
    \begin{matrix}
        4 \frac{2}{4} \\
        3
    \end{matrix} 
    \right) 
    \left( 
    \begin{matrix}
        4 \frac{2}{4} \\
        1
    \end{matrix} 
    \right)
    \left( 
    \frac{4}{4}
    \right)^4
        \\
    +&
    (-1)^{ 2 - 1 + 2 - 2} 
    \left(
    \begin{matrix}
        2 \\
        1
    \end{matrix}
    \right) 
    \left(
    \begin{matrix}
        2 \\
        2
    \end{matrix}
    \right) 
    \left( 
    \begin{matrix}
        4 \frac{1}{3} \\
        3
    \end{matrix} 
    \right) 
    \left( 
    \begin{matrix}
        4 \frac{2}{3} \\
        1
    \end{matrix} 
    \right)
    \left( 
    \frac{3}{4}
    \right)^4
    \\
    +&
    (-1)^{ 2 - 0 + 2 - 1 } 
    \left(
    \begin{matrix}
        2 \\
        0
    \end{matrix}
    \right) 
    \left(
    \begin{matrix}
        2 \\
        1
    \end{matrix}
    \right) 
    \left( 
    \begin{matrix}
        4 \frac{0}{1} \\
        3
    \end{matrix} 
    \right) 
    \left( 
    \begin{matrix}
        4 \frac{1}{1} \\
        1
    \end{matrix} 
    \right)
    \left( 
    \frac{1}{4}
    \right)^4
    \\
    +&
    (-1)^{ 2 - 0 + 2 - 2 } 
    \left(
    \begin{matrix}
        2 \\
        0
    \end{matrix}
    \right) 
    \left(
    \begin{matrix}
        2 \\
        2
    \end{matrix}
    \right) 
    \left( 
    \begin{matrix}
        4 \frac{0}{2} \\
        3
    \end{matrix} 
    \right) 
    \left( 
    \begin{matrix}
        4 \frac{2}{2} \\
        1
    \end{matrix} 
    \right)
    \left( 
    \frac{2}{4}
    \right)^4
\end{align}
The next step is to evaluate the binomial coefficients.  
\begin{align}
    &(-1) \cdot 2 \cdot 1 \cdot 4 \cdot 0 \cdot 
    \left( 
    \frac{1}{4}
    \right)^4
    \\
    +&
    1 \cdot 1 \cdot 4 \cdot 0 \cdot 
    \left( 
    \frac{2}{4}
    \right)^4
    \\
    +&
    2 \cdot 2 \cdot 0 \cdot 2 \cdot
    \left( 
    \frac{2}{4}
    \right)^4
    \\
    -&
    1 \cdot 2 \cdot \frac{8}{9}\frac{5}{6}\frac{2}{3} \cdot \frac{4}{3} \cdot
    \left( 
    \frac{3}{4}
    \right)^4
    \\
    +&
    1 \cdot 1 \cdot 0 \cdot 2 \cdot
    \left( 
    \frac{4}{4}
    \right)^4
        \\
    -& 2 \cdot 1
    \cdot \frac{4}{9}\frac{1}{6} \frac{-2}{3} \cdot \frac{8}{3} \cdot
    \left( 
    \frac{3}{4}
    \right)^4
    \\
    -& 1 \cdot 2 \cdot 0 \cdot 4 \cdot
    \left( 
    \frac{1}{4}
    \right)^4
    \\
    +&
    \cdot 1 \cdot 1 \cdot 1 \cdot 0 \cdot 4 \cdot
    \left( 
    \frac{2}{4}
    \right)^4
\end{align}
Removing the terms that are zero gives us the final result
\begin{align}
    \gamma_{(2, 2),(3, 1)} = &(-1) \cdot 2 \cdot \frac{8}{9}\frac{5}{6}\frac{2}{3} \cdot \frac{4}{3} \cdot
    \left( 
    \frac{3}{4}
    \right)^4
    - 2 \left(\frac{4}{9} \cdot \frac{1}{6} \cdot \frac{-2}{3}\right) \cdot \frac{8}{3} \cdot
    \left( 
    \frac{3}{4}
    \right)^4 
    \\
    &=  \frac{-640}{486}\frac{81}{256} + \frac{128}{486}\frac{81}{256} = \frac{-5}{12}+\frac{1}{12} = -\frac{1}{3}.
\end{align}





\subsection{Applied to the Biharmonic Operator}\label{sec:appendix-biharmonic-details}

To compute \cref{eq:biharm} with \cref{eq:ttc_general}, we first select $K = 4, I = 2, D_1 = D_2 = D, \vi = (2, 2), \vv_{d_1} = \ve_{d_1}$ and $\ve_{d_2} = \ve_{d_2}$. Then we insert these parameters into the general equation \cref{eq:ttc_general} and get
\begin{equation} \label{eq:ttc_for_biharm}
    \begin{aligned}
    \Delta^2 \vf(\vx_0)
    &=
    \sum_{\vj \in \mathbb{N}^2, \lVert \vj \rVert_1 = 4}
    \gamma_{(2, 2), \vj}
    \frac{1}{4!}
    \sum_{d_1=1}^D \sum_{d_2=1}^D 
    \left<
    \partial^4 \vf(\vx_0),
    \left(\ve_{d_1} [\vj]_1 + \ve_{d_2} [\vj]_2\right)^{\otimes 4}
    \right>
    \\
    &=
    \frac{1}{24} 
    \Big(
    \gamma_{(2, 2), (4, 0)}
    \sum_{d_1=1}^D \sum_{d_2=1}^D
    \left<
    \partial^4 \vf(\vx_0),
    \left(4 \ve_{d_1}\right)^{\otimes 4}
    \right>
    \\
    &+
    \gamma_{(2, 2), (0, 4)}
    \sum_{d_1=1}^D \sum_{d_2=1}^D 
    \left<
    \partial^4 \vf(\vx_0),
    \left(4 \ve_{d_2} \right)^{\otimes 4}
    \right>
    \\
    &+
    \gamma_{(2, 2), (3, 1)}
    \sum_{d_1=1}^D \sum_{d_2=1}^D
    \left<
    \partial^4 \vf(\vx_0),
    \left(3 \ve_{d_1} + \ve_{d_2}\right)^{\otimes 4}
    \right>
    \\
    &+
    \gamma_{(2, 2), (1, 3)}
    \sum_{d_1=1}^D \sum_{d_2=1}^D 
    \left<
    \partial^4 \vf(\vx_0),
    \left(
    \ve_{d_1} + 3 \ve_{d_2}
    \right)^{\otimes 4}
    \right>
    \\
    &+
    \gamma_{(2, 2), (2, 2)}
    \sum_{d_1=1}^D \sum_{d_2=1}^D 
    \left<
    \partial^4 \vf(\vx_0),
    \left( 2 \ve_{d_1} +  2\ve_{d_2}\right)^{\otimes 4}
    \right>
    \Big).
    \end{aligned}
\end{equation}
Afterwards, we exploit the symmetry of the coefficients $\gamma_{(2, 2), (4, 0)} = \gamma_{(2, 2), (0, 4)}$ and $\gamma_{(2, 2), (3, 1)} = \gamma_{(2, 2), (1, 3)}$ yielding

\begin{equation} \label{eq:ttc_for_biharm_2}
    \begin{aligned}
    &\frac{1}{24} 
    \Big(
    2D\gamma_{(2, 2),(4, 0)}
    \sum_{d_1=1}^D 
    \left<
    \partial^4 \vf(\vx_0),
    \left(4 \ve_{d_1}\right)^{\otimes 4}
    \right>
    \\
    &+
    2 \gamma_{(2, 2), (3, 1)}
    \sum_{d_1=1}^D \sum_{d_2=1}^D
    \left<
    \partial^4 \vf(\vx_0),
    \left( 
    3 \ve_{d_1} + \ve_{d_2}
    \right)^{\otimes 4}
    \right>
    \\
    &+
    \gamma_{(2, 2), (2, 2)}
    \sum_{d_1=1}^D \sum_{d_2=1}^D 
    \left<
    \partial^4 \vf(\vx_0),
    \left( 2 \ve_{d_1} +  2\ve_{d_2}\right)^{\otimes 4}
    \right>
    \Big).
    \end{aligned}
\end{equation}
Since the first sum captures all diagonal directions $e_{d_1} = e_{d_2}$, we extract this from the second and third sums to further reduce the computational effort. We obtain
\begin{equation} \label{eq:ttc_for_biharm3}
    \begin{aligned}
    &\frac{1}{24} 
    \Big(
    \left(
    2D\gamma_{(2, 2), (4, 0)} + 2 \gamma_{(2, 2), (3, 1)} + \gamma_{(2, 2),(2, 2)}
    \right)
    \sum_{d_1=1}^D 
    \left<
    \partial^4 \vf(\vx_0),
    \left( 4 \ve_{d_1} \right)^{\otimes 4}
    \right>
    \\
    &+
    2 \gamma_{(2, 2),(3, 1)}
    \sum_{d_1=1}^D \sum_{\underset{d_2 \neq d_1}{d_2=1}}^D
    \left<
    \partial^4 \vf(\vx_0),
    \left( 
    3 \ve_{d_1} + \ve_{d_2}
    \right)^{\otimes 4}
    \right>
    \\
    &+
    \gamma_{(2, 2), (2, 2)}
    \sum_{d_1=1}^D \sum_{\underset{d_2 \neq d_1}{d_2 = 1}}^D 
    \left<
    \partial^4 \vf(\vx_0),
    \left( 2 \ve_{d_1} +  2\ve_{d_2}\right)^{\otimes 4}
    \right>
    \Big).
    \end{aligned}
\end{equation}
Exploiting further symmetries, one obtains
\begin{equation} \label{eq:ttc_for_biharm_final}
    \begin{aligned}
    \Delta^2 \vf(\vx_0) &=
    \frac{1}{24}
    \Big(
    \left(
    2D\gamma_{(2, 2), (4, 0)} + 2 \gamma_{(2, 2), (3, 1)} + \gamma_{(2, 2),(2, 2)}
    \right)
    \sum_{d_1=1}^D
    \left<
    \partial^4 \vf(\vx_0),
    \left(4 \ve_{d_1}\right)^{\otimes 4}
    \right>
    \\
    &+
    2 \gamma_{(2, 2), (3, 1)}
     \sum_{d_1=1}^D\sum_{\underset{d_2 \neq d_1}{d_2=1}}^D\!\!\!
    \left<
    \partial^4 \vf(\vx_0),
     \left(
    3 \ve_{d_1}+\ve_{d_2}
    \right)^{\otimes 4}
    \right>
    \\
    & +
    2 \gamma_{(2, 2), (2, 2)}
    \sum_{d_1=1}^{D - 1} \sum_{d_2 = d_1 + 1}^D
    \left<
    \partial^4 \vf(\vx_0),
   \left( 2 \ve_{d_1}+2\ve_{d_2}\right)^{ \otimes 4 }
    \right>
    \Big).
    \end{aligned}
\end{equation}


\subsection{Pedagogical Approach for the Biharmonic Operator with 6-jets}\label{sec:appendix_ttc_other_methods}

A different approach to compute arbitrary-mixed derivatives was proposed in \cite{shi2024stochastic}. This approach relies, for the biharmonic operator, on the hand-selection of certain $6$-jets to extract the required derivatives. The degree and directions for the jets are obtained by considering the Faà di Bruno formula for the 6-th coefficient $\vf_6$ (see \cref{sec:faa-di-bruno-cheatsheet}). Selecting coefficients of the input $6$-jet to $\vx_1 = \ve_{d_1}, \vx_2 = \ve_{d_2}$ and  $\vx_3 = \vx_4 = \vx_5 = \vx_6 = \vzero$ leads us to
\begin{align}\label{eq:felix-biharmonic-jet1}
\begin{split}
    \vf_6
    &=
    \left<
    \partial^{6} \vf(\vx_0), 
    \otimes_{k=1}^6 \ve_{d_1}
    \right>
    +
    15 
    \left<
    \partial^5 \vf(\vx_0), 
    \left(\ve_{d_1}\right)^{\otimes 4} \otimes \ve_{d_2}
    \right>
    \\
    &  \quad{}  +
    {\color{blue}
    45 
    \left<
    \partial^4 \vf(\vx_0),
    \left(\ve_{d_1}\right)^{\otimes 2} \otimes \left(\ve_{d_2} \right)^{\otimes 2}
    \right>
    }
+
    15 
    \left<
    \partial^3 \vf(\vx_0), 
    \otimes_{k=1}^3 \ve_{d_2}
    \right>.
    \end{split}
\end{align}
Notice the \textcolor{blue}{blue term}, which has the same structure as the summands we want to compute for the biharmonic operator. Therefore, a first $6$-jet is computed as explained above. To cancel out the unwanted terms, we evaluate another $6$-jet with the same input except $\vx_2 = -\ve_{d_2}$ and adding the $6$-th coefficient of this jet to \cref{eq:felix-biharmonic-jet1} gives
\begin{align}\label{eq:felix-biharmonic-jet2}
    2 \left<
    \partial^{6} \vf(\vx_0), 
    \otimes_{k=1}^6 \ve_{d_1}
    \right>
    +
    {\color{blue}
    90
    \left<
    \partial^4 \vf(\vx_0),
    \left(\ve_{d_1}\right)^{\otimes 2} \otimes \left(\ve_{d_2} \right)^{\otimes 2}
    \right>.
    }
\end{align}
Finally, a third $6$-jet is computing with $\vx_2 = \vzero$. The $6$-th coefficient of this jet contains only  
\begin{align}\label{eq:felix-biharmonic-jet3}
    \left<
    \partial^{6} \vf(\vx_0),
    \otimes_{k=1}^6 \vx_1
    \right>.
\end{align}
We obtain 
\begin{equation}
        {\color{blue}
    90
    \left<
    \partial^4 \vf(\vx_0),
    \left(\vx_1\right)^{\otimes 2} \otimes \left(\vx_2 \right)^{\otimes 2}
    \right>    
    }
\end{equation}
by subtracting twice of the $6$-th coefficient of the third jet from \cref{eq:felix-biharmonic-jet2}.

To summarize the procedure, we evaluate the 6-jet three times. The first jet has the input $\vx_1 = \ve_{d_1}, \vx_2 = \ve_{d_2}$ and $\vx_3 = \vx_4 = \vx_5 = \vx_6 = \vzero$, the second jet has the same input jet apart from $\vx_2 = \- \ve_{d_2}$. The third 6-jet takes $\vx_2 = \vzero$. Then we add the $6$-th coefficient of the first and the second and subtract twice of the $6$-th coefficient of the third jet. Dividing by $90$ provides the derivative corresponding to the $d_1, d_2$ term of the biharmonic operator.

The standard Taylor mode would propagate $1 + 18D^2$ vectors through every node, where we already exploit that all jets share $\vx_0$. 
Leveraging our collapsed Taylor mode would have the cost of passing $1 + 3 + 15D^2$ vectors to every node of the compute graph. Still, this is very expensive in comparison to our approach described before. In addition, until now, the selection of the jet degree and the input coefficients requires substantial human effort.



\subsection{Another Example}
To make this process more clear we further add an example in the appendix: computing $\sum_{i=1}^D\sum_{j=1}^D\frac{\partial^3}{\partial x_i^2 x_j}f(x)$.
This example is from Appendix F.2 of the stochastic Taylor derivative paper (Shi et al., 2024), which describes how to compute these 3rd-order derivatives using 7-jets.
The interpolation formula allows using multiple 3-jets instead.
We expect it to be favorable as Taylor mode scales quadratically in the derivative order and will experimentally verify that in a future version of the paper.



\paragraph{Procedure for the Example} 
The goal is to compute $\sum_{i=1}^D\sum_{j=1}^D \frac{\partial^3}{\partial x_i^2 \partial x_j} f(x)$.

0. (currently not automatic) Formulate the operator in our notation: 
    $$
    \sum_{i=1}^D\sum_{j=1}^D \langle\partial^3 f(x), e_i^{\otimes 2} \otimes e_j    \rangle
    $$
1. (automatic) Generate the interpolation coefficients $\gamma_{pq}$ where $p=(2, 1)$ and $q \in \{(3, 0), (2, 1), (1, 2), (0, 3)\}$. (We have a script for this)

    $\gamma_{(2, 1)(0, 3)} =  -8/81$
    $\gamma_{(2, 1)(1, 2)} = 16/27$
    $\gamma_{(2, 1)(2, 1)} = -16/9$
    $\gamma_{(2, 1)(3, 0)} = 32/81$

2. (automatic) Apply Equation 11
    $$
    \sum_{i=1}^D\sum_{j=1}^D \sum_{q \in \mathbb{N}^2, \; |q| = 3} \langle\partial^3 f(x), \left([q]_1 e_i + [q]_2 e_j   \right)^{\otimes 3} \rangle
    $$
Applying collapsed Taylor mode can directly applied to these $4D^2$ -- 3-jets. However, the full potential of our requires some further steps that leverages the structure. 

3. (currently not automatic) The sums for $\gamma_{(2, 1)(3, 0)}$ and $\gamma_{(2, 1)(0, 3)}$ are similar. As well as for $\gamma_{(2, 1)(2, 1)}$ and $\gamma_{(2, 1)(1, 2)}$. We only have $2D^2$-- 3 jets:
    $(\gamma_{(2, 1)(3, 0)} + \gamma_{(2, 1)(0, 3)})
    \sum_{i=1}^D\sum_{j=1}^D  \langle\partial^3 f(x), \left(3 e_i   \right)^{\otimes 3} \rangle$ 
    $+(\gamma_{(2, 1)(2, 1)} + \gamma_{(2, 1)(1, 2)})
    \sum_{i=1}^D\sum_{j=1}^D  \langle\partial^3 f(x), \left(2 e_i + e_j       \right)^{\otimes 3} \rangle$

We further observe, that the first summation is independent of $j$:
$(\gamma_{(2, 1)(3, 0)} + \gamma_{(2, 1)(0, 3)})D
    \sum_{i=1}^D \langle\partial^3 f(x), \left(3 e_i   \right)^{\otimes 3} \rangle$ <br>
    $+(\gamma_{(2, 1)(2, 1)} + \gamma_{(2, 1)(1, 2)})
    \sum_{i=1}^D\sum_{j=1}^D  \langle\partial^3 f(x), \left(2 e_i + e_j       \right)^{\otimes 3} \rangle$

Remove the case $i=j$ from the last term gives our final form
$((\gamma_{(2, 1)(3, 0)}D + \gamma_{(2, 1)(0, 3)}D + \gamma_{(2, 1)(2, 1)} + \gamma_{(2, 1)(1, 2)})
    \sum_{i=1}^D \langle\partial^3 f(x), \left(3 e_i   \right)^{\otimes 3} \rangle$ <br>
    $+(\gamma_{(2, 1)(2, 1)} + \gamma_{(2, 1)(1, 2)})
    \sum_{i=1}^D\sum_{j=1, j \neq i}^D  \langle\partial^3 f(x), \left(2 e_i + e_j       \right)^{\otimes 3} \rangle$


This optimized version required $D^2 3-jets$ that can be further collapsed.


\section{JAX benchmark}
\label{sec:jax-benchmark}
This section presents experiments that show that the graph simplifications we propose to collapse standard Taylor mode are currently not applied by the \texttt{jit} compiler in JAX.

\begin{figure*}[!t]
  \centering

  \captionof{table}{\textbf{JAX Benchmark from \Cref{fig:jax-benchmark} in numbers.}
    We fit linear functions and report their slopes, \ie how much run time and memory increase when incrementing the batch size.
    % Our collapsed Taylor mode is up to two times faster than nested first-order autodiff, while using 80\% of memory in the differentiable, and 70\% in the non-differentiable, setting.
    All numbers are shown with two significant digits and bold values are best according to parenthesized values.}
  \label{tab:jax-benchmark}
  \vspace{1.5ex}
  % paths where the performances are stored
  \def\datapathJAXLaplacianExact{../jet/exp/exp04_jax_benchmark/performance/architecture_tanh_mlp_768_768_512_512_1_device_cuda_dim_50_name_jax_laplacian_vary_batch_size}
  \def\datapathJAXBilaplacianExact{../jet/exp/exp04_jax_benchmark/performance/architecture_tanh_mlp_768_768_512_512_1_device_cuda_dim_5_name_jax_bilaplacian_vary_batch_size}
  % configuration options for the \num command
  \sisetup{%
    % scientific-notation=true,%
    round-mode=figures,%
    round-precision=2,%
    detect-weight, % for bolding to work
    tight-spacing=true, % less space around \cdot
  }
  % temporarily overwrite the \num command to process nan's
  \let\origsiunitxnum\num
  % Redefine \num to check for non-numeric strings
  \renewcommand{\num}[1]{\IfStrEq{#1}{nan}{\text{n/a} }{\origsiunitxnum{#1}}}
  \begin{tabular}{ccc|cc}
    \toprule
    \textbf{Mode}
    & \textbf{Per-datum cost}
    & \textbf{Implementation}
    & \textbf{Laplacian}
    & \textbf{Bi-harmonic}
    \\
    \midrule
    \multirow{9}{*}{\textbf{Exact}}
    & \multirow{3}{*}{Time [ms]}
    & \textcolor{tab:blue}{Nested first-order}
    & \input{\datapathJAXLaplacianExact/hessian_trace_best.txt}
    & \input{\datapathJAXBilaplacianExact/hessian_trace_best.txt}
    \\
    &
    & \textcolor{tab:orange}{Standard Taylor}
    & \input{\datapathJAXLaplacianExact/jet_naive_best.txt}
    & \input{\datapathJAXBilaplacianExact/jet_naive_best.txt}
    \\
    &
    & \textcolor{tab:green}{Collapsed (ours)}
    & \textbf{\input{\datapathJAXLaplacianExact/jet_simplified_best.txt}}
    & \textbf{\input{\datapathJAXBilaplacianExact/jet_simplified_best.txt}}
    \\ \cmidrule{2-5}
    & \multirow{3}{*}{\makecell{Mem.\,[MiB] \\ (differentiable)}}
    & \textcolor{tab:blue}{Nested first-order}
    & \input{\datapathJAXLaplacianExact/hessian_trace_peakmem.txt}
    & \textbf{\input{\datapathJAXBilaplacianExact/hessian_trace_peakmem.txt}}
    \\
    &
    & \textcolor{tab:orange}{Standard Taylor}
    & \input{\datapathJAXLaplacianExact/jet_naive_peakmem.txt}
    & \input{\datapathJAXBilaplacianExact/jet_naive_peakmem.txt}
    \\
    &
    & \textcolor{tab:green}{Collapsed (ours)}
    & \textbf{\input{\datapathJAXLaplacianExact/jet_simplified_peakmem.txt}}
    & \input{\datapathJAXBilaplacianExact/jet_simplified_peakmem.txt}
    \\ \cmidrule{2-5}
    & \multirow{3}{*}{\makecell{Mem.\,[MiB] \\ (non-diff.)}}
    & \textcolor{tab:blue}{Nested first-order}
    & \textbf{\input{\datapathJAXLaplacianExact/hessian_trace_peakmem_nondifferentiable.txt}}
    & \input{\datapathJAXBilaplacianExact/hessian_trace_peakmem_nondifferentiable.txt}
    \\
    &
    & \textcolor{tab:orange}{Standard Taylor}
    & \textbf{\input{\datapathJAXLaplacianExact/jet_naive_peakmem_nondifferentiable.txt}}
    & \input{\datapathJAXBilaplacianExact/jet_naive_peakmem_nondifferentiable.txt}
    \\
    &
    & \textcolor{tab:green}{Collapsed (ours)}
    & \input{\datapathJAXLaplacianExact/jet_simplified_peakmem_nondifferentiable.txt}
    & \textbf{\input{\datapathJAXBilaplacianExact/jet_simplified_peakmem_nondifferentiable.txt}}
    \\
    \bottomrule
  \end{tabular}
  % Re-set the \num command to the original one
  \let\num\origsiunitxnum

  \vspace{4ex}

  % From https://tex.stackexchange.com/a/7318
  \newcolumntype{C}{ >{\centering\arraybackslash} m{0.12\textwidth} }
  \newcolumntype{D}{ >{\centering\arraybackslash} m{0.42\textwidth} }
  \begin{tabular}{CDD}
    & \textbf{Laplacian $(D=50)$}
    & \textbf{Bi-harmonic $(D=5)$}
    \\
    \textbf{Exact}
    & \includegraphics{../jet/exp/exp04_jax_benchmark/figures/architecture_tanh_mlp_768_768_512_512_1_device_cuda_dim_50_name_jax_laplacian_vary_batch_size.pdf}
    & \includegraphics{../jet/exp/exp04_jax_benchmark/figures/architecture_tanh_mlp_768_768_512_512_1_device_cuda_dim_5_name_jax_bilaplacian_vary_batch_size.pdf}
  \end{tabular}
  \caption{\textbf{JAX's \texttt{jit} compiler does not apply our graph simplifications to standard Taylor mode.} Colors: \textcolor{tab:green}{Collapsed Taylor mode}, \textcolor{tab:orange}{standard Taylor mode}, and \textcolor{tab:blue}{nested first-order automatic differentiaion}, \textcolor{black!50!white}{opaque} memory consumptions are for non-differentiable computations.
    Results are on GPU and we use a $D \to 768 \to 768 \to 512 \to 512 \to 1$ MLP with tanh activations with $D=50$ for the Laplacians and $D=5$ for the Bi-harmonic operator, varying the batch size.
    For each approach, we fit a line to the data and report the slope in \Cref{tab:jax-benchmark}  to quantify the relative speedup and memory reduction.
  }
  \label{fig:jax-benchmark}
\end{figure*}

\paragraph{Comparing Laplacian implementations in JAX.} Similar to our PyTorch experiment in \Cref{sec:experiments}, we compare three implementations of the Laplacian in JAX (all compiled with \texttt{jax.jit}):

\begin{enumerate}
\item \textbf{\textcolor{tab:blue}{Nested first-order autodiff}} computes the Hessian using \texttt{jax.hessian}, which relies on forward-over-reverse mode, then traces it.

\item \textbf{\textcolor{tab:orange}{Standard Taylor mode}} propagates multiple uni-variate Taylor polynomials, each of which computes one element of the Hessian diagonal, then sums them to obtain the Laplacian.
  This is implemented with \texttt{jax.experimental.jet.jet} and \texttt{jax.vmap}.

\item \textbf{\textcolor{tab:green}{Collapsed Taylor mode}} relies on the forward Laplacian implementation in JAX provided by the \texttt{folx} library \cite{gao2023folx} and implements our proposed collapsed Taylor mode for the specific case of the Laplacian.
  \texttt{folx} also enables leveraging sparsity in the tensors, which is beneficial for architectures in VMC.
  To disentangle run time improvements from sparsity detection versus collapsing Taylor coefficient, we disable \texttt{folx}'s sparsity detection.
\end{enumerate}

We only investigate computing the exact Laplacian, as the forward Laplacian in \texttt{folx} currently does not support stochastic computation.
We use the same neural network architecture as for our PyTorch experiments, fix the input dimension to $D=50$ and vary the batch size, recording the run time and peak memory with the same protocol as described in the main text.
JAX is purely functional and therefore does not have a mechanism to build up a differentiable compute graph similar to evaluating a function in PyTorch where some leafs have \texttt{requires\_grad=True}.
To approximate the peak memory of computing a differentiable Laplacian in JAX, we measure the peak memory of first computing the Laplacian, then evaluating the gradient \wrt the neural network's parameters which backpropagates through the same computation graph built by PyTorch.
Doing so, we encountered a bug when trying to \texttt{jit}-compile the \texttt{folx} implementation.
We believe the solid \textcolor{tab:green}{green} lines in the memory consumption, \ie peak memory of differentiating through differentiable operators, could further be improved by making the computation compile-able.

The left column of \Cref{fig:jax-benchmark} visualizes the performance of the three implementations.
We fit linear functions to each of them and report the cost incurred by adding one more datum to the batch in \Cref{tab:jax-benchmark}.
From them, we draw the following conclusions:

\begin{enumerate}
\item \textbf{Performance is consistent between PyTorch and JAX.} Although our PyTorch implementation does not leverage compilation, the values reported in \Cref{tab:benchmark,tab:jax-benchmark} are consistent and differ by at most a factor of two (in rare cases).
  This confirms that our PyTorch-based implementation of Taylor mode is reasonably efficient, and that the presented performance results in the main text are transferable to other frameworks like JAX.

\item \textbf{Our implementation of collapsed Taylor mode based on graph rewrites in PyTorch achieves consistent speed-up with the Laplacian-specific implementation in JAX.}
  Specifically, we observe that \textcolor{tab:green}{collapsed Taylor mode/forward Laplacian} use roughly half the run time of \textcolor{tab:blue}{nested first-order autodiff} (compare \Cref{tab:benchmark,tab:jax-benchmark}).
  This supports our argument that our collapsed Taylor is indeed a generalization of the forward Laplacian, \ie the latter does not employ additional tricks (leveraging sparsity could also be applied to our approach but are not aware of a drop-in implementation).
  It also illustrates that the savings we report in PyTorch carry over to other frameworks like JAX.

\item \textbf{JAX's \texttt{jit} compiler is unable to apply the graph rewrites we propose in this work.}
  If the JAX compiler was able to perform our proposed graph rewrites, then the \texttt{jit}-compiled \textcolor{tab:orange}{standard Taylor mode} should yield similar performance than the \textcolor{tab:green}{forward Laplacian}.
  However, we observe a clear performance gap in run time and memory, from which we conclude that the compilation did not collapse the Taylor coefficients.
  Our contribution is to point out that such rewrites could easily be added to the compiler's ability to unlock these performance gains at zero user overhead.

\end{enumerate}



%%% Local Variables:
%%% mode: LaTeX
%%% TeX-master: "../main"
%%% End:


% \input{sections/appendix.tex}
% \input{sections/checklist.tex}

\end{document}
%%% Local Variables:
%%% mode: latex
%%% TeX-master: t
%%% End:
